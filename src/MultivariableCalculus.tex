% arara: xelatex: { shell : yes }
% arara: xelatex: { shell : yes }

\documentclass[letter, 11pt, onesided]{memoir}
%\documentclass[letter, 11pt, twosided, openright]{memoir}

\usepackage[no-math]{fontspec}
\usepackage{xpatch}
	\renewcommand{\ttdefault}{ul9}
	\xpatchcmd{\ttfamily}{\selectfont}{\fontencoding{T1}\selectfont}{}{}
	\DeclareTextCommand{\nobreakspace}{T1}{\leavevmode\nobreak\ }
\usepackage{polyglossia} % English please
	\setdefaultlanguage[variant=us]{english}
\usepackage[charter,cal=cmcal]{mathdesign} %different font
\usepackage{avant}
\usepackage{microtype} % Less badboxes

\usepackage{calc, ifthen, xparse}
\usepackage{makeidx}
\usepackage[hidelinks]{hyperref}   % Internal hyperlinks
\usepackage{mathtools} % replaces amsmath
\usepackage{etoolbox}
\usepackage{amsthm, bm}
\usepackage{graphicx}
\usepackage{xcolor}
\usepackage{tikz}
	\tikzset{>=latex}
	\usetikzlibrary{calc}
	\usetikzlibrary{backgrounds}
	\usetikzlibrary{patterns,decorations.pathreplacing}
	\usetikzlibrary{spy}
\usepackage{pgfplots}
	\pgfplotsset{compat=1.12}
	\usepgfplotslibrary{colormaps}
	\usepgfplotslibrary{patchplots}
	\usepgfplotslibrary{fillbetween}
\usepackage[inline,shortlabels]{enumitem}
\setlistdepth{9}

\usepackage{datatool}


\usepgfplotslibrary{external}
\tikzexternalize[prefix=tikz/]
%\tikzexternalize[mode=list and make]


%%% Specify the cool macros
\newcommand{\declarecommand}[1]{\providecommand{#1}{}\renewcommand{#1}}
\declarecommand{\R}{\mathbb{R}}  % we don't care if it's already defined.  We really want *this* command!
\declarecommand{\Z}{\mathbb{Z}}
\declarecommand{\Q}{\mathbb{Q}}
\declarecommand{\N}{\mathbb{N}}
\declarecommand{\C}{\mathbb{C}}
\declarecommand{\d}{\mathrm{d}}
\DeclareMathOperator{\Span}{span}
\DeclareMathOperator{\Img}{img}
\DeclareMathOperator{\Id}{id}
\DeclareMathOperator{\Range}{range}
\DeclareMathOperator{\Rref}{rref}
\DeclareMathOperator{\Rank}{rank}
\DeclareMathOperator{\Speed}{speed}
\DeclareMathOperator{\Vel}{velocity}
\DeclareMathOperator{\Accel}{accel}
\DeclareMathOperator{\Null}{null}
\DeclareMathOperator{\Nullity}{nullity}
\DeclareMathOperator{\Char}{char}
\DeclareMathOperator{\Proj}{proj}
\DeclareMathOperator{\Perp}{perp}
\DeclareMathOperator{\Arclen}{arc\,len}
\newcommand{\Arclenfrom}[3]{\Arclen #1 \Big|_{#2}^{#3}}
\newcommand{\proj}{\Proj}
\newcommand{\xhat}{{\hat {\mathbf x}}}
\newcommand{\yhat}{{\hat {\mathbf y}}}
\newcommand{\zhat}{{\hat {\mathbf z}}}
\newcommand{\mat}[1]{\begin{bmatrix*}[r]#1\end{bmatrix*}}
\newcommand{\matc}[1]{\begin{bmatrix}#1\end{bmatrix}}
\DeclarePairedDelimiter\abs{\lvert}{\rvert}
\DeclarePairedDelimiter\norm{\lVert}{\rVert}
% just to make sure it exists
\providecommand\given{}
% can be useful to refer to this outside \Set
\newcommand\SetSymbol[1][]{%
	\nonscript\::%
	\allowbreak
	\nonscript\:
	\mathopen{}}
\DeclarePairedDelimiterX\Set[1]\{\}{%
	\renewcommand\given{\SetSymbol[\delimsize]}
	#1
}
\renewcommand{\it}{\itshape}
% footnote without marker
\newcommand\blfootnote[1]{%
  \begingroup
  \renewcommand\thefootnote{}\footnote{#1}%
  \addtocounter{footnote}{-1}%
  \endgroup
}


% set up some deep enumerate environments for the CC license
\newlist{ccEnumerate}{enumerate}{9}
\setlist[ccEnumerate,1]{label=\alph*.}
\setlist[ccEnumerate,2]{label=\arabic*.}
\setlist[ccEnumerate,3]{label=\Alph*.}
\setlist[ccEnumerate,4]{label=\roman*.}
\setlist[ccEnumerate,5]{label=(\alph*)}
\setlist[ccEnumerate,6]{label=(\arabic*)}
\setlist[ccEnumerate,7]{label=(\Roman*)}
\setlist[ccEnumerate,8]{label=(\Alph*)}
\setlist[ccEnumerate,9]{label=(\roman*)}


%%% Specify the colors
\definecolor{myorange}{HTML}{F29B23}
\definecolor{myred}{HTML}{D13409}
\definecolor{mypink}{HTML}{B3094F}
\definecolor{mydark}{HTML}{23112A}

\colorlet{chapcolor}{myred}
\colorlet{seccolor}{myorange}
\colorlet{headertext}{myred!70!white}
\colorlet{ocre}{mypink}

%%% Specify custom fonts
%\newfontfamily{\sfreg}{Fira Sans}
%\newfontfamily{\firasans}
%  [Ligatures=TeX, % recommended
%   UprightFont={* Light},
%   ItalicFont={* Light Italic},
%   BoldFont={* Medium},
%   BoldItalicFont={* Medium Italic}]
%  {Fira Sans}



%%% PAGE LAYOUT 
%%%------------------------------------------------------------------------------

\setlrmarginsandblock{0.15\paperwidth}{*}{1} % Left and right margin
\setulmarginsandblock{0.15\paperwidth}{*}{1}  % Upper and lower margin
\checkandfixthelayout

%%% SECTIONAL DIVISIONS
%%%------------------------------------------------------------------------------

%\maxsecnumdepth{subsection} % Subsections (and higher) are numbered
%\setsecnumdepth{subsection}
%
%\makeatletter %
%\makechapterstyle{standard}{
%  \setlength{\beforechapskip}{0\baselineskip}
%  \setlength{\midchapskip}{1\baselineskip}
%  \setlength{\afterchapskip}{8\baselineskip}
%  \renewcommand{\chapterheadstart}{\vspace*{\beforechapskip}}
%  \renewcommand{\chapnamefont}{\centering\normalfont\Large}
%  \renewcommand{\printchaptername}{\chapnamefont \@chapapp}
%  \renewcommand{\chapternamenum}{\space}
%  \renewcommand{\chapnumfont}{\normalfont\Large}
%  \renewcommand{\printchapternum}{\chapnumfont \thechapter}
%  \renewcommand{\afterchapternum}{\par\nobreak\vskip \midchapskip}
%  \renewcommand{\printchapternonum}{\vspace*{\midchapskip}\vspace*{5mm}}
%  \renewcommand{\chaptitlefont}{\centering\bfseries\LARGE}
%  \renewcommand{\printchaptertitle}[1]{\chaptitlefont ##1}
%  \renewcommand{\afterchaptertitle}{\par\nobreak\vskip \afterchapskip}
%}
%\makeatother
%
%\chapterstyle{standard}

%\setsecheadstyle{\normalfont\large\bfseries}
%\setsubsecheadstyle{\normalfont\normalsize\bfseries}
%\setparaheadstyle{\normalfont\normalsize\bfseries}
%\setparaindent{0pt}\setafterparaskip{0pt}

\renewcommand{\chapnumfont}{\normalfont\huge\sffamily\bfseries}
\renewcommand{\chaptitlefont}{\color{chapcolor}\chapnumfont\mdseries}
\renewcommand{\printchaptername}{\chapnumfont Chapter}
\makeatletter
	\renewcommand{\hangsecnum}{%
	  \def\@seccntformat##1{%
	    \makebox[0pt][r]{%
	      \color{seccolor}%
	      \csname the##1\endcsname
	      \quad
	    }%
	  }%
	}
\makeatother
%\setsecnumformat{\llap{\color{seccolor}\csname the#1\endcsname\quad}}

\setsecheadstyle{\Large\bfseries\sffamily\raggedright}
\setsubsecheadstyle{\large\sffamily\raggedright}
\setsubsubsecheadstyle{\normalsize\sffamily\raggedright}
\setsechook{\hangsecnum}
\setsubsechook{\defaultsecnum}
\setsubsubsechook{\defaultsecnum}


\makeatletter
	\setlength{\headwidth}{\textwidth}
	%\addtolength{\headwidth}{\marginparsep}
	%\addtolength{\headwidth}{\marginparwidth}
	\makepagestyle{serifpage}
	\makerunningwidth{serifpage}{\headwidth}
	\makeheadrule{serifpage}{\headwidth}{\normalrulethickness}
	\makeheadposition{serifpage}{flushright}{flushleft}{}{}
	\makepsmarks{serifpage}{%
		\let\@mkboth\markboth
		\def\chaptermark##1{\markboth{\textsf{\color{headertext}##1}}{\textsf{\color{headertext}##1}}}% % left & right marks
		\def\sectionmark##1{\markright{%
		\ifnum \c@secnumdepth>\z@
			{\color{seccolor}\sffamily\thesection\ }
		\fi
		\textsf{\color{headertext}##1}}}
	}
	\makeevenhead{serifpage}%
		{\normalfont\sffamily\thepage}{}{\normalfont\sffamily \leftmark}
	\makeoddhead{serifpage}%
		{\normalfont\sffamily \rightmark}{}{\normalfont\sffamily\thepage}
\makeatother

\pagestyle{serifpage}

%%% Theorem Environments

\makeatletter
	\newcommand{\intoo}[2]{\mathopen{]}#1\,;#2\mathclose{[}}
	\newcommand{\ud}{\mathop{\mathrm{{}d}}\mathopen{}}
	\newcommand{\intff}[2]{\mathopen{[}#1\,;#2\mathclose{]}}
	\newtheorem{notation}{Notation}[chapter]

	%%%%%%%%%%%%%%%%%%%%%%%%%%%%%%%%%%%%%%%%%%%%%%%%%%%%%%%%%%%%%%%%%%%%%%%%%%%
	%%%%%%%%%%%%%%%%%%%% dedicated to boxed/framed environements %%%%%%%%%%%%%%
	%%%%%%%%%%%%%%%%%%%%%%%%%%%%%%%%%%%%%%%%%%%%%%%%%%%%%%%%%%%%%%%%%%%%%%%%%%%
	\newtheoremstyle{orangenumbox}% % Theorem style name
	{0pt}% Space above
	{0pt}% Space below
	{\normalfont}% % Body font
	{}% Indent amount
	{\small\bfseries\sffamily\color{myorange}}% % Theorem head font
	{\;}% Punctuation after theorem head
	{0.25em}% Space after theorem head
	{\small\sffamily\color{myorange!80!black}\thmname{#1}\nobreakspace\thmnumber{\@ifnotempty{#1}{}\@upn{#2}}% Theorem text (e.g. Theorem 2.1)
	\thmnote{\nobreakspace\the\thm@notefont\sffamily\bfseries\color{black}---\nobreakspace#3.}} % Optional theorem note
	\renewcommand{\qedsymbol}{$\blacksquare$}% Optional qed square

	\newtheoremstyle{ocrenumbox}% % Theorem style name
	{0pt}% Space above
	{0pt}% Space below
	{\normalfont}% % Body font
	{}% Indent amount
	{\small\bfseries\sffamily\color{ocre}}% % Theorem head font
	{\;}% Punctuation after theorem head
	{0.25em}% Space after theorem head
	{\small\sffamily\color{ocre}\thmname{#1}\nobreakspace\thmnumber{\@ifnotempty{#1}{}\@upn{#2}}% Theorem text (e.g. Theorem 2.1)
	\thmnote{\nobreakspace\the\thm@notefont\sffamily\bfseries\color{black}---\nobreakspace#3.}} % Optional theorem note
	\renewcommand{\qedsymbol}{$\blacksquare$}% Optional qed square

	\newtheoremstyle{blacknumex}% Theorem style name
	{5pt}% Space above
	{5pt}% Space below
	{\normalfont}% Body font
	{} % Indent amount
	{\small\bfseries\sffamily}% Theorem head font
	{\;}% Punctuation after theorem head
	{0.25em}% Space after theorem head
	{\small\sffamily{\tiny\ensuremath{\blacksquare}}\nobreakspace\thmname{#1}\nobreakspace\thmnumber{\@ifnotempty{#1}{}\@upn{#2}}% Theorem text (e.g. Theorem 2.1)
	\thmnote{\nobreakspace\the\thm@notefont\sffamily\bfseries---\nobreakspace#3.}}% Optional theorem note

	\newtheoremstyle{blacknumbox} % Theorem style name
	{0pt}% Space above
	{0pt}% Space below
	{\normalfont}% Body font
	{}% Indent amount
	{\small\bfseries\sffamily}% Theorem head font
	{\;}% Punctuation after theorem head
	{0.25em}% Space after theorem head
	{\small\sffamily\thmname{#1}\nobreakspace\thmnumber{\@ifnotempty{#1}{}\@upn{#2}}% Theorem text (e.g. Theorem 2.1)
	\thmnote{\nobreakspace\the\thm@notefont\sffamily\bfseries---\nobreakspace#3.}}% Optional theorem note

	%%%%%%%%%%%%%%%%%%%%%%%%%%%%%%%%%%%%%%%%%%%%%%%%%%%%%%%%%%%%%%%%%%%%%%%%%%%
	%%%%%%%%%%%%% dedicated to non-boxed/non-framed environements %%%%%%%%%%%%%
	%%%%%%%%%%%%%%%%%%%%%%%%%%%%%%%%%%%%%%%%%%%%%%%%%%%%%%%%%%%%%%%%%%%%%%%%%%%
	\newtheoremstyle{ocrenum}% % Theorem style name
	{5pt}% Space above
	{5pt}% Space below
	{\normalfont}% % Body font
	{}% Indent amount
	{\small\bfseries\sffamily\color{ocre}}% % Theorem head font
	{\;}% Punctuation after theorem head
	{0.25em}% Space after theorem head
	{\small\sffamily\color{ocre}\thmname{#1}\nobreakspace\thmnumber{\@ifnotempty{#1}{}\@upn{#2}}% Theorem text (e.g. Theorem 2.1)
	\thmnote{\nobreakspace\the\thm@notefont\sffamily\bfseries\color{black}---\nobreakspace#3.}} % Optional theorem note
	\renewcommand{\qedsymbol}{$\blacksquare$}% Optional qed square
	\makeatother

	% Defines the theorem text style for each type of theorem to one of the three styles above
	\newcounter{dummy} 
	\numberwithin{dummy}{section}
	\theoremstyle{orangenumbox}
	\newtheorem{theoremeT}[dummy]{Theorem}
	\theoremstyle{ocrenumbox}
	\newtheorem{problem}{Problem}[chapter]
	\newtheorem{exerciseT}{Exercise}[chapter]
	\theoremstyle{blacknumex}
	\newtheorem{exampleT}{Example}[chapter]
	\theoremstyle{blacknumbox}
	\newtheorem{vocabulary}{Vocabulary}[chapter]
	\newtheorem{definitionT}{Definition}[section]
	\newtheorem{corollaryT}[dummy]{Corollary}
	\theoremstyle{ocrenum}
	\newtheorem{proposition}[dummy]{Proposition}

	%----------------------------------------------------------------------------------------
	%	DEFINITION OF COLORED BOXES
	%----------------------------------------------------------------------------------------

	\RequirePackage[framemethod=default]{mdframed} % Required for creating the theorem, definition, exercise and corollary boxes

	% Theorem box
	\newmdenv[skipabove=7pt,
	skipbelow=7pt,
	rightline=false,
	leftline=true,
	topline=false,
	bottomline=false,
	backgroundcolor=myorange!20,
	linecolor=myorange,
	innerleftmargin=5pt,
	innerrightmargin=5pt,
	innertopmargin=5pt,
	innerbottommargin=5pt,
	leftmargin=0cm,
	rightmargin=0cm,
	linewidth=4pt]{tBox}	

	% Exercise box	  
	\newmdenv[skipabove=7pt,
	skipbelow=7pt,
	rightline=false,
	leftline=true,
	topline=false,
	bottomline=false,
	backgroundcolor=ocre!10,
	linecolor=ocre,
	innerleftmargin=5pt,
	innerrightmargin=5pt,
	innertopmargin=5pt,
	innerbottommargin=5pt,
	leftmargin=0cm,
	rightmargin=0cm,
	linewidth=4pt]{eBox}	

	% Definition box
	\newmdenv[skipabove=7pt,
	skipbelow=7pt,
	rightline=false,
	leftline=true,
	topline=false,
	bottomline=false,
	linecolor=ocre,
	innerleftmargin=5pt,
	innerrightmargin=5pt,
	innertopmargin=0pt,
	leftmargin=0cm,
	rightmargin=0cm,
	linewidth=4pt,
	innerbottommargin=0pt]{dBox}	

	% Corollary box
	\newmdenv[skipabove=7pt,
	skipbelow=7pt,
	rightline=false,
	leftline=true,
	topline=false,
	bottomline=false,
	linecolor=gray,
	backgroundcolor=black!5,
	innerleftmargin=5pt,
	innerrightmargin=5pt,
	innertopmargin=5pt,
	leftmargin=0cm,
	rightmargin=0cm,
	linewidth=4pt,
	innerbottommargin=5pt]{cBox}
\makeatother

% Creates an environment for each type of theorem and assigns it a theorem text style from the "Theorem Styles" section above and a colored box from above
\newenvironment{theorem}{\begin{tBox}\begin{theoremeT}}{\end{theoremeT}\end{tBox}}
\newenvironment{exercise}{\begin{eBox}\begin{exerciseT}}{\end{exerciseT}\end{eBox}}				  
%\newenvironment{exercise}{\begin{eBox}\begin{exerciseT}}{\hfill{\color{ocre}\tiny\ensuremath{\blacksquare}}\end{exerciseT}\end{eBox}}				  
\newenvironment{definition}{\begin{dBox}\begin{definitionT}}{\end{definitionT}\end{dBox}}	
\newenvironment{example}{\begin{exampleT}}{\hfill{\tiny\ensuremath{\blacksquare}}\end{exampleT}}		
\newenvironment{corollary}{\begin{cBox}\begin{corollaryT}}{\end{corollaryT}\end{cBox}}	

\newenvironment{exercises}{{\subsection*{Exercises for \thesection}}}{}
\newenvironment{openstaxexercises}{{\subsection*{Exercises from Openstax for \thesection}}}{}
\newenvironment{answer}{\begin{quote}}{\end{quote}}





%%% THE DOCUMENT
%%% Where all the important stuff is included!
%%%-------------------------------------------------------------------------------

\author{Jason Siefken}
\title{Multivariable Calculus}


\renewcommand{\maketitlehooka}{\color{chapcolor}\chapnumfont\Huge}
\renewcommand{\maketitlehookb}{\color{black}\normalfont}
\renewcommand{\maketitlehookc}{\small
	\begin{center}
		\includegraphics[width=6in]{resources/cover-render-lines-lowres.png}
	\end{center}
}
\renewcommand{\maketitlehookd}{
	\begin{center}
		\includegraphics[width=1in]{resources/doclicense-CC-by-sa.pdf}
	\end{center}
}
\usepackage{lipsum} % Just to put in some text

\begin{document}

% no numbers on chapters here, etc.  it's the front matter!
\frontmatter
\begin{titlingpage}
	\calccentering{\unitlength}
	\begin{adjustwidth*}{\unitlength}{-\unitlength}
		\maketitle
	\end{adjustwidth*}
\end{titlingpage}

\clearpage

\tableofcontents*
\clearpage

\chapter{Introduction}
	
\emph{Multivariable Calculus} approaches the subject from a mathematical,
but not overly technical, perspective.  The key idea of calculus---chop
things into little pieces and put them together again---is emphasized
throughout.

\chapter{Licensing}
	This book would not be possible without the long tradition
of mathematical inquiry that came before.  And like the 
ideas of mathematics, which are free for all to re-imagine,
re-use, and re-purpose, so too is this book.

This book is licensed under the Creative Commons By-Attribution
Share Alike 4.0 International license.  This gives you permission
to reuse, redistribute, and modify the contents of this book provided
you attribute a derived work appropriately and that you
license a derived work under the same terms.  For the fulltext
of the license, see Appendix \ref{APPENDIX-license}.

This book is a derived work of the Creative-Commons licensed 
\emph{ISP Mathematics 281} multivariable calculus textbook by 
Leonard Evans of Northwestern University.

\chapter{Contributors}
	\input{chapters/contributors.tex}
% number the chapters and such
\mainmatter

\chapter{Preliminaries}
	\section{Mathematical Notation}
	Mathematics is a sophisticated and precise language, and
	we best not adventure into calculus without learning some
	basic words.

	The most basic mathematical word is that of a \emph{set}\index{set}.  
	A set is an unordered collection of distinct objects.  We won't try and pin
	it down more exactly than this---our intuition about collections
	of objects will suffice\footnote{ When you pursue more rigorous math,
	you rely on definitions to get yourself out of philosophical jams.  For instance,
	with our definition of set, consider ``the set of all sets that don't
	contain themselves.''  Such a set cannot exist!
	This is called \emph{Russel's Paradox}, and shows
	that if we start talking about sets of sets, we may need more than
	intuition.}. We write a set with curly-braces $\{$ and $\}$ and
	list the objects inside.  For instance
	\[
		\Set{1,2,3}.
	\]
	This would be read aloud as ``the set containing the elements $1$, $2$, and $3$.''
	The symbol $\in$\index{$\in$} is used to specify that some object is an element of a set, and
	$\notin$ is used to specify it is not.  For example,
	\[
		3\in\Set{1,2,3}\qquad 4\notin\Set{1,2,3}.
	\]
	Sets can contain mixtures of objects, including other sets.  For example,
	\[
		\Set{1,2,a,\Set{-70,\infty}, x}
	\]
	is a perfectly valid set.

	It is tradition to use capital letters to name sets.  So we might say $A=\{6,7,12\}$
	or $X=\{7\}$.  There is, however, a special set with a special name---the
	empty set.  The \emph{empty set} is the set containing no elements
	and is written $\emptyset$ or $\Set{}$.  Note that $\Set{\emptyset}$ is \emph{not}
	the empty set.  It is the set containing the empty set!  It is also traditional
	to call elements of a set \emph{points}\index{point} regardless of whether you
	consider them ``point-like'' objects.

	\subsection{Operations on Sets}
	If the set $A$ contains all the elements that the set $B$ does, we call $A$ a \emph{superset}\index{superset}
	and $B$ a \emph{subset}\index{subset}.  We'll give this a formal
	definition.
	\begin{definition}[Subset \& Superset]
		The set $B$ is a \emph{subset} of the set $A$, written $B\subseteq A$, if for all
		$b\in B$ we also have $b\in A$.  In this case, $A$ is called a \emph{superset}
		of $B$.\footnote{
			Some mathematicians use the symbol $\subset$ instead of $\subseteq$.}
	\end{definition}

	Some simple examples are $\Set{1,2,3}\subseteq \Set{1,2,3,4}$ and $\Set{1,2,3}\subseteq\Set{1,2,3}$.
	There's something funny about that last example, though.  Those two sets are not only subsets/supersets
	of each other, they're \emph{equal}.  As surprising as it seems, we actually need to define
	what it means for two sets to be equal.
	\begin{definition}[Set Equality]
		The sets $A$ and $B$ are \emph{equal}, written $A=B$, if $A\subseteq B$ and $B\subseteq A$.
	\end{definition}
	Having a definition of equality to lean on will help us when we need to prove things about sets.

	\begin{example}
		Let $A$ be the set of numbers that can be expressed
		as $2n$ for some whole number $n$, and let $B$ be the
		set of numbers that can be expressed as $m+1$ where $m$ is
		an odd whole number.  We will show $A=B$.

		First, let us show $A\subseteq B$.  If $x\in A$ then $x=2n$
		for some whole number $n$.  Therefore $x=2n=2(n-1)+1+1=m+1$ where
		$m=2(n-1)+1$ is, by definition, an odd number.  Therefore $x\in B$.

		Now we will show $B\subseteq A$.  Let $x\in B$.  By definition,
		$x=m+1$ for some odd $m$ and so by the definition of oddness, $m=2k+1$
		for some whole number $k$.  Thus 
		\begin{align*}
			x=m+1&=(2k+1)+1=2k+2\\
			&=2(k+1)=2n,
		\end{align*} where $n=k+1$, and so $x\in A$.  Since $A\subseteq B$
		and $B\subseteq A$, by definition $A=B$.
	\end{example}
	

	\subsubsection{Set-builder Notation}
	Specifying sets by listing all their elements can be a hassle, and if there are an infinite
	number of elements, it's impossible!  Fortunately, \emph{set-builder notation}\index{set-builder notation}
	solves these problems.
	If $X$ is a set, we can define a subset 
	\[
		Y= \Set{a\in X\given\text{some rule involving }a},
	\]
	which is read ``$Y$ is the set of $a$ in $X$ \emph{such that} some rule
	involving $a$ is true.''  If $X$ is intuitive, we may omit it and
	simply write $Y=\Set{a\given\text{some rule involving }a}$\footnote{ If you want
	to get technical, to make this notation unambiguous, you define a 
	\emph{universe of discourse}.  That is, a set $\mathcal U$ containing
	every object you might want to talk about.  Then $\Set{a\given\text{some rule involving }a}$
	is short for $\Set{a\in\mathcal U\given\text{some rule involving }a}$}.  You may equivalently
	use ``$|$'' instead of ``$:$'', writing $Y=\{a\,|\,\text{some rule involving }a\}$.

	\begin{example}
		The set $\Z$ is the set of integers (positive, negative,
		and zero whole numbers).  To define $E$ as the even integers,
		we could write
		\[
			E=\Set{n\in \Z\given n=2k\text{ for some }k\in \Z}.
		\]
		To define $P$ as the set of positive integers, we could write
		\[
			P=\Set{n\in\Z\given n>0}.
		\]
	\end{example}


	There are also some common operations we can do with two sets.
	\begin{definition}[Intersections \& Unions]
		Let $A$ and $B$ be sets. Then the \emph{intersection} of $A$ and $B$, written
		$A\cap B$, is defined by
		\[
			A\cap B=\Set{x\given x\in A\text{ and }x\in B}.
		\]
		The \emph{union} of $A$ and $B$, written $A\cup B$, is defined by
		\[
			A\cup B= \Set{x\given x\in A\text{ or } x\in B}.
		\]
	\end{definition}
	For example, if $A=\Set{1,2,3}$ and $B=\Set{-1,0,1,2}$, then $A\cap B=\Set{1,2}$ and $A\cup B=
	\Set{-1,0,1,2,3}$.  Set unions and intersections are \emph{associative}, which means it doesn't
	matter how you apply parentheses to an expression involving just unions or just intersections.
	For example $(A\cup B)\cup C=A\cup(B\cup C)$, which means
	we can give an unambiguous meaning to an expression like $A\cup B\cup C$ (just put
	the parentheses wherever you like).  But watch out, $(A\cup B)\cap C$ means something
	different than $A\cup(B\cap C)$!

	\begin{definition}[Set Subtraction]
		For sets $A$ and $B$, the \emph{set-wise difference}\index{set subtraction} between $A$ and $B$,
		written $A\backslash B$, is the set
		\[
			A\backslash B = \Set{x\given x\in A\text{ and }x\notin B}.
		\]
	\end{definition}
	\begin{definition}[Cardinality]
		For a set $A$, the \emph{cardinality}\index{cardinality} of $A$,
		written $\abs{A}$ is the number of elements in $A$.  If $A$
		contains infinitely many elements, we write $\abs{A}=\infty$.
	\end{definition}

	Let's define some notation for common sets.
	\begin{align*}
		\emptyset &= \Set{}\text{, the empty set}\\
		\N &= \Set{0,1,2,3,\ldots}=\Set{\text{natural numbers}}\\
		\Z &= \Set{\ldots, -3,-2,-1,0,1,2,3,\ldots}=\Set{\text{integers}}\\
		\Q &= \Set{\text{rational numbers}}\\
		\R &= \Set{\text{real numbers}}\\
		\R^n &= \Set{\text{vectors in $n$-dimensional Euclidean space}}\\
	\end{align*}

	Besides unions, there's another way to join sets together:
	\emph{products}\index{Cartesian product}\index{set product}.
	\begin{definition}[Cartesian Product]
		Given two sets $A$ and $B$, the \emph{Cartesian product} (sometimes
		shortened to \emph{product}) of the sets $A$ and $B$ is written
		$A\times B$ and defined to be
		\[
			A\times B = \Set{(a,b)\given a\in A\text{ and }b\in B}.
		\]
	\end{definition}
	The Cartesian product of two sets is the set of all ordered pairs of elements from
	those sets.  For example, 
	\[
		\Set{1,2}\times \Set{1,2,3}=\Set{(1,1),(1,2),(1,3),(2,1),(2,2),(2,3)}.
	\]
	You can repeat this operation more than once.   $\R\times \R\times \R$
	is the set of all triples of real numbers.  Extending power notation
	notation, if you take the Cartesian product of a set
	with itself some number of times, you can represent it with
	an exponent.  Thus, $\R\times \R\times \R$ can be written as $\R^3$, which is a set we've
	seen before%
	\footnote{
		If you're scratching your head saying, ``I thought $\R^3$ was vectors in $3$-dimensional
		space.  How do we know that's the same thing as triples of real numbers?'' your mind
		is keen.  This is a theorem of linear algebra.
	}.

	\subsection{Functions}
	You're probably used to seeing functions like $f(x)=x^2$, but it's worth reviewing some of the concepts
	and terminology associated with functions.

	\begin{definition}[Function]
		A \emph{function}\index{function} with \emph{domain}\index{domain} the 
		set $A$ and \emph{co-domain}\index{co-domain} the set $B$ is an object that
		associates every point in the set $A$ with \emph{exactly one} point in the set $B$.
	\end{definition}

	If a function $f$ has domain $A$ and co-domain\footnote{ Some
	people use the word \emph{range} interchangeably with co-domain.} $B$,
	we notate this by writing $f:A\to B$.
	If we want to further specify what the function $f$ actually is, we need to
	express how $f$ associates each point in $A$ to a point in $B$.  This can be done
	with an equation.  For example, we could define the function $f:\R\to\R$ by
	\[
		f(x)=2x,
	\]
	which says that each real number gets associated to its double.  We can notate
	the same thing using a special type of arrow: ``$\mapsto$''.  Now we might write
	\[
		f:\R\to\R\text{ where } x\mapsto 2x,
	\]
	which is read ``$f$ is a function from $\R$ to $\R$ where $x\in \R$ gets mapped to $2x$.''

	
	Note that every point in the co-domain of a function doesn't need to get mapped
	to.  For example $g:\R\to\R$ given by $g(x)=x^2$ outputs only non-negative numbers,
	but it is still valid to specify $\R$ as the co-domain.  However, if we wanted
	to make a point of it, we are perfectly justified in writing $g:\R\to[0,\infty)$
	when defining $g$.

	Many common math operations give rise to functions.  For example,  
	$f(x)=\sqrt{x}$ is the familiar
	square root function.  Sometimes, when we wish to talk about a function
	for which notation already exists, we will put a ``$\ \cdot\ $'' where we would
	normally put a variable.  Thus, we might say, ``$\sqrt{\:\cdot\:}$ is the square
	root function.''\footnote{ Since $\sqrt{x}$ is ``the square root of the quantity
	$x$,'' it is technically a quantity and not a function.  This is why we write $\ \cdot\ $ instead
	of $x$ when we want to refer to the square root \emph{function}.
	}

	\begin{definition}[Range]
		The \emph{range}\index{range} of a function $f:A\to B$
		is the set of all outputs of $f$.  That is
		\[
			\Range f = \Set{y\in B\given y=f(x)\text{ for some }x\in A}.
		\]
	\end{definition}
	\begin{definition}[Image]
		Let $f:A\to B$ be a function.
		The \emph{image}\index{image} of a set $X\subseteq A$, written $f(X)$ is
		defined by
		\[
			f(X)=\Set{y\in B\given y=f(x)\text{ for some }x\in X}.
		\]
	\end{definition}

	We see that if $f:A\to B$, $\Range f = f(A)$.  In words, the range of $f$ is the image
	of its domain.  This language will become useful when we think of functions as transformations
	that move or bend space.  If $f:\R^2\to\R^2$ is a function that warps the Cartesian plane,
	then the image of $X$ under $f$ could be visualized by painting $X$ on the Cartesian plane,
	warping the whole plane, and then looking at the resulting, painted shape.  


	Closely related to images, we have the idea of \emph{restriction}\index{restriction}.
	Suppose $f:\R^2\to\R$ is defined by $f(x,y)=xy$, but we were only really interested in $f$ on the unit circle, $\mathcal C$.
	In this case, we might say $f$ attains a maximum on $\mathcal C$, or $f$ 
	\emph{restricted to} $\mathcal C$ attains a maximum, even though $f$ itself is unbounded.
	This idea comes up often enough to deserve its own notation.

	\begin{definition}[Restriction]
		If $f:A\to B$ and $X\subseteq A$, the \emph{restriction}\index{restriction}
		of $f$ to $X$ is written 
		$f\big|_X$ and is defined to be the function $g:X\to A$ where $x\mapsto f(x)$.
	\end{definition}

	The last important function-related ideas for us are function composition and inverses.
	Given two functions $f:A\to B$ and $g:B\to C$, we can \emph{compose}\index{function composition}
	$g$ and $f$ to get a new function.
	\begin{definition}[Composition]
		Given two functions $f:A\to B$ and $g:B\to C$, the \emph{composition} of
		$g$ and $f$, written $g\circ f$, is the function $h:A\to C$ where 
		$x\mapsto g(f(x))$.
	\end{definition}
	Note that the composition $g\circ f$ has the domain of $f$ and the co-domain of $g$.
	When a point is fed into $g\circ f$, it moves from $A\to B\to C$.  The composition
	$g\circ f$ only makes sense because the outputs of $f$ are allowed as inputs to $g$.
	If we wrote $f\circ g$, it wouldn't mean much, because $g$ outputs points in $C$ and $f$
	has no idea what to do with points in $C$.\footnote{
		It seems a little backward to write $f:A\to B$, $g:B\to C$ and then
		write $g\circ f$ instead of $f\circ g$.  You can thank Euler for that.
		He decided to write functions with their input on the right instead of
		the left.  If we wrote functions backwards, like $((x)f)g$ for ``$g$ of $f$ of $x$,''
		they we could just \emph{follow the arrows} and life would be simpler.
	}

	Inverses relate to composition and the \emph{identity function}\index{identity function}, the function
	that does nothing to its inputs.
	\begin{definition}[Identity Function]
		The \emph{identity function} $\Id:A\to A$ is defined by the relation
		\[
			\Id(x)=x
		\]
		for all $x\in A$.
	\end{definition}
	Notice that for every set, that set
	is the domain of an identity function.  Since the domain
	and co-domain of a function are part of its definition, we don't want to confuse them.
	After all, $f:\Set{0,1}\to \Set{0,1}$ given by $f(x)=x^2$ is a different function from $f:\R\to\R$
	given by $f(x)=x^2$.  For the special case of the identity function, we sometimes write
	the domain of the function as a subscript.  That is, 
	for $\Id:A\to A$ we'd write $\Id_A$ so it
	doesn't get confused with $\Id:B\to B$, which we'd write $\Id_B$.

	\begin{definition}[Inverse Function]
		Let $f:A\to B$ be a function.  If there exists a function $g:B\to A$ such that
		\[
			f\circ g=\Id_B\qquad\text{and}\qquad g\circ f=\Id_A,
		\]
		we say $f$ is \emph{invertible} and we call $g$ the \emph{inverse} of $f$.
		If $f$ is invertible, we notate its inverse by $f^{-1}$.
	\end{definition}

	Inverses can be tricky some times.  For example, consider $f(x)=x^2$ and $g(x)=\sqrt{x}$.
	Here $g\circ f(x)=\sqrt{x^2}=\abs{x}$ and $f\circ g(x) = \sqrt{x}^2=x$.  What's the deal?
	Well, it's all about domains.  $f:\R\to[0,\infty)$ and $g:[0,\infty)\to[0,\infty)$.
	So, the domain of $g\circ f$ is $\R$ and the domain of $f\circ g$ is $[0,\infty)$.  The
	domains are different, and indeed $f$ is not invertible.  However, $g$ \emph{is} invertible,
	and $g^{-1}=f\big|_{[0,\infty)}$.  If we only input non-negative numbers into $f$, then
	$f$ exactly undoes what $g$ did.  This subtle domain trickery can cause us a lot
	of headaches if we're not used to thinking carefully, and many of our favorite functions
	that we're used to calling ``inverse functions'' are actually only inverses when paired with
	specific domains.

\section{Proof}
	Mathematics has the highest standard of proof of any field.  In the Platonic ideal
	of mathematics, we start from some basic assumptions, called \emph{axioms}, that we have
	all agreed upon.  Then from those axioms, using the rules of logic, we deduce \emph{theorems}.
	Every single mathematical statement we make can be traced back from theorem to theorem
	and eventually to our initial axioms.

	This is contrary to other disciplines, like physics.
	In physics, based on observation, we construct
	\emph{laws}.  Laws in physics are like
	axioms in mathematics, but they have an important difference---they
	can be disproven by observation.  A mathematical axiom can never be disproven.  One can
	certainly argue that an axiom is not \emph{useful} or not \emph{interesting}, but you
	cannot say it's \emph{wrong}\footnote{
		There are multiple ways to axiomatize geometry.  In
		Euclidean geometry every pair of lines either coincides, intersects in
		exactly one place, or does not intersect.  In spherical
		geometry, every pair of lines either coincides or intersects in exactly two places.  
		Euclidean geometry is useful when your space looks flat.  Spherical
		geometry is useful when your space is the surface of a sphere (like 
		the Earth).
		Is one of these more \emph{right} than the other?  They're certainly
		contradictory.
	}.
	Of course, as human practitioners, we may misuse logic and be wrong ourselves, but that
	is no fault of the axioms.

	But now, let's deviate from philosophical perfection and visit reality.
	In reality, \emph{mathematics is a human pursuit to understand relationships
	between ideas and their consequences}.  The key there is that \emph{humans} do
	mathematics to \emph{understand} relationships.  If a theorem in math can
	ultimately be reduced to logical statements about axioms, but the argument is
	100000 steps long, it doesn't help a human understand why something is true.
	Instead, a shorter argument that skips over some steps is more useful to us.
	And, indeed, most of our mathematics to date skips over some steps\footnote{
		There are some projects to prove all of mathematics directly from
		the axioms using computer assistance.  They've made progress, but there
		are still theorems in calculus that have not been reduced to the
		axioms.  We believe that they \emph{could be} reduced to the axioms,
		but no one has taken the time to do so.}.

	We call a correct mathematical argument a \emph{proof}.  A proof starts
	from a set of assumptions, and following the rules of logic, arrives at a conclusion.
	Strictly speaking, a proof doesn't need to make sense or show motivation,
	applications, or examples.  It just has to be a sequence of correct logical steps.
	However, for us, as humans studying mathematics, we prove things for two reasons:
	to understand why things are true and to avoid making mistakes.

	Reconciling these two goals can be very hard for a novice mathematician.  If you include
	\emph{all} the steps, it won't help with understanding, but if you don't include enough
	steps, the argument may not be convincing and might contain mistakes.  
	Even professionals struggle to balance
	these competing goals, and how you balance those goals depends on your audience---if you're
	trying to convince your math professor of something your proof will need to have more
	detail than if you were trying to convince your friend (mathematicians are very skeptical!).

	Enough talk, let's go through a 2000-year-old example of a proof.
	\begin{theorem}
		There is no rational number $p/q$ such that $(p/q)^2=2$.
	\end{theorem}
	\begin{proof}
		If $p/q$ is a rational number, it can be expressed
		in lowest terms.
		Suppose $p/q$ is in lowest terms and $(p/q)^2=2$.  Then $p^2=2q^2$ and so $p^2$ is even.  Since
		$p^2$ is even, it must be that $p$ is even, and so by definition,
		$p=2m$ for some integer $m$.  Now,
		\[
			\frac{p^2}{q^2}=\frac{(2m)^2}{q^2}=\frac{4m^2}{q^2}=2,
		\]
		with the last equality following by assumption.  Multiplying both sides by $q^2$
		and dividing by $2$ we arrive at the equation
		\[
			2m^2=q^2,
		\]
		and so $q^2$ is even which means $q$ is even.  By definition, this means $q=2n$ for some integer $n$.
		But now,
		\[
			\frac{p}{q}=\frac{2m}{2n}
		\]
		is not in lowest terms!  This is a contradiction and so it cannot be that $(p/q)^2=2$.
	\end{proof}
	This is nearly identical to the argument the ancient Greeks gave.  It's elegant, beautiful,
	and convincing.  But, if we look closer, it does skip some steps.  For example, it relies on
	the fact that there is such a thing as \emph{lowest terms}.  This is something that would
	need to be proven---a priori, the conclusion of the proof could be that the assumption
	that $p/q$ could be in lowest terms is false.  
	
	You will not, overnight,
	become a master at understanding
	what steps you can leave out and what steps you must show.  However, with feedback,
	you'll get better.
	For a detailed guide on writing good proofs, please see Appendix \ref{APPENDIX-proofstyle}.


	\clearpage
\chapter{Vectors}
	
\newpage
A \emph{vector}\index{vector} is characterized by
a \emph{magnitude} and a \emph{direction}, and using vectors, we can
mathematically describe more situations than with regular numbers.
For example, when driving a car,
 it may be sufficient to
know your speed, which can be described by a single number,
 but the motion of an airplane must be described
by a vector quantity---velocity---which takes into account its
direction as well as its speed.

Ordinary numerical quantities are called \emph{scalars}\index{scalar}
when we want to emphasize that they are not vectors.

Whereas numbers allow us to specify relationships between single quantities
(put in twice as much flour as sugar), vectors will allow us to specify
relationships between geometric objects in space\footnote{
	Though in this book we will treat vectors as objects in Euclidean
	space, but they are much more general.  For instance, someone's internet
	browsing habits could be described by a vector---the topics they
	find most interesting might be the ``direction'' and the amount
	of time they browse might be the ``magnitude.''
}.  If we have two points, $P=(1,1)$ and $Q=(3,2)$, we specify the
\emph{displacement}\index{displacement} from $P$ to $Q$ as a vector.

\begin{center}
	\usetikzlibrary{patterns,decorations.pathreplacing}
	\begin{tikzpicture}
		\coordinate (A) at (1,1);
		\coordinate (B) at (3,2);
		\begin{axis}[
		    anchor=origin,
		    disabledatascaling,
		    xmin=-1,xmax=5,
		    ymin=-1,ymax=3,
		    x=1cm,y=1cm,
		    grid=both,
		    grid style={line width=.1pt, draw=gray!10},
		    %major grid style={line width=.2pt,draw=gray!50},
		    axis lines=middle,
		    minor tick num=0,
		    enlargelimits={abs=0.5},
		    axis line style={latex-latex},
		    ticklabel style={font=\tiny,fill=white},
		    xlabel style={at={(ticklabel* cs:1)},anchor=north west},
		    ylabel style={at={(ticklabel* cs:1)},anchor=south west}
		]

		\draw [mypink,fill] (A) circle (1.5pt) node [below right] {$P$};
		\draw [mypink,fill] (B) circle (1.5pt) node [below right] {$Q$};
		\draw[->,thick,myred!60!white] (A) -- (B) node [midway,above,yshift=2pt] {$\overrightarrow{PQ}$};

		\end{axis}
	\end{tikzpicture}
\end{center}

We notate the displacement vector form $P$ to $Q$ by $\overrightarrow{PQ}$.
The magnitude of $\overrightarrow{PQ}$ is given by the Pythagorean theorem
to be $\sqrt{5}$ and its direction is specified by the directed line segment from
$P$ to $Q$.


\section{Vector Notation}
There are many ways to represent vector quantities in writing.  If
we have two points, $P$ and $Q$, we write $\overrightarrow{PQ}$ to represent the
vector from $P$ to $Q$.  Absent of points, bold-faced letters or a letter
with an arrow over it are the most common typographical representations of vectors.
For example, $\vec a$ or $\mathbf{a}$ may both be used to represent the vector
quantity named ``$a$.''  In this book we will use $\vec a$ to represent a vector.
The notation $\norm{\vec a}$\index{$\norm{\:\cdot\:}$}\index{magnitude}\index{norm}
represents the magnitude of the vector $\vec a$, which is sometimes called
the \emph{norm} or \emph{length} of $\vec a$.

\begin{definition}[Norm]
	The \emph{norm} of a vector $\vec a$, notated $\norm{\vec a}$, is the
	magnitude of $\vec a$.
\end{definition}

Graphically, vectors are represented as directed line segments (a
line segment with an arrow at one end), however, vectors are fundamentally
directions and magnitudes. Since any directed line segment contains a starting
point and an ending point, a directed line segment contains extra information that
is not actually part of the vector.

\begin{center}
	\usetikzlibrary{patterns,decorations.pathreplacing}
	\begin{tikzpicture}
		\coordinate (A) at (1,1);
		\coordinate (B) at (-.5,2);

		\draw [mypink,fill] (A) circle (1.5pt) node [right] {starting point};
		\draw [mypink,fill] (B) circle (1.5pt) node [left] {ending point};
		\draw[->,thick,myred!60!white] (A) -- (B);
	\end{tikzpicture}
\end{center}

For example, let $A=(1,1)$, $B=(3,2)$, $X=(1,0)$, and $Y=(3,1)$ and consider the vectors
$\vec a = \overrightarrow{AB}$ and $\vec x=\overrightarrow{XY}$.  


\begin{center}
	\usetikzlibrary{patterns,decorations.pathreplacing}
	\begin{tikzpicture}
		\coordinate (A) at (2,1);
		\begin{axis}[
		    anchor=origin,
		    disabledatascaling,
		    xmin=-1,xmax=5,
		    ymin=-1,ymax=3,
		    x=1cm,y=1cm,
		    grid=both,
		    grid style={line width=.1pt, draw=gray!10},
		    %major grid style={line width=.2pt,draw=gray!50},
		    axis lines=middle,
		    minor tick num=0,
		    enlargelimits={abs=0.5},
		    axis line style={latex-latex},
		    ticklabel style={font=\tiny,fill=white},
		    xlabel style={at={(ticklabel* cs:1)},anchor=north west},
		    ylabel style={at={(ticklabel* cs:1)},anchor=south west}
		]

			\draw[->,thick,myred!60!white] (1,1) -- +(A) node [midway,above,xshift=-8pt] {$\vec a=\overrightarrow{AB}$};
			\draw[->,thick,mypink] (1,0) -- +(A) node [midway,above,xshift=-8pt] {$\vec x=\overrightarrow{XY}$};
		%\draw[->,thick,myred!60!white] (3,1) -- +(A) node [midway,above,yshift=2pt] {$\vec x$};

		\end{axis}
	\end{tikzpicture}
\end{center}

Are these vectors
the same or different vectors?  As directed line segments,
they are different because they are at different locations in space.  
However, both $\vec a$ and $\vec x$ have the same
magnitude and direction.  Thus, $\vec a=\vec x$ despite the fact that $A\neq X$\footnote{
	Some theories use \emph{rooted vectors}\index{rooted vector} instead of
	vectors as the fundamental object of study. A rooted vector
	represents a magnitude, direction, \emph{and} a starting point. And, 
	as rooted vectors, $\vec a\neq \vec x$ (from the example above).
	But for us, vectors will always be unrooted, even though our graphical
	representations of vectors might appear rooted.
}.


\subsection{Vectors and Points}
The distinction between vectors and points is sometimes nebulous because
they are so closely related to each other.  A \emph{point}\index{point}
in Euclidean space specifies an absolute position whereas a vector
specifies a magnitude and direction.  However, given a point $P$,
one associates $P$ with the vector $\vec p=\overrightarrow{OP}$, where $O$
is the origin.  Similarly, we associate the vector $\vec v$ with
the point $V$ so that $\overrightarrow{OV}=\vec v$.
Thus, we have a way to unambiguously go back and forth between vectors and
points\footnote{ Mathematically, we say there is an \emph{isomorphism} between
vectors and points.}.  As such, \emph{we will treat vectors and points
as interchangeable}.


\section{Vector Arithmetic}
\label{SECVECTORPROPS}
Vectors provide a natural way to give directions.
For example, suppose $\xhat$ points one mile eastwards and $\yhat$
points one mile northwards.  Now, if you were standing at the origin
and wanted to move to a location 3 miles east and 2 miles north, you might say:
``Walk 3 times the length of $\xhat$  in the $\xhat$ direction and 2 times
the length of $\yhat$ in the $\yhat$ direction.''  Mathematically, we express this
as
\[
	3\xhat+2\yhat.
\]
Of course, we've incidentally described a new vector.  Namely, let $P$
be the point at 3-east and 2-north.  Then
\[
	\overrightarrow{OP}=3\xhat+2\yhat.
\]
If the vector $\vec r$ points north but has a length of 10 miles, we have
a similar formula:
\[
	\overrightarrow{OP}=3\xhat+\tfrac{1}{5}\vec r,
\]
and we have the relationship $\vec r=10\yhat$.
Our notation here is very suggestive.  Indeed, if we could make
sense of what $\alpha\vec v$ is for any scalar $\alpha$ and vector
$\vec v$, and we could make sense of what $\vec v+\vec w$
means for any vectors $\vec v$ and $\vec w$, we would be able to
do algebra with vectors.  We might even say we have \emph{an algebra
of vectors}.

Intuitively, for a vector $\vec v$ and a scalar $\alpha>0$, the
vector $\vec w=\alpha\vec v$ should point in the same direction as
$\vec v$ but have magnitude scaled up by $\alpha$.  That is, $\norm{\vec w}=\alpha\norm{\vec v}$.
Similarly, $-\vec v$ should be the vector of the same length as $\vec v$ but
pointing in the exact opposite direction.

\begin{center}
	\usetikzlibrary{patterns,decorations.pathreplacing}
	\begin{tikzpicture}
		\coordinate (A) at (2,1);

		\draw[->,thick,green!50!black] (1,0) -- +($(A) + (A)$) node [midway,below right ] {$2\vec v$};
		\draw[->,thick,myred!60!white] (0,0) -- +(A) node [midway,above left] {$\vec v$};
		\draw[->,thick,mypink] (-1,0) -- +($-.5*(A)$) node [midway,right,xshift=4pt] {$-\tfrac{1}{2}\vec v$};
	\end{tikzpicture}
\end{center}

For two vectors $\vec u$ and $\vec v$, the sum $\vec w=\vec u+\vec v$
should be the displacement vector created by first displacing along $\vec u$
and then displacing along $\vec v$.

\begin{center}
	\usetikzlibrary{patterns,decorations.pathreplacing}
	\begin{tikzpicture}
		\coordinate (A) at (1,1);
		\coordinate (B) at (-.5,2);
		\coordinate (C) at (3,3);

		%\draw [mypink,fill] (A) circle (1.5pt) node [right] {initial point};
		%\draw [mypink,fill] (B) circle (1.5pt) node [left] {terminal point};
		\draw[->,thick,myred!60!white] (A) -- (B) node [midway,below left] {$\vec u$};
		\draw[->,thick,mypink] (B) -- (C) node [midway,above left] {$\vec v$};
		\draw[->,thick,green!50!black] (A) -- (C) node [midway,right] {$\vec w=\vec u+\vec v$};
	\end{tikzpicture}
\end{center}

Now, there is one snag.  What should $\vec v+(-\vec v)$ be?  Well, first we
displace along $\vec v$ and then we displace in the exact opposite direction
by the same amount.  So, we have gone nowhere.  This corresponds to a displacement
with zero magnitude.  But, what direction did we displace?  Here we make a philosophical
stand.
\begin{definition}[Zero Vector]
	The \emph{zero vector}\index{zero vector}, notated as $\vec 0$\index{$\vec 0$},
	is the vector with no magnitude.
\end{definition}
We will be pragmatic about the direction of the zero vector and say,
\emph{the zero vector does not have a well-defined direction}\footnote{
	In the mathematically precise definition of vector, the idea of ``magnitude''
	and ``direction'' are dropped.  Instead, a set of vectors is defined to be
	a set over which you can reasonably define addition and scalar multiplication.
}.  That means
sometimes we consider the zero vector to point in every direction and sometimes
we consider it to point in no directions.  It depends on our mood---but we must
never talk about \emph{the} direction of the zero vector, since it's not defined.

We need the zero vector if we are to make precise mathematical
sense of vector arithmetic.
Further along this line of thinking, we can define precisely how vector arithmetic
should behave.  Specifically, if $\vec u$, $\vec v$, $\vec w$ are vectors and $\alpha$ and $\beta$
are scalars, the
following conditions should be satisfied:
\begin{align*}
	(\vec u+\vec v)+\vec w&=\vec u+(\vec v+\vec w)\tag{Associativity}\\
	\vec u+\vec v&=\vec v+\vec u\tag{Commutativity}\\
	\alpha(\vec u+\vec v)&=\alpha\vec u+\alpha \vec v\tag{Distributivity}
\end{align*}
and
\begin{align*}
	(\alpha\beta)\vec v&=\alpha(\beta \vec v)\tag{Associativity II}\\
	(\alpha+\beta)\vec v&=\alpha\vec v+\beta \vec v\tag{Distributivity II}
\end{align*}

Indeed, if we intuitively think about vectors in flat (Euclidean) space,
all of these properties are satisfied\footnote{
	If we deviate from flat space, some of these
	rules are no longer respected.  Consider moving 100 miles
	north then 100 miles east on a sphere.  Is this the
	same as moving 100 miles east and then 100 miles north?
}.  From now on, these properties of vector operations will be considered
the
\emph{laws (or axioms) of vector arithmetic}.

We'll be talking about these vector operations (scalar multiplication and
vector addition) a lot.  So much so that the concept is worth naming.
\begin{definition}[Linear Combination]
	A \emph{linear combination} of the vectors $\vec v_1,\ldots, \vec v_n$
	is any vector expressible as
	\[
		\alpha_1\vec v_1+\cdots +\alpha_n\vec v_n,
	\]
	where $\alpha_1,\alpha_2,\ldots,\alpha_n$ are scalars.
\end{definition}


We've given laws for linear combinations of vectors, but what about
for magnitudes of vectors?  We'd like the magnitude (or norm\index{norm}) of a vector to obey
the following laws.
\begin{align*}
	\norm{\vec v} &\geq 0\tag{Non-negativity}\\
	\norm{\vec v} &= 0\text{ only when }\vec v=\vec 0\tag{Definiteness}\\
	\norm{\alpha\vec v}&=\abs{\alpha}\norm{\vec v}\tag{Homogeneity}\\
	\norm{\vec v+\vec w}&\leq\norm{\vec v}+\norm{\vec w}\tag{Triangle Inequality}
\end{align*}
for all $\vec v$, $\vec w$, and scalars $\alpha$.  Any function on vectors satisfying
those four properties is called a \emph{norm}, and our usual notion
of length in three-dimensional space indeed obeys those properties\footnote{
	The Euclidean norm comes from the Pythagorean theorem $a^2+b^2=c^2$.
	However, by changing the exponent, we have a whole family of norms
	coming from the equations $\abs{a}^p+\abs{b}^p=\abs{c}^p$.
}.

Homogeneity is a particularly special property of a norm.  It allows us
to easily create \emph{unit vectors}.
\begin{restatable}[Unit Vector]{definition}{DefUnitVector}
	A \emph{unit vector}\index{unit vector} is a vector $\vec u$
	satisfying $\norm{\vec u}=1$.
\end{restatable}

Unit vectors are handy because if $\vec u$ is a unit vector, then $k\vec u$
has length $\abs{k}$.  Further, we can always turn a vector into a unit vector.
\begin{example}
	The vector $\vec v/\norm{\vec v}$ when $\vec v\neq\vec 0$ is always a unit vector in the direction
	of $\vec v$.  Computing,
	\[
		\norm*{\frac{\vec v}{\norm{\vec v}}} = \abs*{\frac{1}{\norm{\vec v}}}\norm{\vec v}=
		\frac{1}{\norm{\vec v}}\norm{\vec v}=1.
	\]
\end{example}

\begin{exercises}
    \begin{problist}
        \prob   Find the displacement vector
        from the point $P(7, 2, 9)$ to the point $Q(-2, 1, 4)$
        in the form $a\xhat + b\yhat + c\zhat$.
        \begin{solution}
            $-9\xhat-1\yhat-5\zhat$
        \end{solution}

        \prob[\openstax]   For the following exercises, consider points $P(-1,3)$, $Q(1,5)$, and $R(-3,7)$.
        Determine the requested vectors and express each of them i. in component form
        and ii. by using the standard unit vectors.
        \begin{enumerate}
            \item $\overrightarrow{PQ}$
            \item $\overrightarrow{PR}$
            \item $\overrightarrow{QP}$
            \item $\overrightarrow{RP}$
        \end{enumerate}
        \begin{solution}
            \begin{enumerate}
                \item   $\overrightarrow{PQ} = (2,2) = 2\xhat + 2\yhat$.
                \item   $\overrightarrow{PR} = (-2, 4) = -2\xhat + 4\yhat$.
                \item   $\overrightarrow{QP} = (-2,-2) = -2\xhat - 2\yhat$.
                \item   $\overrightarrow{RP} = (2, -4) = 2\xhat - 4\yhat$.
            \end{enumerate}
        \end{solution}
    \end{problist}
\end{exercises}


\section{Coordinates}
A rectangular coordinate system in the plane is specified by choosing
an origin $O$ and then choosing two perpendicular axes meeting at
the origin.  These axes are chosen in some order so that we know which
axis (usually the $x$-axis) comes first and which (usually the
$y$-axis) second.   Note that there are many different coordinate systems
which could be used although we often draw pictures
as if there were only one.

In physics, one often has to think carefully about the
coordinate system because
choosing one appropriately may greatly simplify the
analysis.  Note that axes for coordinate systems are usually drawn with
\emph{right-hand orientation}, where the right angle from the positive
$x$-axis to the positive $y$-axis is in the counter-clockwise
direction.  However, it would be equally valid to use the
\emph{left-hand orientation} in which that angle is in the
clockwise direction.  One can easily switch the orientation of
a coordinate system by reversing one of the axes\footnote{
	The concept of
orientation is quite fascinating and it arises in mathematics,
physics, chemistry, and even biology in many interesting ways.
Note that almost all of us base our intuitive concept of orientation
on our inborn notion of ``right'' versus ``left''.}.

\begin{center}
	\begin{tikzpicture}
		\begin{axis}[
		    anchor=origin,
		    disabledatascaling,
		    xmin=-1,xmax=3,
		    ymin=-1,ymax=2,
		    x=1cm,y=1cm,
		    grid=both,
		    grid style={line width=.1pt, draw=gray!10},
		    %major grid style={line width=.2pt,draw=gray!50},
		    axis lines=middle,
		    minor tick num=0,
		    enlargelimits={abs=0.5},
		    axis line style={->},
		    ticklabel style={font=\tiny,fill=white},
		    xlabel={$x$}, ylabel={$y$},
		    xlabel style={at={(ticklabel* cs:1)},anchor=west},
		    ylabel style={at={(ticklabel* cs:1)},anchor=south}
		]

		\end{axis}
		\draw[] (.2,1) node[right,myorange!60!black] {Right Handed};
	\end{tikzpicture}
	\hspace{1cm}
	\begin{tikzpicture}
		\begin{axis}[
		    anchor=origin,
		    disabledatascaling,
		    xmin=-1,xmax=3,
		    ymin=-1,ymax=2,
		    x=1cm,y=1cm,
		    x dir=reverse,
		    grid=both,
		    grid style={line width=.1pt, draw=gray!10},
		    %major grid style={line width=.2pt,draw=gray!50},
		    axis lines=middle,
		    minor tick num=0,
		    enlargelimits={abs=0.5},
		    axis line style={->},
		    ticklabel style={font=\tiny,fill=white},
		    xlabel={$x$}, ylabel={$y$},
		    xlabel style={at={(ticklabel* cs:1)},anchor=west},
		    ylabel style={at={(ticklabel* cs:1)},anchor=south}
		]

		\end{axis}
		\draw[] (-.4,1) node[left,mypink] {Left Handed};
	\end{tikzpicture}
\end{center}

For any coordinate system, there are special vectors
associated with it.  For the plane, the vector pointing one unit along
the positive $x$-axis is called $\xhat$ and the vector pointing one unit along
the positive $y$-axis is called $\yhat$.  The vectors $\xhat$ and $\yhat$ are
called the \emph{standard basis}\index{standard basis} vectors for $\R^2$.

Notice that every point (or vector) in the plane can be represented
as a linear combination of $\xhat$ and $\yhat$, and the vector
$\alpha\xhat+\beta\yhat$ is the vector $\overrightarrow{OP}$ where
$P=(\alpha,\beta)$.  Now, to state an intuitive fact:  if $\vec w$ is
a vector in the plane, \emph{there is
only one way to write a vector as a linear combination of
$\xhat$ and $\yhat$}.  This means, if $\vec w=\alpha\xhat+\beta\yhat$,
the pair $(\alpha,\beta)$ captures all information\footnote{
	Maybe you already knew this because the point $(\alpha,\beta)$
	is described by the pair of numbers $(\alpha,\beta)$, duh!
	But consider, what would we do if we didn't know about coordinates
	at all? One approach is to \emph{define} coordinates in terms
	of vectors, which is really what we're doing.
} about $\vec w$.


For a vector $\vec w=\alpha\xhat+\beta\yhat$,
we call the pair $(\alpha,\beta)$  the
\emph{coordinates}\index{coordinates} of the vector $\vec w$.  There
are many equivalent notations used to represent a vector in coordinates.
\begin{center}
	\begin{tabular}{c p{5cm}}
		$(\alpha,\beta)$ & parenthesis\\
		$\langle \alpha,\beta\rangle$ & angle brackets\\
		$\mat{\alpha&\beta}$ & square brackets in a row (a row matrix)\\
		$\mat{\alpha\\\beta}$ & square brackets in a column (a column matrix)\\
	\end{tabular}
\end{center}

Given what we now know about representing vectors and their equivalency
with points, we can dissect the notation $\R^2$.
On the one hand, $\R^2$ is the set of vectors in two-dimensional Euclidean
space.  On the other hand $\R^2=\R\times \R$ is the set of all pairs of real
numbers.  Via the use of coordinates, we know these concepts represent the
same thing!
Further,
since vectors in $\R^2$ are equivalent to their representation in coordinates,
we will often write
\[
	\vec v=\mat{\alpha\\\beta}
\]
as a shorthand for $\vec v=\alpha\xhat+\beta\yhat$.


Breaking vectors into coordinates, and in particular, viewing vectors as linear
combinations of the standard basis vectors, allows us to solve problems that were
difficult before.  For instance, suppose we have vectors $\vec v$ and $\vec w$.
How can we compute $\norm{\vec v+\vec w}$?  With coordinates, it's easy.

\begin{example}
	\label{EXAMPLE-vecadd}
	Suppose $\vec v=\alpha_1\xhat+\beta_1\yhat$ and $\vec w=\alpha_2\xhat+\beta_2\yhat$.
	By the laws of vector arithmetic we have
	\[
		\vec v+\vec w=(\alpha_1\xhat+\beta_1\yhat)+(\alpha_2\xhat+\beta_2\yhat)
		=(\alpha_1+\alpha_2)\xhat+(\beta_1+\beta_2)\yhat.
	\]
	Now, since $\xhat$ and $\yhat$ are orthogonal to each other,
	the Pythagorean theorem gives
	\[
		\norm{\vec v+\vec w} = \sqrt{(\alpha_1+\alpha_2)^2+(\beta_1+\beta_2)^2}.
	\]
\end{example}

Writing things in terms of the standard basis allowed us to make easy work
of computing $\norm{\vec v+\vec w}$ in Example \ref{EXAMPLE-vecadd}.  We can
use the laws of vector arithmetic to produce rules for working with components.

The rules are are likely familiar:
\[
	\mat{a\\b}+\mat{\alpha\\\beta} = \mat{a+\alpha\\b+\beta}
	\qquad\text{and}\qquad
	\alpha \mat{a\\b}=\mat{\alpha a\\\alpha b}.
\]

\begin{exercise}
	Prove the rules for adding the component representation
	of vectors and multiplying the component representation
	of vectors directly from the laws of vector arithmetic.
\end{exercise}

Armed with these rules, we will be able to tackle sophisticated vector
problems.

\subsection{Three-dimensional Coordinates}
In three-dimensional space, the story is very similar.  Again, we imagine
three perpendicular axes, the $x$, $y$, and $z$ axes.
To draw consistent
pictures, we have a notion of a right-handed three-dimensional coordinate
system given by the \emph{right-hand rule}.

\begin{center}
	\begin{tikzpicture}
		\begin{axis}[
			name=plot0,
		    anchor=origin,
		    scale mode=scale uniformly,
			scale=2,
		    %disabledatascaling,
		    xmin=-2,xmax=2,
		    ymin=-2,ymax=2,
		    zmin=-2,zmax=2,
		    %x=1cm,z=1cm,y=1cm,
		    %grid=both,
		    %grid style={line width=.1pt, draw=gray!10},
		    %major grid style={line width=.2pt,draw=gray!50},
		    xtick={-2,...,2}, ytick={-2,...,2}, ztick={0,...,2},
		    axis lines=middle,
		    minor tick num=0,
		    enlargelimits={abs=0.5},
		    axis line style={->},
		    ticklabel style={font=\tiny},
			xlabel={$x$}, ylabel={$y$}, zlabel={$z$},
		    xlabel style={at={(ticklabel* cs:1)},anchor=west},
		    zlabel style={at={(ticklabel* cs:1)},anchor=south},
		    ylabel style={at={(ticklabel* cs:1)},anchor=south}
		]


		\end{axis}

		\begin{axis}[
		    name=plot1,
			at=(plot0.right of south east),
		    scale mode=scale uniformly,
			scale=2,
			xshift=-1cm,
		    x dir=reverse,
		    y dir=reverse,
		    %disabledatascaling,
		    xmin=-2,xmax=2,
		    ymin=-2,ymax=2,
		    zmin=-2,zmax=2,
		    %x=1cm,z=1cm,y=1cm,
		    %grid=both,
		    %grid style={line width=.1pt, draw=gray!10},
		    %major grid style={line width=.2pt,draw=gray!50},
		    xtick={-2,...,2}, ytick={-2,...,2}, ztick={0,...,2},
		    axis lines=middle,
		    minor tick num=0,
		    enlargelimits={abs=0.5},
		    axis line style={->},
		    ticklabel style={font=\tiny},
			xlabel={$x$}, ylabel={$y$}, zlabel={$z$},
		    xlabel style={at={(ticklabel* cs:0)},anchor=east},
		    zlabel style={at={(ticklabel* cs:1)},anchor=south},
		    ylabel style={at={(ticklabel* cs:0)},anchor=east}
		]
		\end{axis}
		\begin{axis}[
		    name=plot2,
			at=(plot1.right of south east),
		    scale mode=scale uniformly,
			scale=2,
		    x dir=reverse,
		    %disabledatascaling,
		    xmin=-2,xmax=2,
		    ymin=-2,ymax=2,
		    zmin=-2,zmax=2,
		    %x=1cm,z=1cm,y=1cm,
		    %grid=both,
		    %grid style={line width=.1pt, draw=gray!10},
		    %major grid style={line width=.2pt,draw=gray!50},
		    xtick={-2,...,2}, ytick={-2,...,2}, ztick={0,...,2},
		    axis lines=middle,
		    minor tick num=0,
		    enlargelimits={abs=0.5},
		    axis line style={->},
		    ticklabel style={font=\tiny},
			xlabel={$x$}, ylabel={$y$}, zlabel={$z$},
		    xlabel style={at={(ticklabel* cs:0)},anchor=east},
		    zlabel style={at={(ticklabel* cs:1)},anchor=south},
		    ylabel style={at={(ticklabel* cs:1)},anchor=west}
		]
		\end{axis}



		\draw[] (plot0.north) node[yshift=-1cm,myorange!60!black] {Right Handed};
		\draw[] (plot1.north) node[yshift=-1cm,myorange!60!black] {Right Handed};
		\draw[] (plot2.north) node[yshift=-1cm,mypink!80!black] {Left Handed};
	\end{tikzpicture}
\end{center}
\begin{center}
\definecolor{ce5d4b1}{RGB}{229,212,177}
\definecolor{c897f6a}{RGB}{137,127,106}
\definecolor{c963c96}{RGB}{150,60,150}
\definecolor{c2828ff}{RGB}{40,40,255}
\definecolor{ce12828}{RGB}{225,40,40}

\begin{tikzpicture}[y=0.80pt, x=0.80pt, yscale=-1.000000, xscale=1.000000, inner sep=0pt, outer sep=0pt]
\path[draw=black,fill=ce5d4b1,line width=1.109pt] (197.7203,128.5199) ..
  controls (187.5579,127.1341) and (171.3905,117.8956) .. (169.0809,116.5099) ..
  controls (166.7713,115.1241) and (156.1470,103.5760) .. (150.6039,94.7994) ..
  controls (145.0608,86.0228) and (139.9796,75.3985) .. (138.1319,69.3935) ..
  controls (136.2842,63.3885) and (134.4365,57.3834) .. (133.9746,54.1500) ..
  controls (133.5127,50.9165) and (134.4365,43.2947) .. (135.8223,40.0612) ..
  controls (132.5888,37.7516) and (126.1219,37.2897) .. (123.3503,43.2947) ..
  controls (120.5788,49.2997) and (118.2691,58.5382) .. (118.2691,64.5433) ..
  controls (118.2691,70.5483) and (128.4315,89.9492) .. (121.9645,92.7207) ..
  controls (115.4976,95.4923) and (114.1118,99.1877) .. (80.8532,90.8730) ..
  controls (47.5946,82.5584) and (33.7368,81.6345) .. (30.5034,88.1015) ..
  controls (27.2699,94.5684) and (46.2088,97.3400) .. (56.3712,99.1877) ..
  controls (66.5335,101.0354) and (89.6395,109.6429) .. (86.3963,113.9693) ..
  controls (84.3869,116.6493) and (82.0658,116.0424) .. (76.8303,117.8901) ..
  controls (67.7715,121.0875) and (67.9114,122.4936) .. (56.6021,126.3641) ..
  controls (45.3159,130.2267) and (46.0504,141.2501) .. (49.5961,143.0996) ..
  controls (53.1419,144.9492) and (56.0705,146.0292) .. (71.0230,137.5505) ..
  controls (77.6516,135.0839) and (85.7039,133.4463) .. (92.7866,132.8306) ..
  controls (99.8693,132.2148) and (104.1813,158.6984) .. (111.8798,162.7019) ..
  controls (119.5782,166.7054) and (128.5086,166.3973) .. (134.0517,166.3973) ..
  controls (139.5948,166.3973) and (142.6740,167.3216) .. (145.7537,165.7816) ..
  controls (148.8333,164.2415) and (157.4561,166.3968) .. (162.0754,168.8607) ..
  controls (162.0754,168.8607) and (179.9366,177.4835) .. (181.4762,179.3312);
\path[draw=c897f6a,line cap=round,miter limit=4.00,line width=0.739pt]
  (90.3231,113.2759) .. controls (106.0281,112.3525) and (111.5712,127.4418) ..
  (115.2666,135.4488);
\path[draw=c897f6a,line cap=round,miter limit=4.00,line width=0.739pt]
  (123.4275,101.5735) .. controls (123.5817,113.2759) and (128.5091,123.1306) ..
  (135.1294,126.8251);
\path[draw=c897f6a,line cap=round,miter limit=4.00,line width=0.739pt]
  (92.7866,94.7989) .. controls (89.0912,95.4147) and (84.7800,104.0374) ..
  (86.6277,106.5009);
\path[draw=c897f6a,line cap=round,miter limit=4.00,line width=0.739pt]
  (73.0776,89.8716) .. controls (70.6142,91.7193) and (65.6868,96.0304) ..
  (66.9188,98.4943);
\path[draw=c897f6a,line cap=round,miter limit=4.00,line width=0.739pt]
  (50.9052,86.1762) .. controls (47.8256,88.0239) and (46.8246,93.3365) ..
  (46.2088,94.5679);
\path[draw=c897f6a,line cap=round,miter limit=4.00,line width=0.739pt]
  (84.7800,117.5875) .. controls (88.4754,119.4352) and (90.3231,125.5936) ..
  (89.0912,129.9052);
\path[draw=c897f6a,line cap=round,miter limit=4.00,line width=0.739pt]
  (74.9932,120.9466) .. controls (78.0729,122.7943) and (78.3413,132.0296) ..
  (77.1093,133.8773);
\path[draw=c897f6a,line cap=round,miter limit=4.00,line width=0.739pt]
  (63.2381,126.1188) .. controls (67.2329,129.1144) and (67.6685,133.2944) ..
  (65.5889,138.3441);
\path[draw=c897f6a,line cap=round,miter limit=4.00,line width=0.739pt]
  (117.8072,97.3400) .. controls (115.9595,101.9592) and (114.5737,105.6546) ..
  (115.0357,108.4262);
\path[draw=c897f6a,line cap=round,miter limit=4.00,line width=0.739pt]
  (118.2691,64.5433) .. controls (123.7878,65.3318) and (119.5057,66.0455) ..
  (129.8173,64.0813);
\path[draw=c897f6a,line cap=round,miter limit=4.00,line width=0.739pt]
  (166.3865,120.6672) .. controls (164.5388,126.2103) and (162.6911,128.0575) ..
  (159.6114,129.9052);
\path[draw=black,fill=ce5d4b1,line width=1.109pt] (120.5021,161.1628) ..
  controls (127.8490,159.6934) and (133.1283,158.6993) .. (137.1314,157.4674) ..
  controls (141.1344,156.2354) and (146.9852,152.8476) .. (148.5252,149.7685) ..
  controls (150.0653,146.6893) and (148.8333,146.0735) .. (147.9095,143.9177) ..
  controls (146.9856,141.7619) and (136.8238,142.9939) .. (134.3598,143.9177) ..
  controls (131.8959,144.8416) and (129.7406,145.7654) .. (119.5782,147.3050) ..
  controls (109.4159,148.8446) and (106.6443,149.6146) .. (101.4089,148.9984) ..
  controls (97.1167,148.4931) and (102.0255,159.3146) .. (104.1809,160.5466) ..
  controls (106.3362,161.7785) and (106.6443,163.9343) .. (120.5021,161.1628) --
  cycle;
\path[draw=black,fill=ce5d4b1,line width=1.109pt] (143.7845,133.4473) ..
  controls (146.1343,135.9107) and (145.1269,138.3746) .. (144.4557,138.9904) ..
  controls (143.7845,139.6061) and (136.0620,142.9934) .. (132.0331,144.2254) ..
  controls (128.0037,145.4573) and (127.3326,144.5335) .. (119.2752,146.3812) ..
  controls (111.2174,148.2289) and (110.5462,148.8446) .. (108.1955,149.7685) ..
  controls (105.8452,150.6923) and (101.1451,150.3842) .. (96.7808,146.9969) ..
  controls (92.4161,143.6096) and (93.0873,137.7584) .. (93.4231,135.9107) ..
  controls (93.7589,134.0630) and (94.0952,130.0595) .. (102.4884,129.4438) ..
  controls (122.7725,131.9811) and (136.5302,126.1649) .. (143.7845,133.4473) --
  cycle;
\path[draw=c897f6a,line cap=round,miter limit=4.00,line width=0.739pt]
  (95.5577,135.2950) .. controls (94.6694,143.2886) and (96.0699,143.6863) ..
  (100.4850,146.9969);
\path[draw=c897f6a,line cap=round,miter limit=4.00,line width=0.739pt]
  (109.1078,135.4483) .. controls (109.1078,137.9118) and (109.7235,143.4549) ..
  (109.7235,143.4549);
\path[draw=c897f6a,line cap=round,miter limit=4.00,line width=0.739pt]
  (124.3513,134.6783) .. controls (124.3513,135.9102) and (124.9675,140.8371) ..
  (124.9675,142.6848);
\path[draw=c897f6a,line cap=round,miter limit=4.00,line width=0.739pt]
  (127.4305,148.5370) .. controls (127.4305,150.3847) and (130.5102,156.5435) ..
  (130.5102,156.5435);
\path[draw=c897f6a,line cap=round,miter limit=4.00,line width=0.739pt]
  (112.4951,152.0781) .. controls (113.1108,153.3100) and (113.7270,158.8527) ..
  (115.5747,159.4689);
\path[draw=c897f6a,line cap=round,miter limit=4.00,line width=0.739pt]
  (169.4657,135.4488) .. controls (166.3865,140.9919) and (164.5388,155.1582) ..
  (161.4591,158.2374);
\path[draw=c963c96,line cap=round,line width=3.326pt] (131.4340,30.1298) --
  (131.1254,7.4955);
\path[cm={{0.46193,0.0,0.0,0.46193,(-17.99047,-11.80742)}},fill=c963c96]
  (322.8140,0.0000) -- (346.1270,60.3000) -- (322.8140,41.7880) --
  (299.5010,60.3000) -- cycle;
\path[draw=c2828ff,line cap=round,line width=3.326pt] (23.1190,85.4094) --
  (0.9023,81.0705);
\path[cm={{0.46193,0.0,0.0,0.46193,(-17.99047,-11.80742)}},fill=c2828ff]
  (0.0000,192.5000) -- (63.7990,182.0440) -- (40.9000,201.0670) --
  (54.2400,227.6800) -- cycle;
\path[draw=ce12828,line cap=round,line width=3.326pt] (40.3862,143.0553) --
  (26.1627,148.9449);
\path[cm={{0.46193,0.0,0.0,0.46193,(-17.99047,-11.80742)}},fill=ce12828]
  (69.0630,358.1970) -- (96.6050,315.5680) -- (95.5850,348.0050) --
  (118.0640,371.4130) -- cycle;
\begin{scope}[cm={{0.46193,0.0,0.0,0.46193,(-17.99047,-11.80742)}}]
\end{scope}
\path[fill=black,line join=miter,line cap=butt,line width=0.800pt]
  (-2.2839,61.9362) node[above right] (text4204) {$x$-axis};
\path[fill=black,line join=miter,line cap=butt,line width=0.800pt]
  (9.0458,132.6143) node[above right] (text4204-3) {$y$-axis};
\path[fill=black,line join=miter,line cap=butt,line width=0.800pt]
  (146.0527,6.4312) node[above right] (text4204-7) {$z$-axis};
\end{tikzpicture}\footnote{ Image credit: Acdx, from Wikipedia \url{https://en.wikipedia.org/wiki/Cross_product}}
\end{center}

We now have three standard basis vectors, $\xhat$, $\yhat$,
and $\zhat$, each pointing one unit in the positive direction of their
respective axes.
Any vector in three-dimensional
space can be represented
in exactly one way
as a linear combination $\alpha\xhat+\beta\yhat+\gamma\zhat$.  Thus,
vectors in three-dimensional space, notated $\R^3$,
are synonymous with triplets $(\alpha,\beta,\gamma)$
of real numbers.  With some clever geometry, we deduce
\[
	\norm{\alpha\xhat+\beta\yhat+\gamma\zhat}=\sqrt{\alpha^2+\beta^2+\gamma^2}.
\]

Historically, three-dimensional space has been studied a lot and there
are several notations for the standard basis vectors still in use.

The following is a non-exhaustive list.
\begin{center}
	\begin{tabular}{c  c  c}
		$\hat{\mathbf{x}}$ & $\hat{\mathbf{y}}$ &$\hat{\mathbf{z}}$\\
		$\hat{\imath}$ & $\hat{\jmath}$ &$\hat{k}$\\
		$\mathbf{i}$ & $\mathbf j$ & $\mathbf k$\\
		$\vec e_1$ & $\vec e_2$ & $\vec e_3$
	\end{tabular}
\end{center}
Keep these notations in the back of your mind.  You might see them in other classes.

\subsection{Higher dimensions}
One can't progress very far in the study of science and mathematics
without encountering a need for higher dimensional ``vectors.''  For
example, physicists have known since Einstein that the physical
universe is best thought of as a four-dimensional entity called
spacetime in which time plays a role close to that of the
three spatial coordinates.  Since we don't have any way to deal with
$\R^n$\index{$\R^n$}
intuitively, we must
proceed by analogy with two and three dimensions.
The easiest
way to proceed is to generalize the idea of a standard basis.
From there, we can represent vectors in $\R^n$ as $n$-tuples of real numbers.
We then define
\[
	\norm{(x_1,x_2,\ldots,x_n)} = \sqrt{x_1^2+x_2^2+\cdots+x_n^2}.
\]
We've now unified our theory of vectors across all integer dimensions $n>0$.
The case $n=1$ yields  ``geometry'' on a line,
the cases $n = 2$ and $n = 3$ geometry in the plane and in space, and
the case $n = 4$ yields the geometry of ``4-vectors'' which
are  used in the special theory of relativity.
Larger values of $n$ are used in a
variety of contexts, some of which we shall encounter later.

\begin{exercises}
	\begin{problist}
        \prob  Find $\norm{\vec{a}}$, $5\vec a-2\vec b$, and $-3\vec b$ for each of the following
			vector pairs.
			\begin{enumerate}
				\item   $\vec a=2\xhat+3\yhat$, $\vec b=4\xhat-9\yhat$
				\item   $\vec a=(1,2,-1)$, $\vec b=(2,-1,0)$
			\end{enumerate}
            \begin{solution}
                \begin{enumerate}
                    \item   $\norm{\vec{a}} = \sqrt{13}$, $5\vec{a} - 2\vec{b} = 2\xhat + 33\yhat$,
                        $-3\vec{b} = -12\xhat + 27\yhat$.
                    \item   $\norm{\vec{a}} = \sqrt{6}$, $5\vec{a} - 2\vec{b} = (1, 12, -5)$, $-3\vec{b}
                        = (-6, 3, 0)$.
                \end{enumerate}
            \end{solution}

		\prob  Let $P=(7,2,9)$ and $Q=(-2,1,4)$.  Find $\overrightarrow{PQ}$ as a linear
			combination of $\xhat$, $\yhat$, and $\zhat$.
            \begin{solution}
                $\overrightarrow{PQ} = -9\xhat -\yhat - 5\zhat$.
            \end{solution}

		\prob  Find unit vectors with the same direction as the vectors a) $(-4, -2)$,
			b) $3\xhat + 5\yhat$, c) $(1, 3, -2)$.
            \begin{solution}
                \begin{enumerate}
                    \item   $\frac{1}{\sqrt{5}}(-2, -1)$
                    \item   $\frac{1}{\sqrt{34}}(3, 5)$
                    \item   $\frac{1}{\sqrt{14}}(1, 3, -2)$
                \end{enumerate}
            \end{solution}

		\prob  Show by direct calculation that the rule $\overrightarrow{AB} +
            \overrightarrow{BC} + \overrightarrow{CA} = \vec{0}$ holds for the three points
			$A(2, 1, 0)$, $B(-4, 1, 3)$, and $C(0, 12, 0)$.  Can you prove the general rule
			for any three points in space?
            \begin{solution}
                In the specific case, $\overrightarrow{AB} = (-6, 0, 3)$, $\overrightarrow{BC} = (4, 11,
                -3)$, $\overrightarrow{CA} = (2, -11, 0)$, so that 
                \[
                    \overrightarrow{AB} + \overrightarrow{BC} + \overrightarrow{CA} = \mat{-6 + 4 + 2\\
                    0 + 11 - 11\\ 3 - 3 + 0} = \mat{0\\0\\0}
                \]
                More generally, suppose that we have the points $A(a_1, a_2, a_3)$, $B(b_1, b_2, b_3)$,
                and $C(c_1, c_2, c_3)$. Then $\overrightarrow{AB} = (b_1 - a_1, b_2 - a_2, b_3 - a_3)$,
                $\overrightarrow{BC} = (c_1 - b_1, c_2 - b_2, c_3 - b_3)$, and $\overrightarrow{CA} =
                (a_1 - c_1, a_2 - c_2, a_3 - c_3)$. Thus, 
                \[
                    \overrightarrow{AB} + \overrightarrow{BC} + \overrightarrow{CA} = 
                    \mat{(b_1 - a_1) + (c_1 - b_1) + (a_1 - c_1)\\ (b_2 - a_2) + (c_2 - b_2) + (a_2 -
                    c_2) \\ (b_3 - a_3) + (c_3 - b_3) + (a_3 - c_3)} = \mat{0\\0\\0}
                \]
            \end{solution}

		\prob  If an airplane flies with apparent velocity $\vec{v}_a$ relative to air, and the
			wind velocity is denoted $\vec{w}$, then the planes true velocity relative to
			the ground, is $\vec{v}_g = \vec{v}_a + \vec{w}$.  Draw the diagram to assure
			yourself of this.
			\begin{enumerate}
				\item   A farmer wishes to fly his crop duster at 80 km$/$h north over
					his fields.  If the weather vane atop the barn shows easterly
					winds at 10 km$/$h, what should his apparent velocity,
					$\vec{v}_a$ be?
				\item   What if the wind were northeasterly?  Southeasterly?
			\end{enumerate}
            \begin{solution}
                \begin{enumerate}
                    \item   $10 \sqrt{65}$ km/h, $\tan^{-1}\left(\frac18\right)^\circ$ east of north.
                    \item   for a northeasterly wind:
                        $2\sqrt{\left(40+\frac{5\sqrt{2}}{2}\right)^2 + \frac{25}{2}}$ at
                        $\tan^{-1}\left(\frac{5\sqrt{2}}{80+5\sqrt{2}}\right)^\circ$
                        east of north 

                        for a southeasterly wind:
                        $2\sqrt{\left(40-\frac{5\sqrt{2}}{2}\right)^2 + \frac{25}{2}}$ at
                        $\tan^{-1}\left(\frac{5\sqrt{2}}{80-5\sqrt{2}}\right)^\circ$
                        east of north
                \end{enumerate}
            \end{solution}

		\prob  Suppose a right handed coordinate system has been set up in space.  What happens
			to the orientation of the coordinate system if you make the following changes?
            \begin{enumerate}
                \item   Change the direction of one axis.
                \item   Change the direction of two axes.
                \item   Change the direction of all three axes.
                \item   Interchange the $x$ and $y$-axes.
            \end{enumerate}
            \begin{solution}
                \begin{enumerate}
                    \item   the orientation changes.
                    \item   the orientation stays the same.
                    \item   the orientation changes.
                    \item   the orientation changes.
                \end{enumerate}
            \end{solution}

		\prob  In body-centered crystals, a large atom (assumed spherical) is surrounded by
			eight smaller atoms.  If the structure is placed in a "box" that just contains
			the large atom, the smaller atoms each occupy a corner.  If the central atom has
			radius $R$, what is the greatest atomic radii the smaller atoms may have?
            \begin{solution}
                $(2 - \sqrt{3})R$
            \end{solution}

		\prob  Prove that the diagonals of a parallelogram bisect each other.  (Hint: show that
			the position vectors from the origin to the midpoints are equal).
            \begin{solution}
                Suppose that the parallelogram is defined by two vectors $\vec{a}$ and $\vec{b}$. Then
                the first diagonal (which originates from the origin) is given by $\vec{d}_1 = \vec{a} +
                \vec{b}$ and the second diagonal (which originates from the tip of $\vec{b}$) is given
                by $\vec{d}_2 = \vec{a} - \vec{b}$. Then, the position of the midpoint of the first
                diagonal is given by $\vec{d}_1 / 2 = 1/2(\vec{a} + \vec{b})$ and the position of the
                midpoint of the second diagonal is given by $\vec{b} + \vec{d}_2 / 2 = \vec{b} +
                1/2(\vec{a} - \vec{b}) = 1/2(\vec{a} + \vec{b})$. Thus, the position vectors of the
                midpoints are equal, and so the diagonals must bisect each other.
            \end{solution}

	\end{problist}
\end{exercises}



\section{Lines and Planes}

With a handle on vectors, we can now use them to describe some common geometric
objects: lines and planes.

\subsection{Lines}
Consider for a moment the line $\ell$ through the points $P$ and $Q$.  When $P,Q\in\R^2$, we
can describe $\ell$ with an equation of the form $y=mx+b$ (provided it isn't a vertical line), but if
$P,Q\in\R^3$, it's much harder to describe $\ell$ with an equation.  Using vectors
provides an easier way.

Let $\vec d=\overrightarrow{PQ}$ and consider the set of points (or vectors) $\vec x$ that can be expressed as
\[
	\vec x=t\vec d+P
\]
for $t\in \R$.  Geometrically, this is the set of all points we get by starting at $P$ and
displacing by some multiple of $\vec d$.  This is a line!


\begin{center}
	\begin{tikzpicture}
		\coordinate (A) at (1,1);
		\coordinate (B) at (3,2);
		\coordinate (D) at ($(B)-(A)$);
		\begin{axis}[
		    anchor=origin,
		    disabledatascaling,
		    xmin=-1,xmax=5,
		    ymin=-1,ymax=3,
		    x=1cm,y=1cm,
		    grid=both,
		    grid style={line width=.1pt, draw=gray!10},
		    %major grid style={line width=.2pt,draw=gray!50},
		    axis lines=middle,
		    minor tick num=0,
		    enlargelimits={abs=0.5},
		    axis line style={latex-latex},
		    ticklabel style={font=\tiny,fill=white},
		    xlabel style={at={(ticklabel* cs:1)},anchor=north west},
		    ylabel style={at={(ticklabel* cs:1)},anchor=south west}
		]

		\draw [mypink,fill] (A) circle (1.5pt) node [below right] {$P$};
		\draw [mypink,fill] (B) circle (1.5pt) node [below right] {$Q$};
		\draw[->,thick,myred!60!white] (A) -- (B) node [midway,below right,yshift=2pt] {$\vec d$};


		\end{axis}
		\foreach \x in {-1,5} {
			\draw [mypink,fill] ($(A)+\x/3*(D)$) circle (1.5pt) node [left] {\footnotesize$\tfrac{\x}{3}\vec d+P$};
		}
		\foreach \x in {-3,-2,4,6} {
			\draw [mypink,fill] ($(A)+\x/3*(D)$) circle (1.5pt);
		}
	\end{tikzpicture}
	\begin{tikzpicture}
		\coordinate (A) at (1,1);
		\coordinate (B) at (3,2);
		\coordinate (D) at ($(B)-(A)$);
		\begin{axis}[
		    anchor=origin,
		    disabledatascaling,
		    xmin=-1,xmax=5,
		    ymin=-1,ymax=3,
		    x=1cm,y=1cm,
		    grid=both,
		    grid style={line width=.1pt, draw=gray!10},
		    %major grid style={line width=.2pt,draw=gray!50},
		    axis lines=middle,
		    minor tick num=0,
		    enlargelimits={abs=0.5},
		    axis line style={latex-latex},
		    ticklabel style={font=\tiny,fill=white},
		    xlabel style={at={(ticklabel* cs:1)},anchor=north west},
		    ylabel style={at={(ticklabel* cs:1)},anchor=south west}
		]

		\draw [mypink,fill] (A) circle (1.5pt) node [below right] {$P$};
		\draw [mypink,fill] (B) circle (1.5pt) node [below right] {$Q$};
		\draw[->,thick,myred!60!white] (A) -- (B) node [midway,below right,yshift=2pt] {$\vec d$};

		\end{axis}

		\foreach \x in {-3,-2,-1,0,4,5,6} {
			\draw [->, gray!50!white] (0,0) -- ($(A)+\x/3*(D)$);
		}
		\foreach \x in {-1,5} {
			\draw [mypink,fill] ($(A)+\x/3*(D)$) circle (1.5pt) node [left] {\footnotesize$\tfrac{\x}{3}\vec d+P$};
		}
		\foreach \x in {-3,-2,4,6} {
			\draw [mypink,fill] ($(A)+\x/3*(D)$) circle (1.5pt);
		}
	\end{tikzpicture}
\end{center}
We simultaneously interpret this line as a set of points (the points that make up
the line) and as a set of vectors 
rooted at the origin (the vectors pointing from the origin to the line).
Note that sometimes we draw vectors as directed line segments from the origin.
Other times, drawing drawing vectors as line segments makes
it hard to see what is going on, and so it is better to draw
each vector by marking only its ending point.


The line $\ell$ described above can be written in set-builder notation
as:
\[
	\ell=\Set{\vec x\given \vec x=t\vec d+P\text{ for some }t\in \R}.
\]
Notice that in set-builder notation, we write ``for some $t\in \R$.'' Make sure you
understand why replacing ``for some $t\in\R$'' with ``for
all $t\in \R$'' would be incorrect.

Writing lines with set-builder notation all the time can be overkill,
so we will allow ourselves to describe lines in a shorthand called \emph{vector form}\footnote{
	$y=mx+b$ form of a line is also shorthand.  The line $\ell$ described by the equation
	$y=mx+b$ is actually the set $\Set{(x,y)\in\R^2\given y=mx+b}$.
}.

\begin{definition}[Vector form of a Line]
	Let $\ell$ be a line and let $\vec d$ and $\vec p$ be vectors such that
	$\ell=\Set{\vec x\given \vec x=t\vec d+\vec p\text{ for some }t\in\R}$. Then,
	$\ell$ may be  described in \emph{vector form}\index{vector form of line} as
	\[
		\vec x=t\vec d+\vec p.
	\]
	We call $\vec d$ a
	\emph{direction vector} for $\ell$ and the equation $\vec x=t\vec d+\vec p$ a
	\emph{vector equation} or \emph{vector form}
	of $\ell$.
\end{definition}

Note that if $\vec x=t\vec d+\vec p$ is the vector equation of a line $\ell$, by setting $t=0$
we necessarily have $\vec p\in\ell$. The converse is true, too. If $\vec q\in \ell$ and 
$\vec d$ is a direction for $\ell$, then $\ell$ may be expressed in vector form as $\vec x=t\vec d+\vec p$.

The direction of a line is easily obtained by finding the displacement vector between two points
on the line.  Thus, given a line in another form, computing its vector form is straightforward.
\begin{example}
	Find vector form of the line $\ell\subseteq\R^2$ with equation $y=2x+3$.

	First, we find two
	points on the line.  By guess-and-check we see $P=(0,3)$ and $Q=(1,5)$ are on $\ell$.
	Thus, a direction vector for $\ell$ is given by
	\[
		\vec d = (1,5)-(0,3)=(1,2).
	\]
	We may now write the vector equation of $\ell$ as
	\[
		\vec x=t\vec d+P
	\]
	or, in components,
	\[
		\mat{x\\y} = t\mat{1\\2}+\mat{0\\3}.
	\]
\end{example}

It's important to note that when we write a line in vector form, it is a \emph{specific shorthand} notation.
If we augment the notation, we no longer have written a line in ``vector form.''

\begin{example} Let $\ell$ and let $\vec d$ be a direction vector for $\ell$ and let $\vec p\in \ell$
	be a point on $\ell$. Writing
	\[
		\vec x=t\vec d+\vec p
	\]
	or 
	\[
		\vec x=t\vec d+\vec p\quad\text{ where }\quad t\in \R
	\]
	specifies $\ell$ in vector form and are both shorthands for $\Set{\vec x\given
	\vec x=t\vec d+\vec p\text{ for some }t\in \R}$. But,
	\[
		\vec x=t\vec d+\vec p\quad\text{ for some }\quad t\in \R
	\]
	and
	\[
		\vec x=t\vec d+\vec p\quad\text{ for all }\quad t\in \R
	\]
	are logical statements about the vectors $\vec x$, $\vec d$, and
	$\vec p$. These statements are either true or false; they do \emph{not} 
	specify $\ell$ in vector form.

	Similarly, the statement
	\[
		\ell = t\vec d+\vec p
	\]
	is mathematically nonsensical and does not specify $\ell$ in vector form. (On the
	left is a \emph{set} and on the right is a \emph{vector}!)

\end{example}

The downside of writing lines in vector form is that there are multiple direction vectors and multiple points
for every line.  Thus, merely by looking at the vector equation for two lines, it can be hard to tell if
they're equal.

For example,
\[
	\mat{x\\y} = t\mat{1\\2}+\mat{0\\3},\qquad
	\mat{x\\y} = t\mat{2\\4}+\mat{0\\3},\quad\text{and}\quad
	\mat{x\\y} = t\mat{1\\2}+\mat{1\\5}
\]
all represent the same line.  In the second equation, the direction is parallel but scaled, and in
the third equation, a different point on the line was chosen.

In vector form, the variable $t$ is called the \emph{parameter variable}.  It is an instance of
a \emph{dummy variable}; that is, it is mostly there as a placeholder.  Remember, vector
form is shorthand for a set described in set-builder notation.

\bigskip
Let $\vec d_1,\vec d_2\neq\vec 0$ and $\vec p_1,\vec p_2$ be vectors and define the lines
\[
	\ell_1=\Set{\vec x\given \vec x=t\vec d_1+\vec p_1\text{ for some }t\in\R}
\]
\[
	\ell_2=\Set{\vec x\given \vec x=t\vec d_2+\vec p_2\text{ for some }t\in\R}.
\]
These lines have vector equations $\vec x=t\vec d_1+\vec p_1$ and $\vec x=t\vec d_2+\vec p_2$.
However, declaring that $\ell_1=\ell_2$ if and only if $t\vec d_1+\vec p_1=t\vec d_2+\vec p_2$
does \emph{not} make sense.   Instead $\ell_1=\ell_2$ if $\ell_1\subseteq\ell_2$ and $\ell_2\subseteq\ell_1$.
If $\vec x\in\ell_1$ then $\vec x=t\vec d_1+\vec p_1$ for some $t\in\R$.  If $\vec x\in\ell_2$
then $\vec x=t\vec d_2+\vec p_2$ for some \emph{possibly different} $t\in \R$.  This can get
confusing really quickly.  The easiest solution is to use different parameter variables if
we want to compare lines in vector form.

\begin{example}
	Determine if the lines $\ell_1$ and $\ell_2$, represented in vector form by the equations
	\[
		\vec x=t\mat{1\\1}+\mat{2\\1}\qquad\text{and}\qquad
		\vec x=t\mat{2\\2}+\mat{4\\3}
	\]
	are the same line.  To determine this, we need to figure out if $\vec x\in\ell_1$
	implies $\vec x\in \ell_2$ and if $\vec x\in\ell_2$ implies $\vec x\in\ell_1$.

	If $\vec x\in\ell_1$, then $\vec x=t\mat{1\\1}+\mat{2\\1}$ for some $t\in\R$.  If
	$\vec x\in\ell_2$, then $\vec x=s\mat{2\\2}+\mat{4\\3}$ for some $s\in \R$.  Thus if
	\[
		t\mat{1\\1}+\mat{2\\1} = \vec x = s\mat{2\\2}+\mat{4\\3}
	\]
	always has a solution, $\ell_1=\ell_2$.  Moving everything to one side we see
	\begin{align*}
		\vec 0 = \mat{4\\3}-\mat{2\\1} + s\mat{2\\2}-t\mat{1\\1}
		&=\mat{2\\2}+s\mat{2\\2}-t\mat{1\\1}\\
		&=(s+1)\mat{2\\2}-\tfrac{t}{2}\mat{2\\2}\\
		&= (s+1-\tfrac{t}{2})\mat{2\\2}.
	\end{align*}
	This has a solution whenever $0=s+1-t/2$.  Since for every $t\in\R$ we can find an $s\in\R$
	and for every $s\in \R$ we can find a $t\in \R$ satisfying this equation, we know $\ell_1=\ell_2$.
\end{example}


The geometry of lines in space is a bit more complicated than that of lines
in the plane.  Lines in the plane either intersect or are parallel.
In space,  we have to be a bit more careful about what we mean by
``parallel lines,'' since lines with entirely different directions can
still fail to intersect\footnote{ Recall that in Euclidean geometry
two lines are defined to be parallel if they coincide or never intersect.}.

\begin{center}
  \begin{tikzpicture}
    \begin{axis}[grid=major,view={20}{40},z buffer=sort,
	    zmin=0,
	    xticklabels={,,}, yticklabels={,,}, zticklabels={,,}
	    ]
	    \addplot3 [no marks,orange,ultra thick] coordinates {(0,10,10) (20,10,30)};
	    \addplot3 [no marks,orange,dashed, thick] coordinates {(0,10,0) (20,10,0)};
	    \addplot3 [no marks,mypink,ultra thick] coordinates {(0,0,20) (20,20,0)};
	    \addplot3 [no marks,mypink,dashed, thick] coordinates {(0,0,0) (20,20,0)};
	%\addplot3[domain=4:30,samples=80,samples y=0,mark=none,orange, opacity=0.5,ultra thick]
	%    ({x},{118.89/x},{2*x});
    \end{axis}
  \end{tikzpicture}
\end{center}

\begin{example}
Consider the lines described by
\begin{align*}
\vec x &= t( 1, 3, -2 ) + ( 1, 2, 1 ) \\
\vec x &= t( -2, -6, 4) + ( 3, 1, 0 ).
\end{align*}
They have parallel directions since $( -2, -6, 4 ) = -2( 1, 3,-2 )$.
Hence, in this case, we say the lines are \emph{parallel}\index{parallel lines}.  (How can
we be sure the lines are not the same?)
\end{example}

\begin{example}
Consider the lines described by
\begin{align*}
	\vec x &= t(1, 3, -2 ) + ( 1, 2, 1 ) \\
	\vec x &= t( 0, 2, 3) + ( 0, 3, 9 ).
\end{align*}
They are not parallel because neither of the direction
vectors is a multiple
of the other.  They may or may not intersect.  (If they don't,
	we say the lines are \emph{skew}\index{skew lines}.)  How can we find out?
	Mirroring our earlier approach,
	we can set their equations equal and see if we can solve for the point
	of intersection \emph{after ensuring we give their parametric variables
	different names}.   We'll keep one parametric variable named $t$ and name the
	other one $s$.  Thus, we want
\[
\vec x = t( 1, 3, -2 ) + ( 1, 2, 1 ) =
s( 0, 2, 3) + ( 0, 3, 9 ),
\]
which after collecting terms yields
\[
    ( t + 1, 3t + 2, -2t + 1 ) = ( 0, 2s + 3, 3s + 9).
\]
Picking out the components yields three equations
\begin{align*}
    t + 1 &= 0 \\
    3t +2 &= 2s + 3 \\
    -2t + 1 &=  3s + 9
\end{align*}
in 2 unknowns  $s$ and $t$.  This is an {\it overdetermined\/}
system, and it may or may not have a consistent solution.
The first two equations yield $t = -1$  and $s = -2$.  Putting
these values in the last equation yields $(-2)(-1) + 1 = 3(-2) + 9$,
which is indeed true.
Hence, the equations are consistent, and the lines
intersect.   To find the point of intersection, put $t = -1$
in the equation for the first line (or
$s = -2$ in that for the second) to obtain  $( 0, -1, 3 )$.
\end{example}

\subsection{Planes}

Any two distinct points define a line.  To define a plane, we
need three points.  But there's a caveat: the three points cannot
be on the same line, otherwise they'd define a line
and not a plane.  Let $A,B,C\in\R^3$ be three points that are not
collinear and let $\mathcal P$ be the plane that passes through $A$,
$B$, and $C$.

Just like lines, planes have direction vectors.  For $\mathcal P$, both
$\vec d_1=\overrightarrow{AB}$ and $\vec d_2=\overrightarrow{AC}$ are direction
vectors for $\mathcal P$.  Of course, $\vec d_1$, $\vec d_2$ and their multiples
are not the only direction vectors for $\mathcal P$. There are infinitely many more, including
$\vec d_1+\vec d_2$, and $\vec d_1-7\vec d_2$, and so on.  However, since a plane
is a \emph{two}-dimensional object, we only need two different direction vectors to describe it.

Again like lines, planes have a vector form.  $\mathcal P$ can be written in vector form as
\[
	\mat{x\\y\\z} = t\vec d_1+s\vec d_2+A.
\]
Vector form of $\mathcal P$ is not unique.  Any two different directions in $\mathcal P$
suffice for defining $\mathcal P$ in vector form.

\begin{center}
  \begin{tikzpicture}
    \begin{axis}[grid=major,view={20}{40},z buffer=sort,
	    %zmin=0,
	    xticklabels={,,}, yticklabels={,,}, zticklabels={,,}
	    ]
		\addplot3 [data cs=cart,surf,domain=-10:10,samples=2, opacity=0.5]
		{x+y};
		\coordinate (A) at (axis cs:-3,-3,-6);
		\coordinate (B) at (axis cs:3,4,7);
		\coordinate (C) at (axis cs:-4,4,0);

		\draw [mypink,fill] (A) circle (1.5pt) node [below right] {$A$};
		\draw [->, thick] (A) -- (B) node [midway,below right] {$\vec d_1$};
		\draw [->, thick] (A) -- (C) node [midway,above left] {$\vec d_2$};
    \end{axis}
  \end{tikzpicture}
\end{center}

\begin{definition}[Vector form of a plane]
	Let $\mathcal P$ be a plane and let $\vec d_1$, $\vec d_2$, and $\vec p$ be vectors such
	that $\mathcal P=\Set{\vec x\given \vec x=t\vec d_1+s\vec d_2+\vec p\text{ for some }t,s\in\R}$.
	We then say the equation
	\[
		\vec x=t\vec d_1+s\vec d_2+\vec p
	\]
	describes $\mathcal P$ in \emph{vector form}.
	The vectors $\vec d_1$
	and $\vec d_2$ are called \emph{direction vectors} for the plane $\mathcal P$.
\end{definition}

\begin{example}
	Describe the plane $\mathcal P\subseteq \R^3$ with equation $z=2x+y+3$ in vector form.

	To describe $\mathcal P$ in vector form, we need a point on $\mathcal P$ and two direction
	vectors for $\mathcal P$. By guess-and-check, we see the points
	\[
		A=\mat{0\\0\\3}\qquad B=\mat{1\\0\\5}\qquad C=\mat{0\\1\\4}
	\]
	are all on $\mathcal P$. Thus
	\[
		\vec d_1=B-A=\mat{1\\0\\2}\qquad \text{and}\qquad
		\vec d_2=C-A=\mat{0\\1\\1}
	\]
	are both direction vectors for $\mathcal P$.  Therefore, we can express $\mathcal P$ in vector
	form as
	\[
		\vec x=t\vec d_1+s\vec d_2+A=t\mat{1\\0\\2}+s\mat{0\\1\\1}+\mat{0\\0\\3}.
	\]
\end{example}




\begin{exercises}
    \begin{problist}
        \prob   Give, in vector form, an equation for the line that
        \begin{enumerate}
            \item   passes through $(0, 0, 0)$ and is parallel to $\vec{v} = 3\xhat +4\yhat + 5\zhat$,
            \item   passes through $(1, 2, 3)$ and $(4, -1, 2)$,
            \item   passes through $(1, 1)$ and is orthogonal to $\vec{v} = (3, 1)$,
            \item   passes through $(9, -2, 3)$ and $(1, 2, 3)$.
        \end{enumerate}
        \begin{solution}
            \begin{enumerate}
                \item   $r(t) = 3t\xhat + 4t\yhat + 5t\zhat$
                \item   $r(t) = (1+3t)\xhat + (2-3t)\yhat + (3-t)\zhat$
                \item   $r(t) = (1-t)\xhat + (1+3t)\yhat$
                \item   $r(t) = (1+8t)\xhat + (2-4t)\yhat + 3\zhat$.
            \end{enumerate}
        \end{solution}

        \prob   Determine if the lines with the following vector equations intersect:
        \[
            \vec{p}(t) = \mat{1\\-1\\2} + t\mat{2\\1\\1} \quad
            \vec{r}(t) = \mat{0\\1\\1} + t\mat{1\\0\\-1}.
        \]
        \begin{solution}
            The lines intersect if they have at least one common point between them. Therefore, if we
            can find $t, s \in \R$ so that $p(t) = r(s)$, then the lines will intersect. Thus if $p(t) =
            r(s)$ then
            \[
                \mat{1\\-1\\2} + t\mat{2\\1\\1} = \mat{0\\1\\1} + s\mat{1\\0\\-1}.  
            \]
            Therefore, we want to find $t, s \in \R$ so that
            \[
                \mat{1\\-2\\1} = s\mat{1\\0\\-1} - t\mat{2\\1\\1}.
            \]
            In particular, it must be true that $-t = -2$, which says that $t = 2$. Thus, by
            substitution
            \[
                \mat{5\\0\\3} = s\mat{1\\0\\-1}.  
            \]
            There are no selections of $s \in \R$ for which this is true, which says that there is no
            pair $t,s \in \R$ so that $p(t) = r(s)$. Therefore the lines do not intersect.
        \end{solution}

        \prob   Find an equation for the plane with the given normal vector and containing the given
        point:
        \begin{enumerate}
            \item   $\vec{n} = (2, -1, 3)$, $P(1, 2, 0)$
            \item   $\vec{n} = (1, 0, 3)$, $P(2, 4, 5)$
        \end{enumerate}
        \begin{solution}
            \begin{enumerate}
                \item   $2(x-1) - (y-2) + 3z = 0$.
                \item   $(x-2) + 3(z-5) = 0$.
            \end{enumerate}
        \end{solution}

        \prob   Write an equation for the plane that
        \begin{enumerate}
            \item   passes through $(1, 4, 3)$ and is orthogonal to the line with equation $\vec{r}(t) =
                (1 + t, 2 + 4t, t)$,
            \item   passes through the origin and is parallel to the plane with equation $3x + 4y - 5z =
                -1$,
            \item   passes through $(0, 0, 0)$, $(1, -2, 8)$, and $(-2, -1, 3)$.
        \end{enumerate}
        \begin{solution}
            \begin{enumerate}
                \item   $x + 4y + z = 20$
                \item   $3x + 4y -5z = 0$
                \item   $2x - 19y -5z = 0$.           
            \end{enumerate}
        \end{solution}

        \prob   Find a vector (parametric) equation for the line of intersection of the planes with
        equations $2x + 3y - z = 1$ and $x - y - z = 0$.
        \begin{solution}
            $\vec{r}(t) = (1-4t, t, 1-5t)$ is one solution.
        \end{solution}

        \prob   Find the angle between the normals to the following planes:
        \begin{enumerate}
            \item   the planes with equations $x + 2y -z = 2$ and $2x - y + 3z = 1$,
            \item   the plane with equation $2x+3y-5z=0$ and the plane containing the points $(1,3,-2)$,
                $(5,1,3)$, and $(1,0,1)$.
        \end{enumerate}
        \begin{solution}
            \begin{enumerate}
                \item   $\theta = \cos^{-1}\left(\frac{-3}{2\sqrt{21}}\right)$
                \item   $\theta = \cos^{-1}\left(\frac{14}{\sqrt{38}\,\sqrt{41}}\right)$
            \end{enumerate}
        \end{solution}

        \prob[\grout]    Two of these points are on the same side of the plane $3x+2y-2z=0$. Which two
        are they? $P(10,3,2)$, $Q(-2,5,3)$, $R(-2,5,1)$.
        \begin{solution}
            The normal vector $\vec{n} = (3,2,-2)$ points towards one side of the plane. If any point
            lies on this side of the plane, then it will form an angle with the normal vector that is
            less than $90^{\circ}$. If any point lies on the opposite side of the plane, then it will
            form an angle with the normal vector that is larger than $90^{\circ}$. So we should
            calculate the angles each of $P,Q$ and $R$ make with $\vec{n}$ using the dot-product.
            \begin{enumerate}
                \item   Considering $P$, we find that $\theta = \cos^{-1} \left(\frac{32}{\sqrt{113}\,
                    \sqrt{17}}\right) < 90^{\circ}$.
                \item   Considering $Q$, we find that $\theta = \cos^{-1}
                    \left(\frac{-2}{\sqrt{38}\,\sqrt{17}}\right) > 90^{\circ}$.
                \item   Considering $R$, we find that $\theta = \cos^{-1}
                    \left(\frac{2}{\sqrt{30}\,\sqrt{17}}\right) < 90^{\circ}$.
            \end{enumerate}
            Since $P$ and $R$ both make angles with $\vec{n}$ that are larger than $90^{\circ}$, they
            must lie on the same side of the plane.
        \end{solution}
    \end{problist}
\end{exercises}



\section{Geometry \& Sets}

Using vectors, we can describe more than just lines and planes---we can describe 
all sorts of geometric objects.

Recall that vector form of a line is actually a shorthand. When we write $\vec x=t\vec d+\vec p$
to describe the line $\ell$,
what we mean is
\[
	\ell=\Set{\vec x\given \vec x=t\vec d+\vec p\text{ for some }t\in \R}.
\]
We could notate a portion of this line by restricting $t$. For instance, consider the ray $R$
and the line segment $S$:
\begin{align*}
	R&=\Set{\vec x\given \vec x=t\vec d+\vec p\text{ for some }t\geq 0}\\
	S&=\Set{\vec x\given \vec x=t\vec d+\vec p\text{ for some }t\in [0,2]}
\end{align*}

XXX Figure

We can also make polygons by adding restrictions to the vector form of a plane. Let
$\vec a=\mat{2\\1}$ and $\vec b=\mat{-1\\1}$ and consider the unit square $U$ and
the parallelogram $P$ defined by
\begin{align*}
	U&=\Set{\vec x\given \vec x=t\xhat+s\yhat\text{ for some }t,s\in [0,1]}\\
	P&=\Set{\vec x\given \vec x=t\vec a+s\vec b\text{ for some }t\in [0,1]\text{ and }s\in[-1,1]}
\end{align*}

XXX Figure

Each set so far is a set of linear combinations, and we 
have made different shapes by restricting the coefficients of those linear
combinations. There are two ways of restricting linear combinations that arise
up often enough to get their own names.

\begin{definition}[Non-negative Linear Combination]
	Let $\vec v_1,\ldots,\vec v_n$ be vectors.
	The vector 
	\[
		\vec x=\alpha_1\vec v_1+\cdots+\alpha_n\vec v_n.
	\]
	is a \emph{non-negative linear combination}\index{non-negative linear combination} of $\vec v_1,\ldots,\vec v_n$
	if $\alpha_1,\ldots,\alpha_n\geq 0$.
\end{definition}

\begin{definition}[Convex Linear Combination]
	Let $\vec v_1,\ldots,\vec v_n$ be vectors.
	The vector 
	\[
		\vec x=\alpha_1\vec v_1+\cdots+\alpha_n\vec v_n.
	\]
	is a \emph{convex linear combination}\index{convex linear combination} of $\vec v_1,\ldots,\vec v_n$
	if $\alpha_1,\ldots,\alpha_n\geq 0$ and $\alpha_1+\cdots+\alpha_n=1$.
\end{definition}

You can think of a non-negative linear combinations as vector you can arrive at by
only displacing ``forward'' along your vectors.

Convex linear combinations can be thought of as weighted averages of vectors (the average of $\vec v_1,\ldots,
\vec v_n$ would be the convex linear combination with coefficients $\alpha_i=\frac{1}{n}$). 
A convex linear combination
of two vectors gives a point on the line segment connecting them. 

\begin{example}
	Let $\vec a=\mat{2\\1}$ and $\vec b=\mat{-1\\1}$ and define
	\begin{align*}
		A&=\Set{\vec x\given \vec x\text{ is a convex linear combination of }\vec a\text{ and }\vec b}\\
		&=\Set{\vec x\given \vec x=\alpha\vec a+(1-\alpha)\vec b\text{ for some }\alpha\in [0,1]}.
	\end{align*}
	Draw $A$.

	XXX Finish

	XXX Figure
\end{example}

\subsection{Set Operations}
We can also use set operations to define geometric objects. For instance, let $H$ be the upper
half-plane and $C$ be the unit circle. That is
\begin{align*}
	H&=\Set{\vec x\in \R^2\given \vec x=\alpha\xhat+\beta\yhat\text{ for some }
	\alpha,\beta\in \R\text{ with }\alpha\geq 0}\\
	C&=\Set{\vec x\in \R^2\given \norm{\vec x}= 1}.
\end{align*}

We can define now define the semi-circle $S=C\cap H$ using a set intersection.

XXX Figure

But what if we wanted to describe three identical circles but centered in different
locations? We could define three circles $S_1$, $S_2$, and $S_3$ and consider
their union $S_1\cup S_2\cup S_3$, or we could use the operation of \emph{set addition}
or the \emph{sum of sets}\footnote{ Recall that the result of $a+b$ is called 
the \emph{sum} of $a$ and $b$, and the mathematical operation performed to get the sum
is called \emph{addition}}.

\begin{definition}[Set Addition]
	Let $A$ and $B$ be sets of vectors. The \emph{set sum} of $A$ and $B$,
	denoted $A+B$, is the set
	\[
		A+B=\Set{\vec x\given \vec x=\vec a+\vec b\text{ for some }\vec a\in A\text{ and }\vec b\in B}.
	\]
\end{definition}

Set sums are very different than regular sums despite using the same symbol, ``$+$''\footnote{
For example, $A+\Set{}=\Set{}$, which might seem counterintuitive for an ``addition'' operation.
}.
 However, they are very useful.
Let $C=\Set{\vec x\in\R^2\given \norm{\vec x}=1}$ be the unit circle centered at the origin, and consider
the sets
\[
	X=C+\Set{\yhat}\qquad Y=C+\Set{3\xhat,\yhat}\qquad Z=C+\Set{\vec 0, \xhat,\yhat}.
\]
Rewriting, we see $X=\Set{\vec x+\yhat\given \norm{\vec x}=1}$ is just $C$ translated
by $\yhat$. Similarly, $Y=\Set{\vec x+\vec v\given \norm{\vec x}=1\text{ and }\vec v=3\xhat\text{
	or }\vec v=\yhat}=(C+\Set{3\xhat})\cup (C+\Set{\xhat})$, and so $Y$ is the union
of two translated copies of $C$\footnote{ If you want to stretch your mind, consider what $C+C$
is as a set.}.

XXX Figure

Set addition allows us to easily create parallel lines and planes by translation. For example,
consider the lines $\ell_1$ and $\ell_2$ given in vector form as $\vec x=t\vec d$ and $\vec x=t\vec d+\vec p$,
respectively, where  $\vec d=\mat{2\\1}$ and $\vec p=\mat{-1\\1}$.  These lines
differ from each other by translation, and using the idea of set addition we can write
\[
	\ell_2 = \ell_1+\Set{\vec p}.
\]

XXX Figure (with lots of copies of $\vec p$ translating $\ell_1$).

Note that it would be incorrect to write ``$\ell_2=\ell_1+\vec p$''. Because $\ell_1$
is a set and $\vec p$ is not a set, ``$\ell_1+\vec p$'' does not make mathematical sense.

As we pursue our study of linear algebra, we will come to see that lines and planes through
the origin are especially important (and will have special notation). 
Set addition will allow us to describe all lines and planes as translations of lines
and planes through the origin.


\section{Span}

Let $\vec u=\mat{1\\1}$ and $\vec v=\mat{-1\\2}$. Can the vectors $\vec w=\mat{2\\5}$ be obtained
as a linear combination of $\vec u$ and $\vec v$?

By drawing a picture, the answer appears to be \emph{yes}.

XXX Figure

Algebraically, we can use the definition of \emph{linear combination} to set up a system of equations.
We know $\vec w$ can be expressed as a linear combination of $\vec u$ and $\vec v$ if and only if 
the vector equation
\[
	\vec w = \mat{2\\5}=\alpha\mat{1\\1}+\beta\mat{-1\\2}=\alpha \vec u+\beta \vec v
\]
has a solution. By inspection, we see $\alpha=3$ and $\beta=1$ solve this equation.

After initial success, we might be tempted to ask the following:
\emph{what are all the locations in $\R^2$ that can be obtained
as a linear combination of $\vec u$ and $\vec v$?} Geometrically, it appears
any location can be reached. To verify this algebraically, consider the vector equation
\begin{equation}
	\label{EQSPAN1}
	\vec x=\mat{x\\y} = \alpha\mat{1\\1}+\beta\mat{-1\\2} = \alpha\vec u+\beta\vec v.
\end{equation}
Here $\vec x$ represents an arbitrary point in $\R^2$. Thus, if equation \eqref{EQSPAN1} always
has a solution\footnote{ The official terminology would be to say that
the equations is always \emph{consistent}.}, any vector in $\R^2$ can be obtained as a linear combination of $\alpha$ and $\beta$.

We can solve this equation for $\alpha$ and $\beta$ by considering the equations arising from the
first and second coordinates. Namely,
\begin{alignat*}{3}
	x &{}={}& \alpha &{}+{}& \beta\\
	y &{}={}& \alpha &{}-{}& 2\beta.
\end{alignat*}
Subtracting the second equation from the first, we get $x-y=3\beta$ and so $\beta=(x-y)/3$. Plugging 
$\beta$ into the first equation and solving, we get $\alpha=(2x+y)/3$. Thus, equation \eqref{EQSPAN1}
\emph{always} has a solution. Namely,
\begin{align*}
	\alpha &= \tfrac{1}{3}(2x+y)\\
	\beta &= \tfrac{1}{3}(x-y).
\end{align*}

There is a formal term for the set of vectors that can be obtained as linear combinations
of others: \emph{span}\index{Span}.

\begin{definition}[Span]
	Let $\mathcal X$ be a set of vectors. The \emph{span} of $\mathcal X$, written $\Span \mathcal X$,
	is the set of all linear combinations of vectors in $\mathcal X$. Formally,
	\[
	\Span \mathcal X = \Set{\vec x\given \vec x = \alpha_1\vec v_1+
	\cdots+\alpha_n\vec v_n\text{ for some }\vec v_1,\ldots,\vec v_n\in\mathcal X
	\text{ and scalars }\alpha_1,\ldots,\alpha_n}.
	\]
	Further, we define $\Span\emptyset =\Set{\vec 0}$.
\end{definition}

We just showed above that $\Span\Set*{\mat{1\\1},\mat{-1\\2}}=\R^2$.

\begin{example}
	Let $\vec u=\mat{-1\\2}$ and $\vec v=\mat{1\\-2}$. Find $\Span\Set{\vec u,\vec v}$.

	XXX Finish
\end{example}

The objects that arise from spans are familiar. If $\vec v\neq\vec 0$, then $\Span\Set{\vec v}$
is the line through the origin with direction vector $\vec v$. If $\vec v,\vec w\neq \vec 0$ and
aren't parallel, $\Span\Set{\vec v,\vec w}$ is a plane through the origin. In fact, vector form of
a line or a plane is nothing more than a \emph{translated span}.

\subsection{Representing Lines \& Planes as Translated Spans}

Consider the line $\ell$ given in vector form by
\[
	\vec x=t\vec d+\vec 0.
\]
The line $\ell$ passes through the origin, and if we unravel its definition, we see
\[
	\ell=\Set{\vec x\given \vec x=t\vec d+\vec 0\text{ for some }t\in \R}
	=\Set{\vec x\given \vec x=t\vec d\text{ for some }t\in \R}=\Span\Set{\vec d}.
\]

Similarly, if $\mathcal P$ is a plane given in vector form by
\[
	\vec x=t\vec d_1+s\vec d_2+\vec 0,
\]
then
\[
	\mathcal P=
	\Set{\vec x\given \vec x=t\vec d_1+s\vec d_2\text{ for some }t,s\in \R}=\Span\Set{\vec d_1,\vec d_2}.
\]

If the ``$\vec p$'' in our vector form is $\vec 0$, then that vector form actually defines
a span. This means, if you accept that every line/plane through the origin has a
vector form, then every line/plane through the origin can be written as a span. Conversely,
if $X=\Span\Set{\vec v_1,\ldots,\vec v_n}$ is a span, we know $\vec 0=0\vec v_1+\cdots+0\vec v_n\in X$,
and so every span contains the origin.

As it turns out, spans exactly describe points, lines, planes, and volumes\footnote{
 We use the word \emph{volume} to indicate the higher-dimensional analogue of a plane.} through the origin.

\begin{example}
	The line $\ell_1\subseteq\R^2$ is described by the equation $x+2y=0$ and the line 
	$\ell_2\subseteq \R^2$ is described by the equation $4x-2y=6$.
	If possible, describe $\ell_1$ and $\ell_2$ using spans.

	XXX Finish
\end{example}

However, not all points, lines, planes, and volumes pass through the origin
and so we can't describe every such object directly as a span. But, if we \emph{translate} a span using set addition we
can describe objects which don't pass through the origin.

\begin{example}
	Recall $\ell_2\subseteq\R^2$ is the line described by the equation $4x-2y=6$.
	Describe $\ell_2$ as a translated span.

	XXX Finish
\end{example}

We can now see translated spans provide an alternative notation to vector form 
for specifying lines and planes. If $X$ is described in vector form by
\[
	\vec x=t\vec d_1+s\vec d_2+\vec p,
\]
then
\[
	X=\Span\Set{\vec d_1,\vec d_2}+\Set{\vec p}.
\]


\section{Linear Independence}

Let
\[
	\vec u=\mat{1\\0\\0}\qquad\vec v=\mat{0\\1\\0}\qquad \vec w=\mat{1\\1\\0}.
\]
Since $\vec w=\vec u+\vec v$, we know that $\vec w\in\Span\Set{\vec u,\vec v}$. Geometrically,
this is also clear because $\Span\Set{\vec u,\vec v}$ is the $xy$-plane in $\R^3$ and 
$\vec w$ lies on that plane.

What about $\Span\Set{\vec u,\vec v,\vec w}$? Intuitively, since $\vec w$ is already
a linear combination of $\vec u$ and $\vec v$, we can't get anywhere \emph{new} by
taking linear combinations of $\vec u$, $\vec v$, and $\vec w$ compared to linear combinations
of just $\vec u$ and $\vec v$. So $\Span\Set{\vec u,\vec v}=\Span\Set{\vec u,\vec v,\vec w}$.

Can we prove this from the definitions? Yes! Suppose $\vec r\in \Span\Set{\vec u,\vec v,\vec w}$.
By definition,
\[
	\vec r=\alpha\vec u+\beta\vec v+\gamma\vec w
\] for some $\alpha,\beta,\gamma\in\R$. Since $\vec w=\vec u+\vec v$, we see
\[
	\vec r=\alpha\vec u+\beta\vec v+\gamma(\vec u+\vec v) 
	= (\alpha+\gamma)\vec u+(\beta+\gamma)\vec v\in\Span\Set{\vec u,\vec v}.
\]
Thus, $\Span\Set{\vec u,\vec v,\vec w}\subseteq \Span\Set{\vec u,\vec v}$. Conversely, if $\vec s\in\Span\Set{\vec u,\vec v}$,
by definition,
\[
	\vec s=a\vec u+b\vec v=a\vec u+b\vec v+0\vec w
\]
for some $a,b\in \R$,
and so $\vec s\in\Span\Set{\vec u,\vec v,\vec w}$. Thus $\Span\Set{\vec u,\vec v}\subseteq\Span\Set{\vec u,\vec v,\vec w}$.
We conclude $\Span\Set{\vec u,\vec v}=\Span\Set{\vec u,\vec v,\vec w}$.

In this case, $\vec w$ was a redundant vector---it wasn't needed for the span.


\begin{definition}[Linear Independence \& Dependence (Geometric)]
	\label{DEFLININD}
	The vectors $\vec v_1,\ldots,\vec v_n$ are called \emph{linearly dependent}\index{linearly independent} if 
	for at least one $i$, 
	\[
		\vec v_i \in \Span\Set{\vec v_1,\ldots,\vec v_{i-1},\vec v_{i+1},\ldots,\vec v_n}.
	\]
	If there is no such $i$, the vectors $\vec v_1,\ldots,\vec v_n$ are called \emph{linearly
	independent}\index{linearly dependent}.
\end{definition}
We will also refer to sets of vectors (for example $\Set{\vec v_1,\ldots,\vec v_n}$) as being linearly
independent or linearly dependent. For technical reasons, we didn't state the definition in terms
of sets\footnote{ The issue is, every element of a set is unique. Clearly, the vectors $\vec v$ and $\vec v$
are linearly dependent, but $\Set{\vec v,\vec v}=\Set{\vec v}$, and so $\Set{\vec v,\vec v}$ is technically
a linearly independent set. This issue would be resolved by talking about \emph{multisets} instead of sets,
but it isn't worth the hassle.}.

Definition \ref{DEFLININD} says that the vectors $\vec v_1,\ldots,\vec v_n$ are linearly dependent
if you can remove at least one vector without changing the span. In other words, $\vec v_1,\ldots,\vec v_n$ 
are linearly dependent \emph{if there is a redundant vector}.

\begin{example}
	\label{EXLINDEP}
	Let $\vec u=\mat{1\\2}$, $\vec v=\mat{2\\3}$, and $\vec w=\mat{4\\5}$. Determine whether
	$\Set{\vec u,\vec v,\vec w}$ is linearly independent or linearly dependent.

	XXX Finish
\end{example}

\begin{example}
	Determine whether the planes ... (given in vector form) are the same.

	XXX Finish
\end{example}

Suppose the vectors $\vec u$, $\vec v$, and $\vec w$ satisfy 
\begin{equation}
	\label{EQLIND}
	\vec w=\vec u+\vec v.
\end{equation}
The set
$\Set{\vec u,\vec v,\vec w}$ is linearly dependent since $\vec w\in\Span\Set{\vec u,\vec v}$.
But we can think of this another way. In particular, equation \eqref{EQLIND} can be rearranged
to get
\[
	\vec 0=\vec u+\vec v-\vec w.
\]
Here we have expressed $\vec 0$ as a non-trivial linear combination of $\vec u$, $\vec v$, and $\vec w$.
That is, we have written $\vec 0$ as a linear combination without all zero coefficients.

\begin{definition}[Trivial Linear Combination]
	A linear combination $\alpha_1\vec v_1+\cdots+\alpha_n\vec v_n$ is called
	\emph{trivial}\index{trivial linear combination}
	if $\alpha_1=\cdots=\alpha_n=0$. If at least one $\alpha_i\neq 0$,
	the linear combination is called \emph{non-trivial}.
\end{definition}

We can always write $\vec 0$ as a linear combination of vectors if we let all the coefficients
be zero, but it turns out we can only write $\vec 0$ as a \emph{non-trivial} linear combination
of vectors if those vectors are linearly dependent. This is the basis for another definition
of linear independence and dependence.

\begin{definition}[Linear Independence \& Linear Dependence (Algebraic)]
	\label{DEFLININDII}
	The vectors $\vec v_1,\ldots,\vec v_n$ are called \emph{linearly independent}\index{linearly independent}
	if for the only linear combination satisfying
	\[
		\vec 0=\alpha_1\vec v_1+\cdots+\alpha_n\vec v_n
	\]
	is the trivial linear combination (where $\alpha_1=\cdots=\alpha_n=0$).
\end{definition}

The idea of a ``redundant vector'' coming from the geometric definition of linear dependence
is geometrically intuitive, but it can be hard to work with. After all, checking independence
with this definition involves verifying for every vector that it is not in the span of the others.
The algebraic definition on the other hand is less obvious, but checking independence or dependence
of a set involves reasoning about solutions to just one equation.

\begin{theorem}
	The geometric and algebraic definitions of linear independence are equivalent.
\end{theorem}
\begin{proof}
	To show the two definitions are equivalent, we need to show that geometric~$\implies$~algebraic
	and algebraic~$\implies$~geometric.

	\medskip
	\noindent
	(geometric~$\implies$~algebraic) Suppose $\vec v_1,\ldots,\vec v_n$ are linearly dependent by the 
	geometric definition. That means that for some $i$, we have
	\[
		\vec v_i \in \Span\Set{\vec v_1,\ldots,\vec v_{i-1},\vec v_{i+1},\ldots,\vec v_n}.
	\]
	Fix such an $i$. Then, by the definition of span we know
	\[
		\vec v_i=\alpha_1\vec v_1+\cdots \alpha_{i-1}\vec v_{i-1}+\alpha_{i+1}\vec v_{i+1}+\cdots +\alpha_n\vec v_n,
	\]
	and so
	\[
		\vec 0=\alpha_1\vec v_1+\cdots \alpha_{i-1}\vec v_{i-1}-\vec v_i+\alpha_{i+1}\vec v_{i+1}+\cdots +\alpha_n\vec v_n.
	\]
	This must be a non-trivial linear combination because the coefficient of $\vec v_i$ is $-1\neq 0$. Therefore, 
	$\vec v_1,\ldots,\vec v_n$ is linearly dependent by the algebraic definition.
	
	\medskip
	\noindent
	(geometric~$\implies$~algebraic) Suppose $\vec v_1,\ldots,\vec v_n$ are linearly dependent by the 
	algebraic definition. That means there exist $\alpha_1,\ldots,\alpha_n$, not all zero, so that
	\[
		\vec 0=\alpha_1\vec v_1+\cdots +\alpha_n\vec v_n.
	\]
	Fix $i$ so that $\alpha_i\neq 0$ (why do we know there is such an $i$?). Rearranging we get
	\[
		-\alpha_i\vec v_i=\alpha_1\vec v_1+\cdots \alpha_{i-1}\vec v_{i-1}+\alpha_{i+1}\vec v_{i+1}+\cdots +\alpha_n\vec v_n,
	\]
	and since $\alpha_i\neq 0$, we can multiply both sides by $\frac{-1}{\alpha_i}$ to get
	\[
		\vec v_i=\tfrac{-\alpha_1}{\alpha_i}\vec v_1+\cdots +\tfrac{-\alpha_{i-1}}{\alpha_i}\vec v_{i-1}
	+\tfrac{-\alpha_{i+1}}{\alpha_i}\vec v_{i+1}+\cdots +\tfrac{-\alpha_n}{\alpha_i}\vec v_n.
	\]
	This shows that
	\[
		\vec v_i \in \Span\Set{\vec v_1,\ldots,\vec v_{i-1},\vec v_{i+1},\ldots,\vec v_n},
	\]
	and so $\vec v_1,\ldots,\vec v_n$ is linearly dependent by the geometric definition.
\end{proof}

\begin{example}
	Let $\vec u=\mat{1\\2}$, $\vec v=\mat{2\\3}$, and $\vec w=\mat{4\\5}$. Use the algebraic definition
	of linear independence to determine whether
	$\Set{\vec u,\vec v,\vec w}$ is linearly independent or dependent.

	Notice that $\vec u$, $\vec v$, and $\vec w$ are the vectors from Example \ref{EXLINDEP},
	so we already know that they are linearly dependent by the geometric definition of linear dependence.

	XXX Finish
\end{example}

\subsection{Linear Independence and Vector Form}

Fix a vector $\vec d$ and consider the question: When is
\[
	\vec x=t\vec d
\]
the vector form of a line? The answer is whenever $\vec d\neq \vec 0$, a simple enough rule. What about
for planes? Fix $\vec d_1$ and $\vec d_2$. When is
\[
	\vec x=t_1\vec d_1+t_2\vec d_2
\]
vector form of a plane? Here the rule is more complicated: $\vec d_1$ and $\vec d_2$ cannot be zero
nor can they be parallel. 

How about upping the dimension? Fix $\vec d_1$, $\vec d_2$, and $\vec d_3$. When is
\[
	\vec x=t_1\vec d_1+t_2\vec d_2+t_3\vec d_3
\]
vector form of a \emph{volume}? Coming up with such a rule seems hard, until we think
about spans.

Recall that if $\vec x=t\vec d$ is vector form of a line, then that line is $\Span\Set{\vec d}$,
and if $\vec x=t_1\vec d_1+t_2\vec d_2$ is vector form of a plane, that plane is $\Span\Set{\vec d_1,\vec d_2}$.
Similarly, if $\vec x=t_1\vec d_1+t_2\vec d_2+t_3\vec d_3$ is vector form of a volume,
that volume must be $\Span\Set{\vec d_1,\vec d_2,\vec d_3}$. So we can focus on a (maybe) simpler question:
When is $\Span\Set{\vec d_1,\vec d_2,\vec d_3}$ a volume?

This question is answered by considering whether or not $\Set{\vec d_1,\vec d_2,\vec d_3}$ is linearly
dependent. If $\Set{\vec d_1,\vec d_2,\vec d_3}$ is linearly dependent, then $\Span\Set{\vec d_1,\vec d_2,\vec d_3}$
is either a (i) plane, if there is one redundant vector, (ii) a line, if there are two redundant vectors, or (iii)
a point, if $\vec d_1=\vec d_2=\vec d_3=\vec 0$. If $\Set{\vec d_1,\vec d_2,\vec d_3}$ is linearly independent,
then $\Span\Set{\vec d_1,\vec d_2,\vec d_3}$ is a volume.

The moral of the story is that when you describe an object in vector form, the \emph{direction vectors must be linearly
independent}. 


\section{Dot Products}
Let $\vec a$ and $\vec b$ be vectors.  We assume they are placed so their
starting points coincide.  Let $\theta$ denote the \emph{smaller} of the
two angles between them, so $0\le \theta \le \pi$.
The \emph{dot product}\index{dot product} of $\vec a$ and $\vec b$ is defined to be
\[
	\vec a\cdot \vec b=\norm{\vec a}\norm{\vec b}\cos \theta.
\]
We will call this the \emph{geometric definition of the dot product}.
The dot product is also sometimes called the \emph{scalar product} because
the result is a scalar.
Note that $\vec a\cdot\vec b = 0$ when either $\vec a$ or $\vec b$ is zero or,
more interestingly, if their directions are perpendicular.

\begin{center}
	\newcommand{\tikzAngleOfLine}{\tikz@AngleOfLine}
	  \def\tikz@AngleOfLine(#1)(#2)#3{%
	  \pgfmathanglebetweenpoints{%
	    \pgfpointanchor{#1}{center}}{%
	    \pgfpointanchor{#2}{center}}
	  \pgfmathsetmacro{#3}{\pgfmathresult}%
	  }
	\newcommand{\tikzMarkAngle}[3]{
	\tikzAngleOfLine#1#2{\AngleStart}
	\tikzAngleOfLine#1#3{\AngleEnd}
	\draw #1+(\AngleStart:0.35cm) arc (\AngleStart:\AngleEnd:0.35cm);
	}
	\usetikzlibrary{patterns,decorations.pathreplacing}
	\begin{tikzpicture}
		\coordinate (A) at (2,1);
		\coordinate (B) at (.5,2);
		\coordinate (O) at (0,0);

		%\draw [mypink,fill] (A) circle (1.5pt) node [right] {initial point};
		%\draw [mypink,fill] (B) circle (1.5pt) node [left] {terminal point};
		\draw[->,thick,myred!60!white] (0,0) -- +(A) node [midway,below right] {$\vec a$};
		\draw[->,thick,mypink] (0,0) -- +(B) node [midway,above left] {$\vec b$};
		\tikzMarkAngle{(O)}{(A)}{(B)}
		\node at ($(O)+(50:.65)$) {$\theta$};
	\end{tikzpicture}
	\hspace{1cm}
	\begin{tikzpicture}
		\coordinate (A) at (-2,-1);
		\coordinate (B) at (.5,2);
		\coordinate (O) at (0,0);

		%\draw [mypink,fill] (A) circle (1.5pt) node [right] {initial point};
		%\draw [mypink,fill] (B) circle (1.5pt) node [left] {terminal point};
		\draw[->,thick,myred!60!white] (0,0) -- +(A) node [midway,below right] {$\vec a$};
		\draw[->,thick,mypink] (0,0) -- +(B) node [midway,above left] {$\vec b$};
		\tikzMarkAngle{(O)}{(B)}{(A)}
		\node at ($(O)+(140:.65)$) {$\theta$};
	\end{tikzpicture}
\end{center}

Algebraically, we can define the dot product in terms of coordinates:
\[
	\mat{a_1\\a_2\\\vdots\,\,\,\, \\a_n}\cdot \mat{b_1\\b_2\\\vdots\,\,\,\,\\b_n}
	=a_1b_1+a_2b_2+\cdots+a_nb_n.
\]
We will call this the \emph{algebraic definition of the dot product}\footnote{
	Philosophically,
every object should have only one definition from which equivalent characterizations
can be deduced as theorems.  If you're bothered, pick your favorite definition
to be the ``true'' definition and consider the other definition a theorem.
}.

By switching between algebraic and geometric definitions, we can use the dot
product to find quantities that are otherwise difficult to find.
\begin{example}
	Find the angle between the vectors $\vec v=(1,2,3)$ and $\vec w=(1,1,-2)$.

	From the algebraic definition of the dot product, we know
	\[
		\vec v\cdot \vec w = 1(1)+2(1)+3(-2) = -3.
	\]
	From the geometric definition, we know
	\[
		\vec v\cdot \vec w=\norm{\vec v}\norm{\vec w}\cos\theta
		=\sqrt{14}\sqrt{6}\cos\theta=2\sqrt{21}\cos\theta.
	\]
	Equating the two definitions of $\vec v\cdot \vec w$, we see
	\[
		\cos\theta = \frac{-3}{2\sqrt{21}}
	\]
	and so $\theta=\arccos\Big(\tfrac{-3}{2\sqrt{21}}\Big)$.
\end{example}

Recall that for vectors $\vec a$ and $\vec b$, the relationship $\vec a\cdot \vec b=0$
can hold for two reasons: (i) either $\vec a=\vec 0$, $\vec b=\vec 0$, or both
or (ii) $\vec a$ and $\vec b$ meet at $90^{\circ}$.  Thus, the dot product
can be used to tell if two vectors are perpendicular.  There is some strangeness
with the zero vector here, but it turns out this strangeness simplifies our lives
mathematically.

\begin{restatable}[Orthogonal]{definition}{DefOrthogonal}
	The vectors $\vec u$ and $\vec v$ are \emph{orthogonal}\index{orthogonal}
	if $\vec u\cdot\vec v=0$.
\end{restatable}

The definition of orthogonal encapsulates both the idea of two vectors forming
a right angle and the idea of one of them being $\vec 0$.

Before we continue, let's pin down the idea of one vector pointing
in the \emph{direction}\index{direction}\index{positive direction} of another.  
There are many ways we could define
this idea, but we'll go with this one.

\begin{definition}
	The vector $\vec u$ points in the \emph{direction} of
	the vector $\vec v$ if $k\vec u=\vec v$ for some scalar $k$.
	The vector $\vec u$ points in the \emph{positive direction} of
	$\vec v$ if $k\vec u=\vec v$ for some positive scalar $k$.
\end{definition}

The vector $2\xhat$ points in the direction of $\xhat$ since
$\frac{1}{2}(2\xhat)=\xhat$.  Since $\frac{1}{2}>0$, $2\xhat$ also points
in the positive direction of $\xhat$. In contrast, 
$-\xhat$ points in the direction $\xhat$ but not the positive direction of $\xhat$.

This idea can be rephrased using using linear combinations.
\begin{definition}
	The vector $\vec u$ points in the \emph{direction} of the vector $\vec v$
	if $\vec u\in\Span\Set{\vec v}$.
\end{definition}

An important use of the dot product is to determine two what extent two
vectors point in the same direction.

\begin{theorem}
	Let $\vec u$ and $\vec v$ be non-zero vectors. Then, $\vec u$ is in the direction
	of $\vec v$ if and only if 
	\[
		\frac{\vec u\cdot \vec v}{\Norm{\vec u}\Norm{\vec v}} = \pm 1.
	\]
\end{theorem}
\begin{proof}
	Let $\vec u$ and $\vec v$ be non-zero vectors.
	By the geometric definition of the dot product,
	\[
		\frac{\vec u\cdot \vec v}{\Norm{\vec u}\Norm{\vec v}} = \frac{\Norm{\vec u}\Norm{\vec v}\cos\theta}{\Norm{\vec u}\Norm{\vec v}}
		=\cos\theta,
	\]
	where $\theta$ is the angle between $\vec u$ and $\vec v$.

	The vector $\vec u$ is in the direction of $\vec v$ if and only if $\theta=0$ or $\pi$.
	But, for $\theta\in[0,\pi]$, we know $\cos\theta=\pm1$ if and only if $\theta=0$ or $\pi$.
\end{proof}

\begin{example}
	Let $\vec a=\mat{1\\2}$, $\vec b=\mat{3\\3}$, $\vec c=\mat{2\\1}$, and $\vec v=\mat{3\\4}$. Which vector out of
	$\vec a$, $\vec b$, and $\vec c$ has a direction closest to the direction of $\vec v$?

	XXX Finish
\end{example}


\subsection{Normal Form of Lines and Planes}


Since we will commonly be working in $\R^3$ there is another way to define a plane.  Given
any vector $\vec n\in\R^3$, we can consider the set $\mathcal Q\subseteq\R^3$ of vectors orthogonal to $\vec n$.
If $\vec n=\vec 0$, then $\mathcal Q=\R^3$.  Otherwise, $\mathcal Q$ is a plane through the origin.
In this case, $\vec n$ is called the \emph{normal vector}\index{normal vector} of the plane $\mathcal Q$.


\begin{center}
\begin{tikzpicture}
	\newcommand{\RightAngle}[4][5pt]{%
        \draw ($#3!#1!#2$)
        --($ #3!2!($($#3!#1!#2$)!.5!($#3!#1!#4$)$) $)
        --($#3!#1!#4$) ;
        }
    \begin{axis}[grid=major,view={20}{40},z buffer=sort,
	    width=12cm,
	    scale mode=scale uniformly,
	    zmin=-5,zmax=5,xmin=-10,xmax=10,ymin=-10,ymax=10,
	    xticklabels={,,}, yticklabels={,,}, zticklabels={,,},
	    xtick={-10,-5,...,10}, ytick={-10,-5,...,10}
	    ]
		\addplot3 [data cs=cart,surf,domain=-10:10,samples=2, opacity=0.5]
		{.25*x+.25*y};
		\coordinate (A) at (axis cs:-4,-4,-2);
		\coordinate (N) at (axis cs:-6,-6,6);
		\coordinate (B) at (axis cs:4,4,2);
		\coordinate (C) at (axis cs:-4,4,0);

		\draw [mypink,fill] (A) circle (1.5pt) node [below right] {$A$};
		\draw [->, thick] (A) -- (B) node [midway,below right] {$\vec d_1$};
		\draw [->, thick] (A) -- (C) node [midway,above left] {$\vec d_2$};
		\draw [->, thick] (A) -- (N) node [midway,left] {$\vec n$};
		\RightAngle{(N)}{(A)}{(C)}
		%\RightAngle{(N)}{(A)}{(B)}
    \end{axis}
  \end{tikzpicture}
\end{center}

\begin{definition}[Normal form of a plane]
	The plane $\mathcal P$ is described in \emph{normal form} if for some $\vec n$ and $\vec p$,
	the equation
	\[
		\vec n\cdot(\vec x-\vec p)=0
	\]
	if and only if $\vec x\in\mathcal P$.  Equivalently, $\mathcal P$ is described in normal form if
	for some $\vec n$ and scalar $\alpha\in\R$ the equation
	\[
		\vec n\cdot \vec x=\alpha
	\]
	is satisfied if and only if $\vec x\in \mathcal P$.  In either case, the vector $\vec n$
	is call a \emph{normal vector} for $\mathcal P$.
\end{definition}

Normal form of a plane only exists in $\R^3$, but it is often useful\footnote{ Just like $y=mx+b$ form
of a line only exists in $\R^2$.}.  The equivalence of the two ways to write a normal form of a plane
is straight forward.
\[
		\vec n\cdot(\vec x-\vec p)=0
\]
if and only if
\[
		\vec n\cdot\vec x = \vec n\cdot \vec p = \alpha.
\]
Since $\vec n$ and $\vec p$ are fixed, $\alpha$ is a constant. Expanding normal form in terms of
coordinates we see
\[
		\vec n\cdot(\vec x-\vec p)=\vec n\cdot\vec x-\alpha=
		n_xx+n_yy+n_zz-\alpha=0
\]
and so
\begin{equation}
	\label{EQScalarForm}
		n_xx+n_yy+n_zz=\alpha
\end{equation}
is another way to write a plane.  Equation \eqref{EQScalarForm} is sometimes
called \emph{scalar form}\index{scalar form of a plane}
of a plane.  For us, it will not be important to distinguish between scalar and normal form.

It should be noted that like vector form of a plane, normal form of a plane is not unique.
For example, the plane described by $\vec n\cdot(\vec x-\vec p)=0$ is the same as the
plane $(2\vec n)\cdot(\vec x-\vec p)=0$.

\begin{example}
	Find vector form and normal form of the plane $\mathcal P$ passing
	through the point $A=(1,0,0)$, $B=(0,1,0)$ and $C=(0,0,1)$.

	To find vector form of $\mathcal P$, we need a point on the plane and
	two direction vectors.  We have three points on the plane, so we can
	obtain two direction vectors by subtracting these points in different ways.
	Let
	\[
		\vec d_1=\overrightarrow{AB} = \mat{-1\\1\\0}\qquad\vec d_2=\overrightarrow{AC}=
		\mat{-1\\0\\1}.
	\]
	Using the point $A$, we may now write vector form of $\mathcal P$ as
	\[
		\mat{x\\y\\z} = t\mat{-1\\1\\0}+s\mat{-1\\0\\1}+\mat{1\\0\\0}.
	\]

	To write normal form we need to find a normal vector to $\mathcal P$.  By symmetry,
	we can see that $\vec n=(1,1,1)$ is a normal vector to $\mathcal P$.  If we weren't
	so insightful, we could also compute $\vec d_1\times \vec d_2 = (1,1,1)$ to find a
	normal vector.  Now, we may express $\mathcal P$ in normal form as
	\[
		\mat{1\\1\\1}\cdot\left(\mat{x\\y\\z}-\mat{1\\0\\0}\right)=0
	\]
	or equivalently,
	\[
		x+y+z=1.
	\]
\end{example}

\begin{example}
	Find the line $\mathcal P_1\cap \mathcal P_2$ where
	$\mathcal P_1$ is the plane given by the equation
	\[
		x+y+z=2
	\]
	and $\mathcal P_2$ is the plane given by the equation
	\[
		2x-y+z=0.
	\]

	Let $\ell=\mathcal P_1\cap \mathcal P_2$.  Since $\ell\subseteq \mathcal P_1$
	and $\ell\subseteq\mathcal P_2$, every direction vector for $\ell$ is also
	a direction vector for $\mathcal P_1$ and $\mathcal P_2$.

	Let $\vec n_1=(1,1,1)$
	be a normal vector for $\mathcal P_2$ and $\vec n_2=(2,-1,1)$ be a normal vector
	for $\mathcal P_2$.  If $\vec d$ is a direction vector for $\ell$, then
	$\vec n_1\cdot \vec d=0$ and $\vec n_2\cdot \vec d=0$.  Thus,
	\[
		\vec d=\vec n_1\times\vec n_2=\mat{2\\1\\-3}
	\]
	is a direction vector for $\ell$.  By guess and check we find that $\vec p=(0,1,1)$
	satisfies $\vec p\in\mathcal P_1$ and $\vec p\in\mathcal P_2$ and so $\vec p\in\ell$.
	Thus, we may write $\ell$ in vector form as
	\[
		\mat{x\\y\\z} = t\mat{2\\1\\-3}+\mat{0\\1\\1}.
	\]
\end{example}

%  \begin{tikzpicture}
%    \begin{axis}[grid=major,view={20}{40},z buffer=sort]
%      %\addplot3 [surf, domain=0:360, domain y=5:10,samples=30, samples y=10]
%      %{-y+5};
%\addplot3[domain=4:30,samples=80,samples y=0,mark=none,black, opacity=0.5,thick]
%	    ({x},{118.89/x},{2*x});
%      \addplot3 [data cs=cart,surf,domain=-10:10,samples=2, opacity=0.5]
%      {0};
%      %\addplot3 [domain=-10:10,samples=10, black,mark=none]
%	%\addplot3[domain=0:30,samples=80,samples y=0,mark=none,black, opacity=0.5,thick]
%	 %   ({x},{y},{x+y});
%      \addplot3 [data cs=cart,surf,domain=-10:10,samples=2, opacity=0.5]
%      {x+y};
%      \addplot3 [data cs=cart,surf,domain=-10:10,samples=2, opacity=0.5]
%      {-2*x+y+3};
%      %\addplot3 [domain=0:360, samples y=0, samples=30, thick, z buffer=auto]
%      %(x,5.1,0);
%      %\addplot3 [surf,domain=0:360, domain y=0:5,samples=30, samples y=10]
%      %{-y+5};
%    \end{axis}
%  \end{tikzpicture}



\section{Projection}
Another common vector operation is \emph{projection}\index{projection}.
Projection measures how much a vector points in the direction
of another.  This quantity is encoded as a vector.  We make this
definition mathematically precise as follows.

\begin{restatable}[Projection]{definition}{DefProjection}
	For a vector $\vec u$ and a non-zero vector $\vec v$,
	the \emph{projection} of $\vec u$ onto $\vec v$ is written
	as $\Proj_{\vec v}\vec u$ and is a vector in the direction
	of $\vec v$ with the property that $\vec u-\Proj_{\vec v}\vec u$
	is orthogonal to $\vec v$.

	The vector $\vec u-\Proj_{\vec v}\vec u$ is called the
	\emph{perpendicular component} of the projection of $\vec u$
	onto $\vec v$ and is notated $\Perp_{\vec v}\vec u$.
\end{restatable}

We can visualize
projections with the following diagram.

	\begin{center}
	\begin{tikzpicture}[>=latex,scale=2.5]
		\draw[->,thick,black] (0,0) -- (2,1) node [above] {$\vec u$};
		\draw[->,thick,black] (0,0) -- (3,0) node [above] {$\vec v$};
		\draw[->,thick,black,yshift=-.07cm] (0,0) -- (2,0);
		\draw[decoration={brace, mirror}, decorate, yshift=-.15cm] (0,0) -- (2,0) node [midway,below,yshift=-4pt] {$\Proj_{\vec v}\vec u$};

		\draw[dashed,thick,black] (2,0) -- (2,1);
		\draw[decoration={brace, mirror}, decorate, xshift=1.15cm] (2,0) -- (2,1)
			node [midway,right,xshift=4pt] {$\Perp_{\vec v}\vec u$};
		\draw[thin,black] (1.85,0)--(1.85,.15)--(2,.15);

	\end{tikzpicture}
	\end{center}

From the picture, it appears that $\vec u$, $\Proj_{\vec v}\vec u$,
and $\Perp_{\vec v}\vec u$ form a right triangle.  Of course, we shouldn't
trust the picture. We should verify this mathematically.

\begin{theorem}
	If $\vec u$ and $\vec v$ are non-zero vectors, then $\vec v$, $\Proj_{\vec v}\vec u$,
	and $\Perp_{\vec v}\vec u$ form a (possibly degenerate) right triangle.
\end{theorem}
\begin{proof}
	We need to verify that the sides $\Proj_{\vec v}\vec u$ and $\Perp_{\vec v}\vec u$
	meet at a right angle and that the hypotenuse $\vec u$ meets the sides.  That is,
	$\Perp_{\vec v}\vec u+\Proj_{\vec v}\vec u=\vec u$.

	By the definition of projection, $\Perp_{\vec v}\vec u=\vec u-\Proj_{\vec v}\vec u$
	is orthogonal to $\vec v$.  Since $\Proj_{\vec v}\vec u$ points in the
	direction of $\vec v$, we have $\Proj_{\vec v}\vec u=k\vec v$ and so
	$\Perp_{\vec v}\vec u$ is orthogonal to $\Proj_{\vec v}\vec u$.

	Finally, consider
	\[
		\Perp_{\vec v}\vec u+\Proj_{\vec v}\vec u=
		(\vec u-\Proj_{\vec v}\vec u)+\Proj_{\vec v}\vec u=\vec u,
	\]
	so indeed the vectors form a right triangle.
\end{proof}

Now that we've proved $\vec u$, $\Proj_{\vec v}\vec u$,
and $\Perp_{\vec v}\vec u$ form a right triangle, we are free to use
trigonometry to compute projections.  If $\theta$ is the angle between $\vec u$
and $\vec v$ and $0\leq \theta\leq \pi/2$,
we know $\norm{\Proj_{\vec v}\vec u}=\norm{\vec u}\cos\theta$.  This means
\[
	\Proj_{\vec v}\vec u = k\vec v=\norm{\vec u}\cos\theta\frac{\vec v}{\norm{\vec v}}
\]
(Recall that $\vec v/\norm{\vec v}$ is a unit vector in the direction of $\vec v$).
But $\cos \theta$ appears in the formula for the dot product.  Solving for
$\cos\theta$ in the dot product formula, we see $\cos\theta=\frac{\vec u\cdot\vec v}{\norm{\vec u}\norm{\vec v}}$.
Thus,
\[
	\Proj_{\vec v}\vec u = \norm{\vec u}\cos\theta\vec v
	=\frac{\vec u\cdot \vec v}{\norm{\vec u}\norm{\vec v}}\left(\frac{\vec v}{\norm{\vec v}}\right)
	=\frac{\vec u\cdot \vec v}{\norm{\vec v}^2}\vec v.
\]
Upon close inspection, we see $\norm{\vec v}^2=\vec v\cdot \vec v$ (since $\cos 0=1$)
and so we finally arrive at the formula
\[
	\Proj_{\vec v}\vec u
	=\frac{\vec u\cdot \vec v}{\vec v\cdot \vec v}\vec v
\]
Incredibly, if we use the algebraic definition of the dot product, we can
compute a projection without computing cosine of anything!

\begin{exercises}
	\begin{problist}
		\prob   In each case determine if the given pair of vectors is orthogonal.
			\begin{enumerate}
				\item $\vec{a} = 3\xhat + 4\yhat$, $\vec{b} = -4\xhat + 3\yhat$
				\item $\vec{a} = (4, -1, 2)$, $\vec{b} = (3, 0, -6)$
				\item $\vec{a} = 3\xhat - 2\yhat$, $\vec{b} = -2\xhat - 4\zhat$
			\end{enumerate}
            \begin{solution}
                \begin{enumerate}
                    \item   $\vec{a} \cdot \vec{b} = -12 + 12 = 0$, so $\vec{a}$ and $\vec{b}$ are
                        orthogonal.
                    \item   $\vec{a} \cdot \vec{b} = 12 + 0 - 12 = 0$, so $\vec{a}$ and $\vec{b}$ are
                        orthogonal.
                    \item   $\vec{a} \cdot \vec{b} = -6 + 8 = 2$, so $\vec{a}$ and $\vec{b}$ are not
                        orthogonal.
                \end{enumerate}
            \end{solution}

		\prob   Assume $\vec{u} = (a, b)$ is a non-zero plane vector.  Show that $\vec{v} = (-b,
			a)$ is orthogonal to $\vec{u}$.  By examining all possible signs for $a$ and
			$b$, convince yourself that the $90$ degree angle between $\vec{u}$ and
			$\vec{v}$ is in the counter-clockwise direction.
            \begin{solution}
                $\vec{u} \cdot \vec{v} = -ab + ba = 0$, so $\vec{u}$ and $\vec{v}$ are orthogonal.
            \end{solution}

		\prob   The methane molecule, $CH_4$, has four Hydrogen atoms at the vertices of a
			regular tetrahedron and a Carbon atom at its center.  Choose as the vertices of
			this tetrahedron the points $(0,0,0)$, $(1,1,0)$, $(1,0,1)$, and $(0,1,1)$.
			\begin{enumerate}
				\item Find the angle between two edges of the tetrahedron.
				\item Find the bond angle between two Carbon--Hydrogen bonds.
			\end{enumerate}
            \begin{solution}
                \begin{enumerate}
                    \item   $60^\circ$
                    \item   $\cos^{-1}\left(\frac{1}{3}\right)^\circ$
                \end{enumerate}
            \end{solution}

		\prob   An inclined plane makes an angle of $30$ degrees with the horizontal.  Use
			vectors and the dot product to find the projection of the gravitational
			acceleration vector $-g\yhat$ along a unit vector pointing along the inclined
			plane.
            \begin{solution}
                $\frac{-g\sqrt3}{4}\xhat - \frac{g}{4}\yhat$
            \end{solution}

		\prob   Derive the formula \[\norm{\vec{a} + \vec{b}}^2 = \norm{\vec{a}}^2 +
			\norm{\vec{b}}^2 + 2\norm{\vec{a}}\,\norm{\vec{b}}\cos\theta.\] How is it
			related to the Law of Cosines?  A picture might help.
            \begin{solution}
                \begin{align*}
                    \norm{\vec{a}+\vec{b}}^2 &= (\vec{a}+\vec{b})\cdot(\vec{a}+\vec{b}) \\
                                             &= \vec{a}\cdot\vec{a} + \vec{a}\cdot\vec{b} + \vec{b}
                    \cdot \vec{a} + \vec{b} \cdot \vec{b} \\
                    &= \norm{\vec{a}}^2 + \norm{\vec{b}}^2 + 2\vec{a}\cdot\vec{b} \\
                    &= \norm{\vec{a}}^2 + \norm{\vec{b}}^2 + 2\norm{\vec{a}}\,\norm{\vec{b}}\cos\theta
                \end{align*}
            \end{solution}

		\prob   Use the dot product to determine if the points $P(3,1,2)$, $Q(-1,0,2)$, and
			$R(11, 3, 2)$ are collinear.
            \begin{solution}
                $P$, $Q$ and $R$ are collinear if and only if the angle between $\overrightarrow{PQ}$
                and $\overrightarrow{QR}$ is $180^\circ$.  But $\overrightarrow{PQ}=(-4,-1,0)$ and
                $\overrightarrow{QR} = (12,3,0)$. So $\overrightarrow{PQ}\cdot \overrightarrow{QR} = -51
                = \norm{\overrightarrow{PQ}}\,\norm{\overrightarrow{QR}} \cos \theta = 51\cos \theta$,
                which implies that $\cos \theta=-1$ and so $\theta=180^\circ$ and the points are
                collinear.
            \end{solution}

        \prob[\openstax]	For the following exercises, the vectors $\vec{u}$ and $\vec{v}$ are given.
        Find the projection $\vec{w} = \Proj_{\vec{u}}\vec{v}$ of vector $\vec{v}$ onto vector
        $\vec{u}$. Express your answer in component form.
			\begin{enumerate}
				\item	$\vec{u} = 5\xhat + 2\yhat$, $\vec{v} = 2\xhat + 3\yhat$.
				\item	$\vec{u} = (−4, 7)$, $\vec{v} = (3, 5)$
				\item	$\vec{u} = 3\xhat + 2\zhat$, $\vec{v} = 2\yhat + 4\zhat$
				\item	$\vec{u} = (4, 4, 0)$, $\vec{v} = (0, 4, 1)$
			\end{enumerate}
            \begin{solution}
                Generally, the projection of $\vec{v}$ onto $\vec{u}$ is given by
                \[ 
                    \Proj_{\vec{u}}\vec{v} = \frac{\vec{u} \cdot \vec{v}}{\vec{u} \cdot \vec{u}}
                    \vec{u}.
                \]
                Thus, using this formula, we can compute the following.
                \begin{enumerate}
                    \item   $\vec{w} = [16/29] \, (5\xhat + 2\yhat)$.
                    \item   $\vec{w} = [23/65] \, (-4,7)$.
                    \item   $\vec{w} = [8/13] \, (3\xhat + 2\zhat)$.
                    \item   $\vec{w} = [1/2] \, (4,4,0)$.
                \end{enumerate}
            \end{solution}

        \prob[\openstax]	Consider the vectors $\vec{u} = 4\xhat - 3\yhat$ and $\vec{v} = 3\xhat +
        2\yhat$
			\begin{enumerate}
				\item	Find the components of vector $\vec{w} = \Proj_{\vec{u}}\vec{v}$
					that represents the projection of $\vec{v}$ onto $\vec{u}$.
				\item	Write the decomposition $\vec{v} = \vec{w} + \vec{q}$ of vector
					$\vec{v}$ into the orthogonal components $\vec{w}$ and
					$\vec{q}$, where $\vec{w}$ is the projection of $\vec{v}$ onto
					$\vec{u}$ and $\vec{q}$ is a vector orthogonal to the direction
					of $\vec{u}$.
			\end{enumerate}
            \begin{solution}
                \begin{enumerate}
                    \item   $\vec{w} = [\vec{u} \cdot \vec{v} / \vec{u} \cdot \vec{u}] \, \vec{u}$.
                        Therefore, we find that $\vec{w} = [6/25] \, (4,-3)$, which is to say that the
                        first component of $\vec{w}$ is $24/25$ and the second component of $\vec{w}$ is
                        $-18/25$.
                    \item   Begin by noting that $\vec{w} = \Proj_{\vec{u}}\vec{v}$. Since $\vec{w}$
                        points in the same direction as $\vec{u}$, then it follows that $\vec{q}$ must
                        be orthogonal to $\vec{w}$, which is to say that $\vec{q} = \Perp_{\vec{u}}
                        \vec{v} = \vec{v} - \Proj_{\vec{u}} \vec{v}$. Then clearly $\vec{v} = \vec{w} +
                        \vec{q} = \Proj_{\vec{u}}\vec{v} + \vec{v} - \Proj_{\vec{u}}\vec{v}$.
                        Computation of $\vec{w}$ and $\vec{q}$ gives that
                        \[
                            \vec{w} = \left( \frac{24}{25}, \frac{-18}{25} \right) \quad \vec{q} =
                            \left( \frac{51}{25}, \frac{68}{25} \right)  
                        \]
                \end{enumerate}
            \end{solution}

		\prob[\openstax]	Consider the vectors $\vec{u} = 2\xhat + 4\yhat$ and $\vec{v} = 4\xhat + 2\zhat$
			\begin{enumerate}
				\item	Find the components of vector $\vec{w} = \Proj_{\vec{u}}\vec{v}$
					that represents the projection of $\vec{v}$ onto $\vec{u}$.
				\item	Write the decomposition $\vec{v} = \vec{w} + \vec{q}$ of vector
					$\vec{v}$ into the orthogonal components $\vec{w}$ and
					$\vec{q}$, where $\vec{w}$ is the projection of $\vec{v}$ onto
					$\vec{u}$ and $\vec{q}$ is a vector orthogonal to the direction
					of $\vec{u}$.
			\end{enumerate}
            \begin{solution}
                \begin{enumerate}
                    \item   $\vec{w} = [2/5] \, (2,4,0)$, which is to say that the first component of
                        $\vec{w}$ is $4/5$, the second component of $\vec{w}$ is $8/5$, and the last
                        component of $\vec{w}$ is $0$.
                    \item   Computing $\vec{w} = \Proj_{\vec{u}}\vec{v}$ and $\vec{q} = \vec{v} -
                        \Proj_{\vec{u}}\vec{v}$ gives
                        \[
                            \vec{w} = \left( \frac{4}{5}, \frac{8}{5}, 0 \right) \quad
                            \vec{q} = \left( \frac{16}{5}, -\frac{8}{5}, 2 \right)
                        \]
                \end{enumerate}
            \end{solution}
	\end{problist}
\end{exercises}

\section{The Cross Product}

For vectors $\vec a$ and $\vec b$, the dot product $\vec a\cdot \vec b$ measures
how close $\vec a$ and $\vec b$ are to being orthogonal.  In contrast,
the \emph{cross product}\index{cross product} of $\vec a$ and $\vec b$,
written $\vec a\times \vec b$, will
measure the \emph{area} of the parallelogram whose sides are given by $\vec a$ and
$\vec b$.

Let's explore this idea.  Since the cross product is a \emph{product}, we will
demand it follow reasonable distribution laws\footnote{ The technical term
for satisfying these laws is \emph{bilinearity}.}:
\begin{align*}
	\vec a\times (\vec b+\vec c) &= \vec a\times \vec b+\vec a\times\vec c\\
	(\vec a+\vec b)\times\vec c &= \vec a\times \vec c+\vec b\times \vec c\\
	(\alpha\vec a)\times \vec b &= \alpha(\vec a\times \vec b)\\
	\vec a\times(\alpha \vec b) &= \alpha(\vec a\times \vec b)
\end{align*}
for vectors $\vec a$, $\vec b$, $\vec c$ and scalars $\alpha$.

Now, suppose $\vec a\times \vec b$ indeed encapsulates the area of the parallelogram
with sides $\vec a$ and $\vec b$.  If we slide the tip of $\vec b$ parallel to the vector
$\vec a$, we should not change the area.  Thus the cross product of $\vec a$ and $\vec b$
should be the same as that of $\vec a$ and $\vec b+\alpha\vec a$.

\begin{center}
	\usetikzlibrary{patterns,decorations.pathreplacing}
	\begin{tikzpicture}
		\coordinate (A) at (2,0);
		\coordinate (B) at (1,2);
		\coordinate (O) at (0,0);

		%\draw [mypink,fill] (A) circle (1.5pt) node [right] {initial point};
		%\draw [mypink,fill] (B) circle (1.5pt) node [left] {terminal point};
		\draw[fill,gray!30!white] (O) -- +(B) -- +($(A)+(B)$) -- +(A) -- (O);
		\draw[->,thick,myred!60!white] (0,0) -- +(A) node [midway,below] {$\vec a$};
		\draw[->,thick,mypink] (0,0) -- +(B) node [midway,above left] {$\vec b$};
		\draw[black,dashed,<->] ($.6*(A)$) -- ($.6*(A)+(0,2)$) node [midway,right] {height};
	\end{tikzpicture}
	\hspace{1cm}
	\begin{tikzpicture}
		\coordinate (A) at (2,0);
		\coordinate (B) at (1,2);
		\coordinate (C) at ($(B)+2*(A)$);
		\coordinate (O) at (0,0);

		%\draw [mypink,fill] (A) circle (1.5pt) node [right] {initial point};
		%\draw [mypink,fill] (B) circle (1.5pt) node [left] {terminal point};
		\draw[fill,gray!30!white] (O) -- +(C) -- +($(A)+(C)$) -- +(A) -- (O);
		\draw[->,thick,myred!60!white] (0,0) -- +(A) node [midway,below] {$\vec a$};
		\draw[->,thick,gray,dashed] (0,0) -- +(B);% node [midway,above left] {$\vec b$};
		\draw[->,thick,gray,dashed] (B) -- +($2*(A)$);% node [midway,above] {$2\vec a$};
		\draw[->,thick,mypink] (0,0) -- +(C) node [midway,above left] {$\vec b+2\vec a$};
		\draw[black,dashed,<->] ($2.6*(A)$) -- ($2.6*(A)+(0,2)$) node [midway,right] {height};
	\end{tikzpicture}
\end{center}

Using this invariance along with our distributive rules,
we now see
\[
	\vec a\times \vec b = \vec a\times (\vec b+\alpha\vec a)=
	\vec a\times \vec b+\alpha(\vec a\times\vec a),
\]
and so $\vec a\times \vec a=0$.  We can apply this newly-found fact to
the vector $\vec a+\vec b$ to deduce
\begin{align*}
	0=(\vec a+\vec b)\times (\vec a+\vec b)&=
	\vec a\times \vec a+\vec a\times\vec b+\vec b\times \vec a+\vec b\times \vec b\\
	&=0+\vec a\times \vec b+\vec b\times \vec a+0\\
	&=\vec a\times \vec b+\vec b\times \vec a,
\end{align*}
and so
\[
	\vec a\times \vec b=-\vec b\times \vec a.
\]
Products with this property are called \emph{anti-commutative}.  Now
for an incredible fact of the universe: the result of the cross product of
two vectors in $\R^3$ can be represented by another vector in $\R^3$ whose magnitude
corresponds to the area of the parallelogram
with sides $\vec a$ and $\vec b$.\footnote{ This is \emph{only} true in $\R^3$.  In $\R^4$ a product that
produces area-like quantities does exist, but the output cannot be described by
a vector.  In higher dimensions, the cross product is called the \emph{wedge product}.}
Using trigonometry, we deduce
\[
	\norm{\vec a\times \vec b}=\norm{\vec a}\norm{\vec b}\sin \theta
\]
where $0\leq \theta\leq \pi$ is smaller of the two angles between $\vec a$ and $\vec b$.
What remains to be seen is what direction $\vec a\times \vec b$ points in.  For this,
we use the standard basis for $\R^3$ as a launching point. Recall
$\xhat$, $\yhat$, and $\zhat$ are all unit vectors and all orthogonal to each other.
Thus, crossing any two of them must result in a unit vector.  By convention,
\begin{align*}
	\xhat\times\yhat &=  \zhat, \\
	\yhat\times\zhat &=  \xhat,\\
	\zhat\times\xhat &=  \yhat.
\end{align*}
Let $\vec a=a_x\xhat +a_y\yhat +a_z\zhat$ and $\vec b=b_x\xhat +b_y\yhat +b_z\zhat$.
Using the distributive laws of the cross product we see,
\begin{align*}
	\vec a\times\vec b &= (a_x \xhat + a_y\yhat + a_z\zhat) \times (b_x\xhat + b_y\yhat + b_z\zhat) \\
	&= \phantom{+}a_xb_x\xhat\times\xhat + a_xb_y\xhat\times\yhat + a_xb_z\xhat\times\zhat \\
       &\phantom{=}+ a_yb_x\yhat\times\xhat + a_yb_y\yhat\times\yhat + a_yb_z\yhat\times\zhat \\
       &\phantom{=}+ a_zb_x\zhat\times\xhat + a_zb_y\zhat\times\yhat + a_zb_z\zhat\times\zhat \\
	&= \phantom{+}\vec 0  + a_xb_y\zhat - a_xb_z\yhat \\
       &\phantom{=}- a_yb_x\zhat + \vec 0 + a_yb_z \xhat \\
	&\phantom{=}+ a_zb_x\yhat - a_zb_y\xhat + {\vec 0}\\
\intertext{so}
\vec a\times \vec b &= (a_yb_z - a_zb_y)\xhat - (a_xb_z - a_zb_x)\yhat + (a_xb_y - a_yb_x)\zhat.
\end{align*}

\begin{exercise}
	Verify that $\norm{\vec a\times \vec b}=\norm{\vec a}\norm{\vec b}\sin\theta$. (Hint: you
	can use $\vec a\cdot \vec b=\norm{\vec a}\norm{\vec b}\cos\theta$ to solve for $\theta$
	and the proceed using components.)
\end{exercise}

Now that we know what the cross product is and how to compute it, let's explore
some of its incredible properties.  First,
\[
	(\vec a\times\vec b)\cdot \vec a=(a_yb_z - a_zb_y)a_x - (a_xb_z - a_zb_x)a_y + (a_xb_y - a_yb_x)a_z=0
\]
and
\[
	(\vec a\times\vec b)\cdot \vec b=(a_yb_z - a_zb_y)b_x - (a_xb_z - a_zb_x)b_y + (a_xb_y - a_yb_x)b_z=0.
\]
Thus, $\vec a\times\vec b$ is orthogonal to both $\vec a$ and $\vec b$.
Just based on this property, since the length of $\vec a\times \vec b$ is fixed, $\vec a\times\vec b$
can be one of two vectors in space.
If we investigate further, we'll
see that $\vec a\times\vec b$ is the vector that satisfies the \emph{right-hand rule}.

\begin{center}
\definecolor{ce5d4b1}{RGB}{229,212,177}
\definecolor{c897f6a}{RGB}{137,127,106}
\definecolor{c963c96}{RGB}{150,60,150}
\definecolor{c2828ff}{RGB}{40,40,255}
\definecolor{ce12828}{RGB}{225,40,40}

\begin{tikzpicture}[y=0.80pt, x=0.80pt, yscale=-1.000000, xscale=1.000000, inner sep=0pt, outer sep=0pt]
\path[draw=black,fill=ce5d4b1,line width=1.109pt] (197.7203,128.5199) ..
  controls (187.5579,127.1341) and (171.3905,117.8956) .. (169.0809,116.5099) ..
  controls (166.7713,115.1241) and (156.1470,103.5760) .. (150.6039,94.7994) ..
  controls (145.0608,86.0228) and (139.9796,75.3985) .. (138.1319,69.3935) ..
  controls (136.2842,63.3885) and (134.4365,57.3834) .. (133.9746,54.1500) ..
  controls (133.5127,50.9165) and (134.4365,43.2947) .. (135.8223,40.0612) ..
  controls (132.5888,37.7516) and (126.1219,37.2897) .. (123.3503,43.2947) ..
  controls (120.5788,49.2997) and (118.2691,58.5382) .. (118.2691,64.5433) ..
  controls (118.2691,70.5483) and (128.4315,89.9492) .. (121.9645,92.7207) ..
  controls (115.4976,95.4923) and (114.1118,99.1877) .. (80.8532,90.8730) ..
  controls (47.5946,82.5584) and (33.7368,81.6345) .. (30.5034,88.1015) ..
  controls (27.2699,94.5684) and (46.2088,97.3400) .. (56.3712,99.1877) ..
  controls (66.5335,101.0354) and (89.6395,109.6429) .. (86.3963,113.9693) ..
  controls (84.3869,116.6493) and (82.0658,116.0424) .. (76.8303,117.8901) ..
  controls (67.7715,121.0875) and (67.9114,122.4936) .. (56.6021,126.3641) ..
  controls (45.3159,130.2267) and (46.0504,141.2501) .. (49.5961,143.0996) ..
  controls (53.1419,144.9492) and (56.0705,146.0292) .. (71.0230,137.5505) ..
  controls (77.6516,135.0839) and (85.7039,133.4463) .. (92.7866,132.8306) ..
  controls (99.8693,132.2148) and (104.1813,158.6984) .. (111.8798,162.7019) ..
  controls (119.5782,166.7054) and (128.5086,166.3973) .. (134.0517,166.3973) ..
  controls (139.5948,166.3973) and (142.6740,167.3216) .. (145.7537,165.7816) ..
  controls (148.8333,164.2415) and (157.4561,166.3968) .. (162.0754,168.8607) ..
  controls (162.0754,168.8607) and (179.9366,177.4835) .. (181.4762,179.3312);
\path[draw=c897f6a,line cap=round,miter limit=4.00,line width=0.739pt]
  (90.3231,113.2759) .. controls (106.0281,112.3525) and (111.5712,127.4418) ..
  (115.2666,135.4488);
\path[draw=c897f6a,line cap=round,miter limit=4.00,line width=0.739pt]
  (123.4275,101.5735) .. controls (123.5817,113.2759) and (128.5091,123.1306) ..
  (135.1294,126.8251);
\path[draw=c897f6a,line cap=round,miter limit=4.00,line width=0.739pt]
  (92.7866,94.7989) .. controls (89.0912,95.4147) and (84.7800,104.0374) ..
  (86.6277,106.5009);
\path[draw=c897f6a,line cap=round,miter limit=4.00,line width=0.739pt]
  (73.0776,89.8716) .. controls (70.6142,91.7193) and (65.6868,96.0304) ..
  (66.9188,98.4943);
\path[draw=c897f6a,line cap=round,miter limit=4.00,line width=0.739pt]
  (50.9052,86.1762) .. controls (47.8256,88.0239) and (46.8246,93.3365) ..
  (46.2088,94.5679);
\path[draw=c897f6a,line cap=round,miter limit=4.00,line width=0.739pt]
  (84.7800,117.5875) .. controls (88.4754,119.4352) and (90.3231,125.5936) ..
  (89.0912,129.9052);
\path[draw=c897f6a,line cap=round,miter limit=4.00,line width=0.739pt]
  (74.9932,120.9466) .. controls (78.0729,122.7943) and (78.3413,132.0296) ..
  (77.1093,133.8773);
\path[draw=c897f6a,line cap=round,miter limit=4.00,line width=0.739pt]
  (63.2381,126.1188) .. controls (67.2329,129.1144) and (67.6685,133.2944) ..
  (65.5889,138.3441);
\path[draw=c897f6a,line cap=round,miter limit=4.00,line width=0.739pt]
  (117.8072,97.3400) .. controls (115.9595,101.9592) and (114.5737,105.6546) ..
  (115.0357,108.4262);
\path[draw=c897f6a,line cap=round,miter limit=4.00,line width=0.739pt]
  (118.2691,64.5433) .. controls (123.7878,65.3318) and (119.5057,66.0455) ..
  (129.8173,64.0813);
\path[draw=c897f6a,line cap=round,miter limit=4.00,line width=0.739pt]
  (166.3865,120.6672) .. controls (164.5388,126.2103) and (162.6911,128.0575) ..
  (159.6114,129.9052);
\path[draw=black,fill=ce5d4b1,line width=1.109pt] (120.5021,161.1628) ..
  controls (127.8490,159.6934) and (133.1283,158.6993) .. (137.1314,157.4674) ..
  controls (141.1344,156.2354) and (146.9852,152.8476) .. (148.5252,149.7685) ..
  controls (150.0653,146.6893) and (148.8333,146.0735) .. (147.9095,143.9177) ..
  controls (146.9856,141.7619) and (136.8238,142.9939) .. (134.3598,143.9177) ..
  controls (131.8959,144.8416) and (129.7406,145.7654) .. (119.5782,147.3050) ..
  controls (109.4159,148.8446) and (106.6443,149.6146) .. (101.4089,148.9984) ..
  controls (97.1167,148.4931) and (102.0255,159.3146) .. (104.1809,160.5466) ..
  controls (106.3362,161.7785) and (106.6443,163.9343) .. (120.5021,161.1628) --
  cycle;
\path[draw=black,fill=ce5d4b1,line width=1.109pt] (143.7845,133.4473) ..
  controls (146.1343,135.9107) and (145.1269,138.3746) .. (144.4557,138.9904) ..
  controls (143.7845,139.6061) and (136.0620,142.9934) .. (132.0331,144.2254) ..
  controls (128.0037,145.4573) and (127.3326,144.5335) .. (119.2752,146.3812) ..
  controls (111.2174,148.2289) and (110.5462,148.8446) .. (108.1955,149.7685) ..
  controls (105.8452,150.6923) and (101.1451,150.3842) .. (96.7808,146.9969) ..
  controls (92.4161,143.6096) and (93.0873,137.7584) .. (93.4231,135.9107) ..
  controls (93.7589,134.0630) and (94.0952,130.0595) .. (102.4884,129.4438) ..
  controls (122.7725,131.9811) and (136.5302,126.1649) .. (143.7845,133.4473) --
  cycle;
\path[draw=c897f6a,line cap=round,miter limit=4.00,line width=0.739pt]
  (95.5577,135.2950) .. controls (94.6694,143.2886) and (96.0699,143.6863) ..
  (100.4850,146.9969);
\path[draw=c897f6a,line cap=round,miter limit=4.00,line width=0.739pt]
  (109.1078,135.4483) .. controls (109.1078,137.9118) and (109.7235,143.4549) ..
  (109.7235,143.4549);
\path[draw=c897f6a,line cap=round,miter limit=4.00,line width=0.739pt]
  (124.3513,134.6783) .. controls (124.3513,135.9102) and (124.9675,140.8371) ..
  (124.9675,142.6848);
\path[draw=c897f6a,line cap=round,miter limit=4.00,line width=0.739pt]
  (127.4305,148.5370) .. controls (127.4305,150.3847) and (130.5102,156.5435) ..
  (130.5102,156.5435);
\path[draw=c897f6a,line cap=round,miter limit=4.00,line width=0.739pt]
  (112.4951,152.0781) .. controls (113.1108,153.3100) and (113.7270,158.8527) ..
  (115.5747,159.4689);
\path[draw=c897f6a,line cap=round,miter limit=4.00,line width=0.739pt]
  (169.4657,135.4488) .. controls (166.3865,140.9919) and (164.5388,155.1582) ..
  (161.4591,158.2374);
\path[draw=c963c96,line cap=round,line width=3.326pt] (131.4340,30.1298) --
  (131.1254,7.4955);
\path[cm={{0.46193,0.0,0.0,0.46193,(-17.99047,-11.80742)}},fill=c963c96]
  (322.8140,0.0000) -- (346.1270,60.3000) -- (322.8140,41.7880) --
  (299.5010,60.3000) -- cycle;
\path[draw=c2828ff,line cap=round,line width=3.326pt] (23.1190,85.4094) --
  (0.9023,81.0705);
\path[cm={{0.46193,0.0,0.0,0.46193,(-17.99047,-11.80742)}},fill=c2828ff]
  (0.0000,192.5000) -- (63.7990,182.0440) -- (40.9000,201.0670) --
  (54.2400,227.6800) -- cycle;
\path[draw=ce12828,line cap=round,line width=3.326pt] (40.3862,143.0553) --
  (26.1627,148.9449);
\path[cm={{0.46193,0.0,0.0,0.46193,(-17.99047,-11.80742)}},fill=ce12828]
  (69.0630,358.1970) -- (96.6050,315.5680) -- (95.5850,348.0050) --
  (118.0640,371.4130) -- cycle;
\begin{scope}[cm={{0.46193,0.0,0.0,0.46193,(-17.99047,-11.80742)}}]
\end{scope}
\path[fill=black,line join=miter,line cap=butt,line width=0.800pt]
  (-2.2839,61.9362) node[above right] (text4204) {$\vec a$};
\path[fill=black,line join=miter,line cap=butt,line width=0.800pt]
  (9.0458,132.6143) node[above right] (text4204-3) {$\vec b$};
\path[fill=black,line join=miter,line cap=butt,line width=0.800pt]
  (146.0527,6.4312) node[above right] (text4204-7) {$\vec a\times \vec b$};
\end{tikzpicture}\footnote{ Image credit: Acdx, from Wikipedia \url{https://en.wikipedia.org/wiki/Cross_product}}
\end{center}

A vector that encodes area, points orthogonally to others, and obeys the right-hand
rule is handy indeed, and the cross product will be a useful tool for solving
many problems.


\begin{exercises}
    \begin{problist}
        \prob   Find $\vec{a} \times \vec{b}$ for the following pairs:
        \begin{enumerate}
            \item   $\vec{a} = (4, -2, 0)$, $\vec{b} = (2, 1, -1)$
            \item   $\vec{a} = (3, 3, 0)$, $\vec{b} = (4, -3, 2)$
            \item   $\vec{a} = 2\xhat + 3\yhat + 4\zhat$, $\vec{b} = \xhat - 3\yhat + 4\zhat$.
        \end{enumerate}
        \begin{solution}
            \begin{enumerate}
                \item   $(2,4,8)$
                \item   $(15,6,-21)$
                \item   $24\xhat -4\yhat -9\zhat$
            \end{enumerate}
        \end{solution}

        \prob   Use the cross product to find the areas of the following figures.
        \begin{enumerate}
            \item   The parallelogram with vertices $(0,0,0)$, $(1, 1, 0)$, $(1,2,1)$ and $(0,1,1)$.
                (Perhaps you should first check that this is a parallelogram.)
            \item   The triangle with vertices $(1,0,0)$, $(0,1,0)$, and $(0,0,1)$.
        \end{enumerate}
        \begin{solution}
            \begin{enumerate}
                \item   $\sqrt3$
                \item   $\frac{\sqrt3}{2}$
            \end{enumerate}
        \end{solution}

        \prob   Prove that the cross product is not associative by calculating $\vec{a} \times (\vec{b}
        \times \vec{c})$ and $(\vec{a} \times \vec{b}) \times \vec{c}$ for $\vec{a} = \xhat$, $\vec{b} =
        \vec{c} = \yhat$.
        \begin{solution}
            $\xhat \times (\yhat \times \yhat)=\xhat \times \vec{0} = \vec{0}$ while $(\xhat \times
            \yhat) \times \yhat = \zhat \times \yhat = -\xhat$
        \end{solution}
        
        \prob   Verify the formula.
        \[ 
            \norm{\vec{a} \times \vec{b}}^2 = \norm{\vec{a}}^2 \norm{\vec{b}}^2 - (\vec{a} \cdot
            \vec{b})^2.  
        \] 
        Hint: Use the definitions in terms of the sine and cosine of the included angle $\theta$.
        \begin{solution}
            \begin{align*}
                \norm{\vec{a} \times \vec{b}}^2 &= (\norm{\vec{a}} \norm{\vec{b}}\sin\theta)^2 \\
                                                &= \norm{\vec{a}}^2 \norm{\vec{b}}^2\sin^2\theta \\
                                                &= \norm{\vec{a}}^2 \norm{\vec{b}}^2(1-\cos^2\theta) \\
                                                &= \norm{\vec{a}}^2 \norm{\vec{b}}^2 - \norm{\vec{a}}^2
                \norm{\vec{b}}^2\cos^2\theta \\
                &= \norm{\vec{a}}^2 \norm{\vec{b}}^2 - (\norm{\vec{a}} \norm{\vec{b}}\cos\theta)^2 \\
                &= \norm{\vec{a}}^2 \norm{\vec{b}}^2 - (\vec{a} \cdot\vec{b})^2.
            \end{align*}
        \end{solution}
    \end{problist}
\end{exercises}



	\clearpage
\chapter{Parameterization}
	\newpage
\section*{Exploration Questions}
\addcontentsline{toc}{section}{Exploration Questions}

\section{Span}



	\clearpage
\chapter{Multi-variable Functions}
	\newpage
\section*{Exploration Questions}
\addcontentsline{toc}{section}{Exploration Questions}

	Being able to picture multi-dimensional objects is 
	invaluable during mathematical exploration.  There are two main ways we visualize
	surfaces: perspective drawings and level-curves.

	\clearpage
\chapter{Integrals}
	\section{Multivariable Integrals}

Fundamentally, an integral is the result of chopping a region up into tiny
pieces, adding all the pieces up again, and taking a limit as the size of
the tiny pieces goes to zero.  Thus far, the domain of this procedure---the
thing we chop into tiny pieces---has been a line or curve.  We will now consider
integrating over multi-dimensional domains.

Consider the motivating example of finding the area of a region in the plane.
Let $\mathcal R\subseteq \R^2$ be the region below the line $y=1$ and above the
curve $y=x^2$.  We wish to find the area of $\mathcal R$.


\begin{center}
	\begin{tikzpicture}

		\begin{axis}[
		    anchor=origin,
		    set layers=standard,
		    disabledatascaling,
		    xmin=-2,xmax=2,
		    ymin=-1,ymax=2,
		    x=1.3cm,y=1.3cm,
		    grid=both,
		    grid style={line width=.1pt, draw=gray!10},
		    %major grid style={line width=.2pt,draw=gray!50},
		    axis lines=middle,
		    minor tick num=0,
		    enlargelimits={abs=0.5},
		    axis line style={latex-latex},
		    ticklabel style={font=\tiny,fill=white},
		    xlabel style={at={(ticklabel* cs:1)},anchor=north west},
		    ylabel style={at={(ticklabel* cs:1)},anchor=south west}
		]

	    	\addplot[name path=f,no marks,mypink,thick,domain=-1:1, samples=25, draw=none] ({x},{x*x});
	    	\addplot[no marks,mypink,thick,domain=-2:2, samples=25,smooth] ({x},{x*x});
	    	\addplot[name path=g, no marks,green!50!black,thick,domain=-3:3, samples=2] ({x},{1});
		\addplot [thick, color=blue, fill=blue, fill opacity=0.05] fill between [of=f and g, split];
		\draw (0,.3) node[above right] {$\mathcal R$};
		\draw [mypink,fill] (1.2,1)  node [above right] {$y=x^2$};
		\draw [green!50!black,fill] (-2,1)  node [below right] {$y=1$};

		\end{axis}
	\end{tikzpicture}
\end{center}

Using the usual calculus strategy, we will chop $\mathcal R$ up into little rectangles
of width $\Delta x$ and height $\Delta y$.  Then
\[
	\text{area of }\mathcal R \quad\approx \sum_{\text{tiny rectangles}}\text{area of tiny rectangle}
\]
and
\[
	\text{area of }\mathcal R =\lim_{\Delta x,\Delta y\to 0} \sum_{\text{tiny rectangles}}\text{area of tiny rectangle}.
\]

\begin{center}
	\begin{tikzpicture}[x=2cm,y=2cm,
]

\begin{scope}[	spy using outlines={circle, lens={scale=4}, size=3cm, connect spies}]
		\coordinate (A) at (.3,.15*4);
		\coordinate (B) at (.4,.15*5);
		\fill[gray,opacity=.5] (A) rectangle (B);


		\begin{axis}[
		    anchor=origin,
		    set layers=standard,
		    disabledatascaling,
		    xmin=-1,xmax=1,
		    ymin=0,ymax=1,
		    x=2cm,y=2cm,
		    grid=both,
		    grid style={line width=.1pt, draw=gray!10},
		    %major grid style={line width=.2pt,draw=gray!50},
		    axis lines=middle,
		    minor tick num=0,
		    enlargelimits={abs=0.4},
		    axis line style={latex-latex},
		    ticklabel style={font=\tiny,fill=white},
		    xlabel style={at={(ticklabel* cs:1)},anchor=north west},
		    ylabel style={at={(ticklabel* cs:1)},anchor=south west}
		]

	    	\addplot[name path=f,no marks,mypink,thick,domain=-1:1, samples=25, draw=none] ({x},{x*x});
	    	\addplot[no marks,mypink,thick,domain=-2:2, samples=25,smooth] ({x},{x*x});
	    	\addplot[name path=g, no marks,green!50!black,thick,domain=-3:3, samples=2] ({x},{1});
		\addplot [thick, color=blue, fill=blue, fill opacity=0.05] fill between [of=f and g, split];

		\end{axis}
			\foreach \x in {-1,-.9,...,1} {
				\draw ($(\x,1)$) -- ($(\x,\x*\x)$);
			}
			\foreach \y in {0,.15,...,1} {
				\draw ({-\y^.5,\y}) -- ({\y^.5,\y});
			}


		\spy [blue] on ($.5*(A)+.5*(B)$)
             in node (a) at (3,1);


\end{scope}
	\coordinate (L) at ($(a.center) - (.2,.3)$);
\coordinate (R) at ($(a.center) - (-.2,.3)$);
\coordinate (T) at ($(a.center) + (.2,.3)$);
		\draw[decoration={brace, mirror}, decorate, mypink] ($(R)+(.03,0)$) -- ($(T)+(.03,0)$) node[midway, right] {$\Delta y$};
\draw[decoration={brace, mirror}, decorate,mypink] ($(L)-(0,.03)$) -- ($(R)-(0,.03)$) node[midway, below] {$\Delta x$};

		
	\end{tikzpicture}
\end{center}


Using integral notation, we would write
\[
	\text{area of }\mathcal R=\int_{\mathcal R} \d A.
\]
Here $\d A$ represents a ``tiny area,'' the subscript $\mathcal R$ represents the region of integration,
and the integral sign means we're adding things up.  In this case, we're finding area, so $\d A=1\d A$ is
exactly what we're adding up.  In other situations, we'll be adding up more complicated functions.

This is all well and good, but how do we actually \emph{find} the area.  To do this, we'll
need to convert $\int_{\mathcal R}\d A$ into a more traditional-looking integral---one that we
know how to evaluate.

Let's write down our sum more carefully.  We need to sum over all tiny rectangles that fit inside
$\mathcal R$.  To do so, we can take a systematic approach: let's sum all the rectangles
in a column first and then sum up all the columns.  The lower left corner of all 
rectangles in a single column share a common $x$-coordinate.  Consider
the column with lower left corner at $(x_0,y_0)$.  Counting, we see there
are approximately $(1-x_0^2)/\Delta y$ rectangles in this column.  Further, there are approximately
$2/\Delta x$ columns.  Therefore,
\[
	\text{area of }\mathcal R\approx 
	\sum_{i=1}^{2/\Delta x}\quad\sum_{j=1}^{(1-(i\Delta x)^2)/\Delta y} \Delta y\Delta x.
\]
That sum is really hard to parse, so we'll write it another way.
\[
	\text{area of }\mathcal R\approx
	\sum_{x_0=-1,-1+\Delta x,-1+2\Delta x,\ldots,1}
	\quad
	\sum_{y_0=x_0^2,x_0^2+\Delta y,x_0^2+2\Delta y,1}
	\Delta y\Delta x.
\]
This is still hard to read, but it's looking more like an integral.  The inner sum
is adding up things from $y_0=x_0^2$ to $y_0=1$ and the outer sum is adding up
things from $x_0=-1$ to $x_0=1$.  Upon taking a limit, we may rewrite this as an
integral, giving
\begin{equation}
	\label{EQITERATEDINT}
	\text{area of }\mathcal R\quad
	=\quad
	\int_{\mathcal R}\d A \quad =\quad \int_{x=-1}^{x=1}\int_{y=x^2}^{y=1} \d y\d x.
\end{equation}
Now, we should take a moment to make sure we understand what we've just written.  The
right side of Equation \eqref{EQITERATEDINT} is an \emph{iterated integral}\index{iterated integral}.
That is,
\[
	\int_{x=-1}^{x=1}\int_{y=x^2}^{y=1} \d y\d x
	\quad 
	=
	\quad \int_{x=-1}^{x=1}\left(\int_{y=x^2}^{y=1} \d y\right)\d x
\]
and so the integral with respect to $y$ \emph{must} be done before the integral with respect to $x$.
To be clear, $\d y$ and $\d x$ are \emph{not} being multiplied.  However, $\Delta y$ and $\Delta x$ 
\emph{were} being multiplied in our sum expression.  What happened?  The answer is some slight of
hand.  We can write a more complete list of steps:\footnote{ Even still, we skipped steps.
In particular, we split up $\lim_{\Delta x,\Delta y\to 0}$ into two limits 
$\lim_{\Delta x\to 0}\lim_{\Delta y\to 0}$.  Then, we exchanged a limit and a sum.  These
steps require justification, but we won't trouble ourselves with that here.}
\[
	\lim_{\Delta x,\Delta y\to 0} \sum_{x_i}\sum_{y_i} \Delta y\Delta x
	=
	\lim_{\Delta x,\Delta y\to 0} \sum_{x_i}\left(\sum_{y_i} \Delta y\right)\Delta x
	=\int_{x=-1}^{x=1}\left(\int_{y=x^2}^{y=1} \d y\right)\d x.
\]

Now we can evaluate this iterated integral to conclude
\[
	\text{area of }\mathcal R
	\quad
	=\quad
	\int_{\mathcal R}\d A \quad=\quad 
	\int_{x=-1}^{x=1}\left(\int_{y=x^2}^{y=1} \d y\right)\d x \quad=\quad
	\tfrac{4}{3}.
\]
But, there was another way we could have divided up our original sum.  We could have summed
along rows first and then summed up each row.  Using this approach, we see
\[
	\lim_{\Delta x,\Delta y\to 0} \sum_{y_i}\sum_{x_i} \Delta x\Delta y
	=
	\lim_{\Delta x,\Delta y\to 0} \sum_{y_i}\left(\sum_{x_i} \Delta x\right)\Delta y
	=\int_{y=0}^{y=1}\left(\int_{x=-\sqrt{y}}^{x=\sqrt{y}} \d x\right)\d y.
\]
Computing this integral, we again get $4/3$, as expected.

Note that when we swapped the order of the sums (and hence the order of the integrals)
the bounds changed.  This is worth considering carefully.

When we integrate with respect to $y$ and then $x$, it means that we first imagine
$x$ is fixed and we let $y$ vary between bounds (which may depend on the fixed $x$).
When we integrate with respect to $x$ first, the situation is reversed.  Geometrically,
it's easy to see why our bounds change.

XXX Figure

However, there's also an algebraic argument for why our bounds change.
Suppose $(x,y)\in\mathcal R$.  Then $-1\leq x\leq 1$ and $x^2\leq y\leq 1$.  These were the
bounds we used adding up along columns first.  However, we also know that $0\leq y\leq 1$
and $-\sqrt{x}\leq y\leq \sqrt{x}$, which gives the bounds when adding up
along rows first.  These two system of inequalities don't contradict each
other; they contain the exact same information.  The difference is in
which coordinates you can verify without the others.

Given $-1\leq x\leq 1$ and $x^2\leq y\leq 1$, you can check the $x$-coordinate of the
point $(x_0,y_0)$ without needing the $y$-coordinate.  To check the $y$-coordinate, you 
need to compute $x_0^2$ and so you need the $x$-coordinate.  On the other hand, if you
were given $0\leq y\leq 1$
and $-\sqrt{x}\leq y\leq \sqrt{x}$, you could check the $y$-coordinate without needing
to know what the $x$-coordinate was, but you would need the $y$-coordinate to check the $x$ coordinate.

When computing an iterated integral, we expect to end up with a number.  Thus, the bounds
of the outer-most integral cannot depend on any other variables.  The bounds of the second
outer-most integral can only depend on on variables ``further out.''  Keeping this in mind
will give us a way to quickly judge if we've written down correct bounds for iterated integrals.

Now we turn to integrating other functions over regions.

\begin{example}
	Let $f(x,y) = x^2 + y^2$,
	and let $\mathcal D$ be the region between the
	parabola $x = y^2$ on the left and the line $x = y + 2$
	on the right.  These curves intersect when $y^2=y+2$.  In
	other words, the curves intersect when $y=2$ or $y=1$.  Thus,
	$\mathcal D$ 
	also lies between $y = -1$ below and $y = 2$
	above.   

\begin{center}
	\begin{tikzpicture}
		\coordinate (A) at (1,1);
		\coordinate (B) at (3,2);
		\begin{axis}[
		    anchor=origin,
		    set layers=standard,
		    disabledatascaling,
		    xmin=-1,xmax=5,
		    ymin=-1,ymax=2,
		    x=1cm,y=1cm,
		    grid=both,
		    grid style={line width=.1pt, draw=gray!10},
		    %major grid style={line width=.2pt,draw=gray!50},
		    axis lines=middle,
		    minor tick num=0,
		    enlargelimits={abs=0.5},
		    axis line style={latex-latex},
		    ticklabel style={font=\tiny,fill=white},
		    xlabel style={at={(ticklabel* cs:1)},anchor=north west},
		    ylabel style={at={(ticklabel* cs:1)},anchor=south west}
		]

	    	\addplot[name path=f,no marks,mypink,thick,domain=-1:2, samples=25, draw=none] ({x*x},{x});
	    	\addplot[no marks,mypink,thick,domain=-2:3, samples=20,smooth] ({x*x},{x});
	    	\addplot[name path=g, no marks,green!50!black,thick,domain=-2:5, samples=2] ({x+2},{x});
		\addplot [thick, color=blue, fill=blue, fill opacity=0.05] fill between [of=f and g, split];
		\draw (1,0) node[above] {$\mathcal D$};
		\draw [mypink,fill] (A)  node [above left] {$x=y^2$};
		\draw [green!50!black,fill] (3,1)  node [below right] {$x=y+2$};

		\end{axis}
	\end{tikzpicture}
\end{center}

	Thus
	\begin{align*}
		\int_{\mathcal D} f\, \d A &= 
		\int_{y=-1}^{y=2}\left(\int_{x = y^2}^{x=y+2} x^2 + y^2\, \d x\right) \d y \\
	   &=\int_{-1}^2 \left( \left.\frac{x^3}3 + xy^2\right|_{x=y^2}^{x=y+2}\right) \d y \\
	   &=\int_{-1}^2 \left(\frac {(y + 2)^3}3 + (y + 2)y^2 -
	    \frac{y^6}3 - y^4 \right) \d y \\
	   &=\left. \left(\frac {(y + 2)^4}{12} + \frac{y^4}4 + \frac{2y^3}3 -
	    \frac{y^7}{21} - \frac{y^5}5 \right)\right|_{-1}^2 \\
	   &=\tfrac{256}{12} + \tfrac{16}4 + \tfrac{16}3 - \tfrac{128}{21} - \tfrac{32}5
	     - \tfrac 1{12} - \tfrac 14 + \tfrac 23 - \tfrac 1{21} - \tfrac 15 \\
	   &=\tfrac{639}{35} \approx 18.26.
	\end{align*}
	Note that the region is also bounded vertically by graphs, so in
	principle the integral could be evaluated in the other order.
	However, there is a serious hindrance to trying this.
	The top graph is that of  $y = h_{\text{top}}(x) = \sqrt x$,
	 but the bottom
	graph is described by two different formulas depending on what $x$
	is.  It is a parabola to the left of the point $(1,-1)$ and a line
	to the right of that point, so  
	\[
		h_{\text{bot}}(x) = \begin{cases}
			-\sqrt{x} & \text{ if }0\leq x\leq 1\\
			x-2 &\text{ if }1 < x\leq 4
			\end{cases}.
	\]

\begin{center}
	\usetikzlibrary{patterns,decorations.pathreplacing}
	\begin{tikzpicture}
		\coordinate (A) at (1,1);
		\coordinate (B) at (3,2);
		\begin{axis}[
		    anchor=origin,
		    disabledatascaling,
		    xmin=-1,xmax=5,
		    ymin=-1,ymax=2,
		    x=1cm,y=1cm,
		    grid=both,
		    grid style={line width=.1pt, draw=gray!10},
		    %major grid style={line width=.2pt,draw=gray!50},
		    axis lines=middle,
		    minor tick num=0,
		    enlargelimits={abs=0.5},
		    axis line style={latex-latex},
			xticklabels={,,},
			yticklabels={,,}
		]

	    	\addplot[name path=f,no marks,mypink,thick,domain=-1:2, samples=25, draw=none] ({x*x},{x});
	    	\addplot[no marks,mypink,thick,domain=0:3, samples=25] ({x*x},{x});
	    	\addplot[no marks,blue,thick,domain=-2:0, samples=25] ({x*x},{x});
	    	\addplot[name path=g, no marks,green!50!black,thick,domain=-2:5, samples=2] ({x+2},{x});
		\addplot [thick, color=blue, fill=blue, fill opacity=0.05] fill between [of=f and g, split];

		\addplot[color=gray, dashed, thick, domain=-3:3] ({1},{x});
		\draw (1.2,0) node[above right] {$\mathcal D$};
		\draw [mypink,fill=white] (A)  node [above left] {$y=\sqrt{x}$};
		\draw [blue,fill=white] (.5,-1)  node [left] {$y=-\sqrt{x}$};
		\draw [green!50!black,fill] (3,1)  node [below right] {$y=x+2$};

		\end{axis}
	\end{tikzpicture}
\end{center}

	(The $x$-values at the relevant points are determined from the corresponding
	$y$-values which were calculated above.)  That means to actually
	compute the integral we must decompose the region
	$\mathcal D$ into two subregions meeting along the line $x = 1$ and treat each
	one separately.   Doing so,
	\[
		\int_{\mathcal D} f\, \d A
	=
	\int_{x=0}^{x=1}\left(\int_{y = -\sqrt x}^{y= \sqrt x} x^2 + y^2\, \d y\right) \d x 
	+
	\int_{x=1}^{x=4}\left(\int_{y = x-2}^{y= \sqrt x} x^2 + y^2\, \d y\right) \d x .
	\]
	You should work out the two iterated integrals on the right just to check
	that their sum gives the same answer as above.

\end{example}

\subsection{Integral Notation}

In single-variable calculus, you used the notation $\tfrac{\d}{\d x}f(x)=\tfrac{\d f}{\d x}(x)$ 
to represent the derivative
of the function $f$ and $\int_{a}^b f(x)\d x$ to represent the integral from $a$ to $b$ of the function
$f$.  This notation was invented by Gottfried Leibniz in the 1600s and was inspired by geometric thinking.
Though the infinitely large and infinitely small need to be handled with mathematical care,
the idea that ``$\d x$'' represents and infinitesimally small change in the variable $x$ gives
rise to fruitful intuitions.

We will extend on this idea.  Consider $f:\R^n\to\R$ and a region $\mathcal R\subseteq \R^n$.
We will write
\[
	\int_{\mathcal R} f\,\d V
\]
to notate \emph{the integral of $f$ over the region $\mathcal R$ with respect to volume}.
If $f:\R^2\to\R$, we may write $\d A$ in place of $\d V$, since a two-dimensional volume
is traditionally called an area.  If the context is clear, we might even omit the $\d V$
entirely, opting to write
\[
	\int_{\mathcal R} f.
\]

This notation can replace single-variable calculus notation.  For example
\[
	\int_a^b f(x)\,\d x = \int_{[a,b]} f.
\]
As always, the purpose of notation is to facilitate our thinking, not to be a formal grammar\footnote{
	Sometimes notation's dual purpose is to be a formal grammar.  For instance,
	most programming languages could be though of as very stringent 
	systems of notation.}, and so you should use whatever notation
most clearly conveys your thoughts.  For us, $\int_{\mathcal R}f$ will often suffice.

When we write iterated integrals, we will omit parenthesis.  For example, we write
\[
	\int \left(\int f(x,y)\,\d x\right)\d y\qquad\text{as}\qquad \int \int f(x,y)\,\d x\d y,
\]
and to keep ourselves from getting the bounds of integration mixed up, we will label
the bounds of integration for iterated integrals with the variable they bound.  For example,
we write
\[
	\int_{y=a}^{y=b}\int_{x=c}^{x=d} f(x,y)\,\d x\d y\qquad\text{instead of}\qquad
	\int_{a}^{b}\int_{c}^{d} f(x,y)\,\d x\d y.
\]
This will prevent us from becoming confused when we swap the order of integration.

Finally, it is worth being aware of other notation that you might encounter.  For example,
some textbooks write
\[
	\iint f\,\d A\qquad\text{and}\qquad \iiint f\,\d V
\]
for integrals with respect to area and volume.  The motivation being that the number of integral
signs should match the number of dimensions you're integrating over.

\subsection{Iterated Integrals}

We've already seen how an integral over a region can be turned into an iterated
integral in multiple ways.  We will turn this into a technique for evaluating
some difficult iterated integrals.  Namely, if we are given an iterated integral,
we can try to convert it into an integral over a region, and then convert it back to a different
iterated integral.

\begin{example}
	Consider the iterated integral
	\[
		\int_{x=0}^{x=1}\int_{y=x}^{y=1} \frac{\sin y}{y}\d y \d x.
	\]
	This is the iterated integral obtained from the  integral
	of the function $f(x,y) = \sin y/y$ over the triangular
	region $\mathcal D$ contained
	between the vertical lines  $x = 0, x = 1$, the line
	$y = x$ below, and the line $y = 1$ above.

	\begin{center}
	\begin{tikzpicture}
		\coordinate (A) at (1,1);
		\coordinate (B) at (3,2);
		\begin{axis}[
		    anchor=origin,
		    set layers=standard,
		    disabledatascaling,
		    xmin=-.5,xmax=1,
		    ymin=-.5,ymax=1,
		    x=2.2cm,y=2.2cm,
		    grid=both,
		    grid style={line width=.1pt, draw=gray!10},
		    %major grid style={line width=.2pt,draw=gray!50},
		    axis lines=middle,
		    minor tick num=1,
		    enlargelimits={abs=0.4},
		    axis line style={latex-latex},
		    ticklabel style={font=\tiny,fill=white},
		    xlabel style={at={(ticklabel* cs:1)},anchor=north west},
		    ylabel style={at={(ticklabel* cs:1)},anchor=south west}
		]

	    	\addplot[name path=f,no marks,mypink,thick,domain=0:1, samples=2, draw=none] ({x},{x});
	    	\addplot[no marks,mypink,thick,domain=-2:3, samples=20,smooth] ({x},{x});
	    	\addplot[name path=g, no marks,green!50!black,thick,domain=-2:5, samples=2] ({x},{1});
			\addplot[no marks,green!50!black,thick,domain=-2:5, samples=2] ({0},{x});
			\addplot[no marks,green!50!black,thick,domain=-2:5, samples=2] ({1},{x});
		\addplot [thick, color=blue, fill=blue, fill opacity=0.05] fill between [of=f and g, split,soft clip={domain=0:1}];
		\draw (.33,.66) node[] {$\mathcal D$};
		\draw [mypink,fill] (.5,.5)  node [below right] {$y=x$};
		\draw [green!50!black,fill] (.5,1)  node [above] {$y=1$};
		\draw [green!50!black,fill] (1,-.75)  node [left] {$x=1$};
		\draw [green!50!black,fill] (0,-.75)  node [left] {$x=0$};

		\end{axis}
	\end{tikzpicture}
	\end{center}

	   The indefinite
	integral (anti-derivative) 
	\[
	     \int \frac{\sin y}y dy
	\]
	\emph{cannot be expressed in terms of known elementary functions}.
	(Try to integrate it or look in an integral table if you don't
	believe that.)   Hence, the iterated integral cannot be evaluated
	by anti-derivatives.  However, the triangular region may be
	described just as well by bounding it horizontally by graphs:
	it lies between $y = 0$ and $y = 1$ and for each $y$ between
	$x = 0$ and $x = y$.  

	\begin{center}
	\begin{tikzpicture}
		\begin{axis}[
		    anchor=origin,
		    set layers=standard,
		    disabledatascaling,
		    xmin=-.5,xmax=1,
		    ymin=-.5,ymax=1,
		    x=2.2cm,y=2.2cm,
		    grid=both,
		    grid style={line width=.1pt, draw=gray!10},
		    %major grid style={line width=.2pt,draw=gray!50},
		    axis lines=middle,
		    minor tick num=1,
		    enlargelimits={abs=0.4},
		    axis line style={latex-latex},
		    ticklabel style={font=\tiny,fill=white},
		    xlabel style={at={(ticklabel* cs:1)},anchor=north west},
		    ylabel style={at={(ticklabel* cs:1)},anchor=south west}
		]

	    	\addplot[name path=f,no marks,mypink,thick,domain=0:1, samples=2, draw=none] ({x},{x});
	    	\addplot[no marks,mypink,thick,domain=-2:3, samples=20,smooth] ({x},{x});
	    	\addplot[name path=g, no marks,green!50!black,thick,domain=-2:5, samples=2] ({x},{1});
			\addplot[no marks,green!50!black,thick,domain=-2:5, samples=2] ({0},{x});

		\addplot [thick, color=blue, fill=blue, fill opacity=0.05] fill between [of=f and g, split,soft clip={domain=0:1}];
		\draw (.25,.75) node[] {$\mathcal D$};
		\draw [mypink,fill] (.5,.5)  node [below right] {$y=x$};
		\draw [green!50!black,fill] (.5,1)  node [above] {$y=1$};
		\draw [green!50!black,fill] (0,-.75)  node [left] {$x=0$};

		\coordinate (A) at (-.01,.58);
		\coordinate (B) at (-1,.57);
		\coordinate (C) at (.6,.58);
		\coordinate (D) at (1.2,.57);

		\draw[fill=gray,draw=none] (0,.57) rectangle (.6,.6);
		\end{axis}


	
		\path[thick] (B) edge[bend left, ->] node[]{} (A);
		\draw[] (B) node[left] {$(0,y_0)$};
		\path[thick] (D) edge[bend right, ->] node[]{} (C);
		\draw[] (D) node[right] {$(y_0,y_0)$};

	\end{tikzpicture}
	\end{center}

	 Thus, the double integral can be evaluated
	from the  iterated integral
	\begin{align*}
		\int_{y=0}^{y=1}\int_{x=0}^{x=y}\frac{\sin y}y\d x\d y &=
		\int_0^1\, \left. \frac{\sin y}y x\right|_{x=0}^{x=y}\, \d y  \\
		&= \int_0^1\,  \frac{\sin y}y y\, \d y
		= \int_0^1\ \sin y\, \d y  \\
		&= \left. -\cos y\right|_0^1 = 1 - \cos 1\approx 0.46.
	\end{align*}

	Note that in order to set up the iterated integral in the other
	order, \emph{we had to draw a diagram} and work directly from
	that.   \emph{There are no algebraic rules which will allow you
	to switch orders without using a diagram.}
\end{example}

Swapping the order of integration is a neat trick, and if an
iterated integral comes from the integral of some function over a
region, then we can always swap the order of integration\footnote{ Depending
on how we define things, this may be a circular argument.  In our definition
of an integral over a region, we had to sum up a bunch of tiny areas in an unspecified
order.  We might \emph{define} this to make sense only if every possible order we could
sum results in the same number.}.  However, if we're given an iterated integral,
a priori, we don't know that we can swap the order of integration.

\begin{example}
	Let $f(x,y)=\frac{x^2-y^2}{(x^2+y^2)^2} = 
	-\tfrac{\partial}{\partial x}\left(\tfrac{\partial}{\partial y}\arctan(y/x)\right)$.  Now,
	\[
		\int_{x=0}^{x=1}\int_{y=1}^{y=1} f(x,y)\,\d y\d x = \frac{\pi}{4}
	\]
	and
	\[
		\int_{y=0}^{y=1}\int_{x=1}^{x=1} f(x,y)\,\d x\d y = -\frac{\pi}{4}.
	\]
	The order of integration matters for $f$ and so the iterated integrals $\iint f\, \d x\d y$
	and $\iint f\,\d y\d x$ cannot have come from the integration of $f$ over a region---an
	integral of $f$ over a region cannot be reasonably defined.
\end{example}

For us, it will be uncommon to come across iterated integrals where the order of integration
cannot be changed.  However, the following theorem from analysis gives us explicit conditions
for when it is okay.

\begin{theorem}[Fubini's Theorem]
	Let $f:\R^2\to R$ and suppose that 
	\[
		\int_{x=a}^{x=b}\int_{y=c}^{y=d} \abs{f(x,y)}\,\d y\d x < \infty.
	\]
	Then,
	\[
		\int_{x=a}^{x=b}\int_{y=c}^{y=d} f(x,y)\,\d y\d x =
		\int_{y=c}^{y=d}\int_{x=b}^{x=a} f(x,y)\,\d x\d y.
	\]
\end{theorem}\index{Fubini's Theorem}

\begin{exercises}
\end{exercises}

\section{The Volume Form}

Let $f:\R^2\to \R$ and $\mathcal D\subseteq \R^2$ be the unit disk centered at the origin.
If we wanted to compute $\int_{\mathcal D} f\,\d A$, we could split it up into an iterated
integral like so:
\[
	\int_{\mathcal D} f\,\d A = \int_{x=-1}^{x=1}\int_{y=-\sqrt{1-x^2}}^{y=\sqrt{1-x^2}} f(x,y)\,\d y\d x.
\]
However, the thought of integrating square roots makes me cringe.  Especially, when $\mathcal D$ has
a simple description in polar coordinates.  If we could use polar coordinates, our iterated
integral might look more like $\int_{\theta=0}^{\theta=2\pi}\int_{r=0}^{r=1} f$.  Much less intimidating!

Let's see if we can figure out how to integrate in polar coordinates.  The principle should be the same.
If we divide $\mathcal D$ into tiny \emph{sectors}, then
\[
	\int_{\mathcal D} f\,\d A =\lim_{\text{sector size}\to 0} \sum_{\text{sectors}\ \in\ \mathcal D} (\text{sector
	area})f(\text{sector location}).
\]
When we previously used iterated integrals to compute integrals over regions, we chopped 
our region into little rectangles of width $\Delta x$ and height $\Delta y$.  This meant
that the area of each sector was $\Delta x\Delta y$, and so we had the formula
\[
	\int_{\mathcal D} f\,\d A =\lim_{\text{sector size}\to 0} \sum_{\text{sectors}\ \in\ \mathcal D} 
	(\Delta x\Delta y) f(x,y).
\]
For polar coordinates\index{polar coordinates}, 
the natural way to chop up $\mathcal D$ is not into rectangles, but
into \emph{annular sectors}.

XXX Figure comparing rectangular chopping and annular sector chopping.

When we compute the exact area of the annular sector between the curves
$\theta=\theta_0$, $\theta=\theta_0+\Delta\theta$, $r=r_0$, and $r=r_0+\Delta r$, 
we get 
\[
	\text{annular sector area} = r_0\Delta r\Delta \theta + \tfrac{1}{2}\Delta\theta(\Delta r)^2.
\]
The first thing to note is that the quantity $r_0$ shows up in the area formula.  
This makes sense, because if you fix two $\theta$ values, the further ``out'' (i.e., the larger
$r_0$ is) you are, the larger the area.  This didn't happen in rectangular coordinates---all the rectangular
sectors had exactly the same area.

Now, just as before, we can write a sum expression for our integral.
\begin{equation}
	\label{EQPOLARSUM}
	\int_{\mathcal D} f\,\d A =\lim_{\Delta r,\Delta \theta\to 0}
	\sum_{r=0,\Delta r,2\Delta r,\ldots,1}\ \ \sum_{\theta=0,\Delta\theta,
	2\Delta\theta, \ldots, 2\pi} \Big(
		r\Delta r\Delta \theta + \tfrac{1}{2}\Delta\theta(\Delta r)^2
	\Big)f(r,\theta)
\end{equation}
However, it isn't clear how to rewrite Equation \eqref{EQPOLARSUM} as an iterated integral
for easy evaluation.

\subsection{Order Analysis}
We can rewrite Equation \eqref{EQPOLARSUM} into one that looks more like a Riemann sum by
doing some \emph{order analysis}\index{order analysis}.  Order analysis allows us to determine
with precision
which parts of an expression become negligible and which parts are non-negligible\footnote{ \emph{Negligible}
is a fancy way of saying ``this can be safely ignored.''}.  Often
times order analysis will show that we can ignore or erase large parts of a formula and still
get the correct answer in the end.


Consider the right hand side of Equation \eqref{EQPOLARSUM}.
\[
	\lim_{\Delta r,\Delta \theta\to 0}
	\sum_{r=0,\Delta r,2\Delta r,\ldots,1}\ \ \sum_{\theta=0,\Delta\theta,
	2\Delta\theta, \ldots, 2\pi} \Big(
		r\Delta r\Delta \theta + \tfrac{1}{2}\Delta\theta(\Delta r)^2
	\Big)f(r,\theta)
\]
In this expression, both $\Delta r$
and $\Delta \theta$ are tending towards $0$ and so 
\[
	r\Delta r\Delta \theta + \tfrac{1}{2}\Delta\theta(\Delta r)^2\to 0.
\]
But, we don't expect the entire limit to go to zero because as $\Delta\theta$ and $\Delta r$
get smaller, each of the two sums get more and more terms.  The outer sum will have roughly
$1/\Delta r$ terms and the inner sum will have roughly $1/\Delta \theta$ terms\footnote{
We say roughly because the number of terms involved is always an integer and the quantities
$1/\Delta r$ and $1/\Delta \theta$ may not be integers.
}.  
Now, $\sum_{i=1}^n K=nK$ if $K$ is a constant.  The inside of our sums are not constant, but if
they were bounded, we could replace them with constants and apply squeeze-theorem arguments to them.
For order analysis, since we are only interested in whether a term
\emph{must} go to zero, we bypass the detailed squeeze theorem arguments 
by just imagining all bounded terms inside our sum are constant and seeing what happens.
To proceed, we first split up
our sum to figure out which parts are negligible.
\[
			\sum_{r=0,\Delta r,2\Delta r,\ldots,1}\ \ \sum_{\theta=0,\Delta\theta,
	2\Delta\theta, \ldots, 2\pi} \Big(
		r\Delta r\Delta \theta + \tfrac{1}{2}\Delta\theta(\Delta r)^2
		\Big)f(r,\theta)
\]
\[
	\phantom{XXXXXXXX}=\sum_{\theta}\sum_{r} r\Delta r\Delta \theta f(r,\theta)
	+ 
		\sum_{\theta}\sum_{r} \tfrac{1}{2}(\Delta r)^2\Delta \theta f(r,\theta)
\]
Doing order analysis on the first part of the sum, we see
\begin{align*}
	\lim_{\Delta r,\Delta\theta\to 0}
	\sum_{\theta}\sum_{r} r\Delta r\Delta \theta f(r,\theta)
	&\sim
	\lim_{\Delta r,\Delta\theta\to 0}
	\frac{1}{\Delta r}\left(\frac{1}{\Delta \theta} \Big(r\Delta r\Delta\theta f(r,\theta)\Big)\right)\\
	&=rf(r,\theta),
\end{align*}
where $\sim$ means \emph{order equivalent}\footnote{ There is a formal theory of
order equivalence and a slew of specialized notations.  We don't need those formalities
right now.}.  Since this term is not order-equivalent to zero, it is non-negligible, and so 
we must keep it.  

Analyzing the other component of the sum tells a different story:
\begin{align*}
	\lim_{\Delta r,\Delta\theta\to 0}
	\sum_{\theta}\sum_{r} \tfrac{1}{2}(\Delta r)^2\Delta \theta f(r,\theta)
	&\sim
	\lim_{\Delta r,\Delta\theta\to 0}
	\frac{1}{\Delta r}\left(\frac{1}{\Delta \theta} \Big(\tfrac{1}{2}(\Delta r)^2\Delta\theta f(r,\theta)\Big)\right)\\
	&=\lim_{\Delta r,\Delta\theta \to 0}\tfrac{1}{2}\Delta rf(r,\theta) = 0,
\end{align*}
and so this part of the sum \emph{is} negligible.  We may therefore conclude,
\[
	\lim_{\Delta r,\Delta \theta\to 0}
	\sum_{r=0,\Delta r,2\Delta r,\ldots,1}\ \ \sum_{\theta=0,\Delta\theta,
	2\Delta\theta, \ldots, 2\pi} \Big(
		r\Delta r\Delta \theta + \tfrac{1}{2}\Delta\theta(\Delta r)^2
	\Big)f(r,\theta)
\]
\[
	=\lim_{\Delta r,\Delta \theta\to 0}
	\sum_{r=0,\Delta r,2\Delta r,\ldots,1}\ \ \sum_{\theta=0,\Delta\theta,
	2\Delta\theta, \ldots, 2\pi} 
		r\Delta r\Delta \theta f(r,\theta),
\]
which is looking a lot more line a Riemann sum that can be converted into an iterated integral.
In fact, if $f$ is a reasonable function (for example, one where Fubini's theorem would hold),
\[
	\lim_{\Delta r,\Delta \theta\to 0}
	\sum_{r=0,\Delta r,2\Delta r,\ldots,1}\ \ \sum_{\theta=0,\Delta\theta,
	2\Delta\theta, \ldots, 2\pi} 
		r\Delta r\Delta \theta f(r,\theta)
		=\int_{\theta=0}^{\theta=2\pi}\int_{r=0}^{r=1}f(r,\theta)r\d r\d \theta.
\]

Order analysis can be very powerful, but we should take note about the assumptions we made.
First off, for the order analysis that we did, we assume $f$ was \emph{bounded} on the region
$\mathcal D$.  We could still do order analysis on an unbounded function, but it would take more 
care\footnote{
	Order analysis on $\sum_{i=1}^n i$ shows that it is order $n^2$, not order $n$.
	This is because the part inside the sum, $i$, is not bounded.
}.  Further, it'd be nice if there were a quicker way to come up with an iterated
integral in various coordinate systems, rather than lugging around Riemann sums
all the time---and there is.

\subsection{The Volume Form}

The \emph{volume form} gives a formal way to rewrite integrals as iterated integrals
with respect to different coordinate systems.  What it really does is give a way
to relate areas or volumes described in a particular coordinate system to ``true''
areas and volumes in Euclidean space---and if your mind is tingling with thoughts
of parameterizations, you're on the right track.


Let's think back to line integrals.  Suppose $\vec p:[a,b]\to\R^n$ is an
arc-length parameterization
of the curve $\mathcal S$. Then, if $f:\R^n\to\R$,
\[
	\int_{\mathcal S} f = \int_{a}^b f\circ \vec p(t) \d t.
\]
However, if $\vec q:[c,d]\to\R^n$ is a non-arc-length parameterization of $\mathcal S$, we have to
use a different formula.  Namely,
\[
	\int_{\mathcal S} f = \int_{c}^d f\circ \vec q(t)\norm{\vec q\,'(t)} \d t.
\]
In this expression, $\norm{\vec q\,'(t)}$ compensates for the fact that as the parameter
$t$ marches on, the arc-length traced out by $\vec q$ is non-uniform.  The volume form\index{volume form} does
the same thing but for parameterizations of multi-dimensional regions.


\begin{definition}[Volume Form]
	Suppose $\mathcal F$ is a coordinate system for $\R^2$ with a relationship
	to rectangular coordinates given by $(x,y)=\vec f(a,b)$.
	The \emph{pre-volume form} associated with $\mathcal F$ at the point $(a,b)$
	is written $\Delta V(a,b)$ and is defined to be the area of
	$\vec f\big([a,a+\Delta a]\times [b,b+\Delta b]\big)$.
	
	The \emph{volume form} associated with $\mathcal F$ at the point $(a,b)$
	is the infinitesimal $\d V(a,b)=V(a,b)\d a\d b$ where $V:\R^2\to\R$ is the 
	unique function satisfying
	\[
		\lim_{\Delta a,\Delta b\to 0} \frac{\Delta V(a,b)}{V(a,b)\Delta a\Delta b}=1.
	\]
\end{definition}
We only wrote down the definition of the volume form for two-dimensional coordinate systems,
but the definition for three and higher dimensions is similar.  To define the
volume form in an $n$-dimensional coordinate system, you take the image
of $n$-dimensional rectangles instead of two-dimensional ones and you replace $\Delta a\Delta b$
with $\Delta x_1\Delta x_2\Delta x_3\cdots\Delta x_n$.

There are a few peculiarities in the definition of the volume form.
First, the word ``infinitesimal'' shows up in the definition.  As of yet,
this has not been a formal term for us.  Further, the volume form has
``$\d a\d b$'' built into it.  These features hark back to the origins
of the volume form.  The term \emph{volume form}
comes from the field of differential geometry, and in differential geometry,
$\d a$ has a unique meaning all by itself and is called a \emph{differential form}.  
Differential forms can be multiplied with each other and added together and the
result still has meaning.
For
us, $\d a$  by itself has no precise
meaning and only appears as a notational book-keeping device in integrals and derivatives.
Of course, the intuitive meaning of $\d a$ is that it is an infinitesimal quantity.
At this point, we won't dive into the details of how and why differential geometry and differential forms
work, but we
will use the language of differential geometry.

\begin{example}
	Compute the volume form for polar coordinates.

	We know that $p:[0,\infty)\times [0,2\pi)\to\R^2$ given 
	by $\vec p(r,\theta)=(r\cos\theta,r\sin\theta)$ converts from polar to rectangular
	coordinates.  Further, we already computed the pre-volume form for polar
	coordinates.  Namely, $\Delta V(r,\theta) = r\Delta r\Delta \theta+\tfrac{1}{2}(\Delta r)^2\Delta \theta$.

	XXX Figure


	To compute the volume form for polar coordinates, we need to find a function $V$ so that
	\begin{align*}
		\lim_{\Delta r,\Delta \theta\to 0}
		\frac{r\Delta r\Delta \theta+\tfrac{1}{2}(\Delta r)^2\Delta \theta}{V(r,\theta)\Delta r\Delta\theta}
		&=\lim_{\Delta r,\Delta \theta\to 0}
		\frac{r}{V(r,\theta)} + \frac{\Delta r}{2V(r,\theta)}\\
		&=\frac{r}{V(r,\theta)} = 1.
	\end{align*}
	There's a clear choice.  Namely, $V(r,\theta)=r$.  Therefore, the volume form for
	polar coordinates is $r\d r\d \theta$.

\end{example}

\subsection{Using the Volume Form}

Once we have gone through the work of finding the volume form for a particular
coordinate system, writing down iterated integrals in that coordinate
system becomes easy---you can just replace ``$\d V$'' with the volume form
and integrate away.

Intuitively the volume form plays the same role that $\norm{\vec q\,'(t)}\d t$ did
in line integrals.  It \emph{compensates} for any stretching or warping that
an alternate coordinate system may do to area.

\begin{example}
	Let $f:\R^2\to\R$ be defined by $f(x,y)=x^2+y^2$ and let $\mathcal D$ be the unit
	disk centered at the origin.  Compute $\int_{\mathcal D} f\,\d V$.

	We've already computed the volume form in polar coordinates.  It is
	$r\d r\d\theta$.  Further, $f$ is easy to describe in polar coordinates,
	since $x^2+y^2=r^2$.  Lastly, in polar coordinates, $\mathcal D$ 
	is the region where $r\in[0,1]$ and $\theta\in[0,2\pi)$.  Therefore,
	\[
		\int_{\mathcal D}f\,\d V =
		\int_{\theta=0}^{\theta=2\pi}\int_{r=0}^{r=1} (r^2)r\d r\d \theta
		=\int_{\theta=0}^{\theta=2\pi}\int_{r=0}^{r=1} r^3\d r\d \theta.
	\]
\end{example}

\begin{example}
	Consider the two-dimensional 
	skewed coordinate system $\mathcal S$.  The $\mathcal S$
	coordinate system uses the variables $a$ and $b$ and relates
	to rectangular coordinates by the equations $x=a+b$ and $y=a-b$.
	Let $f:\R^2\to\R$ be given by the formula $f(a,b)=a+6b$ and let $\mathcal R\subseteq\R^2$
	be the region defined in $\mathcal S$ coordinates by the inequalities
	$1\leq a\leq 2$ and $4\leq b\leq 6$.  Compute $\int_{\mathcal R} f\d V$.

	At present, we have two options.  We could rewrite everything in 
	rectangular coordinates or we could figure out the volume form for $\mathcal S$ 
	coordinates and set our iterated integral up with respect to $\mathcal S$ coordinates.
	Let's do the second way---it sounds more fun.

	Finding the pre-volume form is often the hardest part.  It helps 
	to write down change-of-coordinate functions and draw a picture.
	First off, let $\vec s:\R^2\to\R^2$ be the function that takes points
	written in $\mathcal S$ coordinates and rewrites them in rectangular coordinates.
	From the relationships given above, we see $\vec s(a,b) = (a+b,a-b)$.  Now
	we can draw $[a_0,a_0+\Delta a]\times [b_0,b_0+\Delta b]$ in the $ab$-coordinate
	plane and draw $\vec s([a_0,a_0+\Delta a]\times [b_0,b_0+\Delta b])$ in the
	$xy$-coordinate plane.

	\begin{center}
	\begin{tikzpicture}
		\coordinate (A) at (1,1);
		\coordinate (B) at (3,2);
		\coordinate (D) at ($(B)-(A)$);
		\begin{axis}[
			name=plot1,
		    anchor=origin,
		    disabledatascaling,
		    xmin=-1,xmax=5,
		    ymin=-1,ymax=3,
		    x=1cm,y=1cm,
		    grid=both,
		    grid style={line width=.1pt, draw=gray!10},
		    %major grid style={line width=.2pt,draw=gray!50},
		    axis lines=middle,
		    minor tick num=0,
		    enlargelimits={abs=0.5},
		    axis line style={latex-latex},
			xticklabels={,,},
			yticklabels={,,},
			xlabel={$a$},
			ylabel={$b$},
		    xlabel style={at={(ticklabel* cs:1)},anchor=west},
		    ylabel style={at={(ticklabel* cs:1)},anchor=south}
		]

	\draw[fill=blue,opacity=0.05] (1,1) rectangle (1.8,2.5);	
\addplot[color=myorange, dashed, thick, domain=-3:5] ({1},{x}) node[midway, left,yshift=.5cm] {$a=a_0$};
\addplot[color=myorange, dashed, thick, domain=-3:5] ({1.8},{x}) node[midway, right,yshift=.5cm] {$a=a_0+\Delta a$};
\addplot[color=blue, dashed, thick, domain=-3:5] ({x},{1}) node[midway, below left] {$b=b_0$};
\addplot[color=blue, dashed, thick, domain=-3:5] ({x},{2.5}) node[midway,above left] {$b=b_0+\Delta b$};
	

		\end{axis}

		\draw[] (2.5,3.5) node[above right] {Area in $ab$-plane = $\Delta a\Delta b$};
		\path[thick] (3.5,3.5) edge[bend left, ->] node[] {} (1.5,1.8);


		\begin{axis}[
			%at=(plot1.east),
			at={(7.5,0)},
		    anchor=origin,
		    disabledatascaling,
		    xmin=-1,xmax=5,
		    ymin=-1.4,ymax=3,
		    x=1cm,y=1cm,
		    grid=both,
		    grid style={line width=.1pt, draw=gray!10},
		    %major grid style={line width=.2pt,draw=gray!50},
		    axis lines=middle,
		    minor tick num=0,
		    enlargelimits={abs=0.5},
		    axis line style={latex-latex},
			xticklabels={,,},
			yticklabels={,,},
			xlabel={$x$},
			ylabel={$y$},
		    xlabel style={at={(ticklabel* cs:1)},anchor=west},
		    ylabel style={at={(ticklabel* cs:1)},anchor=south}
		]

	\draw[fill=blue,opacity=0.05] (2,0) -- (3.5,-1.5) -- (4.3,-.7) -- (2.8,.8);	
\addplot[name path=aone, color=myorange, dashed, thick, domain=-3:5] ({1+x},{1-x}) node[near start, left] {$a=a_0$};
\addplot[name path=atwo, color=myorange, dashed, thick, domain=-3:5] ({1.8+x},{1.8-x}) node[near start, right] {$a=a_0+\Delta a$};
\addplot[name path=bone, color=blue, dashed, thick, domain=-3:5] ({x+1},{x-1}) node[at={(0,-2)}, above left] {$b=b_0$};
\addplot[name path=btwo, color=blue, dashed, thick, domain=-3:5] ({x+2.5},{x-2.5}) node[at={(3,-2)}, above left] {$b=b_0+\Delta b$};

	
		\end{axis}	
		\draw[] (2.5+7.5,3.5) node[above right,xshift=-1cm] {Area in $xy$-plane = $V(a_0,b_0)\Delta a\Delta b$};
		\path[thick] (3.5+7.5,3.5) edge[bend left, ->] node[] {} (3+7.5,.2);

	\end{tikzpicture}
	\end{center}

	Conveniently enough, the result is still a rectangle in the $xy$-coordinate plane, so 
	it's area is easy to compute (after finding the lengths of each
	side of the rectangle): $2\Delta a\Delta b$.

	Now, we're looking for a function $V:\R^2\to\R$ so that
	\[
		\lim_{\Delta a,\Delta b\to 0} \frac{2\Delta a\Delta b}{V(a,b)\Delta a\Delta b} = 1,
	\]
	and such a function can be directly guessed.  It's given by $V(a,b)=2$.  Thus,
	the volume form for the $\mathcal S$ coordinate system is $2\d a\d b$.

	Since everything else is already written in terms of the $\mathcal S$ coordinate system,
	we can just write down the iterated integral.
	\begin{align*}
		\int_{\mathcal R} f\,\d V &= \int_{b=4}^{b=6}\int_{a=1}^{a=2} f(a,b)\, 2\d a\d b\\
			&= \int_{b=4}^{b=6}\int_{a=1}^{a=2} 2a+12b\,\d a\d b=126
		.
	\end{align*}
\end{example}

Volume forms are really convenient, but you may be asking: didn't you just stress that
in an expression like $\iint \d a\d b$, the symbols $\d a$ and $\d b$ are not being multiplied
and are just notation?  When we use volume forms aren't we treating $\d a \d b$ like it has meaning
and then working backwards to write down an integral?  This is an astute observation.  Yes,
when we use volume forms, we're assuming that everything will work out as purported.  We won't prove
the details (if you're interested, look in a book on differential geometry), but it is worth noting
that volume forms only exist for \emph{smooth} parameterizations.  That means, every component function
of the parameterization is differentiable in every coordinate.  Most coordinate systems that we
will come across have this property.

When trying to write down an iterated integral in non-rectangular coordinates, computing the volume
form can be the hardest part.  We've been computing volume forms directly from the definition, but
there are several ways that avoid the trouble of finding the pre-volume form.  They include 
the algebra of differential forms and the Jacobian.  We won't discuss the details of these computational
tools here, but it is worth listing the volume forms for some common coordinate systems.

\begin{center}
	\begin{tabular}{c c}
		Coordinate System & Volume Form\\
		\hline
		Rectangular Coordinates in $\R^2$ & $\d x\d y$\\
		Polar Coordinates & $r\d r\d \theta$\\
		Rectangular Coordinates in $\R^3$ & $\d x\d y\d z$\\
		Cylindrical Coordinates & $r\d r\d \theta\d z$\\
		Spherical Coordinates & $r^2\sin\phi\d r\d \phi\d \theta$
	\end{tabular}
\end{center}

\subsection{The Chain Rule for Volume Forms}
Suppose $\vec p:\R^2\to\R^2$ is a parameterization.  We can think of $\vec p$ as a function
that takes points in one coordinate system and outputs points in rectangular coordinates.
Call the coordinate system that $\vec p$ inputs the $\mathcal P$ coordinate system with variables
$\alpha$ and $\beta$.  Now, there is some volume form associated with the $\mathcal P$ coordinate
system and it takes the form $V_{\vec p}(\alpha,\beta)\d\alpha \d\beta$.  We interpret $V_{\vec p}(\alpha,\beta)$
as the amount that $\vec p$ stretches or shrinks volume by at the point $(\alpha,\beta)$ (in $\mathcal P$ coordinates).

Now, suppose $\vec q:\R^2\to\R^2$ is another parameterization.  Similarly to $\vec p$, associate
$\vec q$ with the $\mathcal Q$ coordinate system with variables $\gamma$ and $\delta$.  Let
$V_{\vec q}(\gamma,\delta)\d\gamma\d\delta$ be the volume form for $\mathcal Q$ coordinates.

Now, $\vec p$ and $\vec q$ relate to coordinate systems, but they are really just parameterizations with domain
$\R^2$ and range $\R^2$.  So, $\vec s=\vec q\circ \vec p$ is another parameterization.
Associate it with the $\mathcal S$ coordinate system with the variables $a$ and $b$.  Can we write the volume
form for the $\mathcal S$ coordinate system in terms of the volume form for $\mathcal P$ and
$\mathcal Q$ coordinates?

Yes, we can!  The process is fairly straightforward.  We know $V_{\vec p}(\alpha,\beta)$ is how
much $\vec p$ changes area by at the point $(\alpha,\beta)$ and we know $V_{\vec q}(\gamma,\delta)$
is how $\vec q$ changes area at the point $(\gamma,\delta)$.  Further, considering the composition
$\vec s=\vec q\circ \vec p$, we could interpret $\vec p$ as inputing $\vec P$ coordinates and then
outputting $\vec Q$ coordinates, which then get taken by $\vec q$ and finally converted to rectangular coordinates.
Thus, at the point $(\alpha,\beta)$, area first gets changed by $V_{\vec p}(\alpha,\beta)$
and then at the point $\vec p(\alpha,\beta)$, area gets changed by $V_{\vec q}\circ \vec p(\alpha,\beta)$.
Change in area is multiplicative, so
\begin{equation}
	\label{EQVOLFORMCHAINRULE}
	V_{\vec s}(\alpha,\beta) = V_{\vec p}(\alpha,\beta)\Big(V_{\vec q}\circ \vec p(\alpha,\beta)\Big).
\end{equation}
Does that look a little like the chain rule?

Let's try this again in one dimension.  Let $f:\R\to\R$ and $g:\R\to\R$ both be parameterizations.
In one dimension the limit involved in finding a volume form turns out to be equivalent to taking
a derivative (go ahead, compute the one-dimension volume form of a few of your favorite functions!).
Therefore,
\[
	V_{f}(a) = f'(a)\qquad\text{and}\qquad V_{g}(b) = g'(b).
\]
Let $h=g\circ f$.  Expanding with Equation \eqref{EQVOLFORMCHAINRULE}, we get
\[
	h'(a) = V_{h}(a) = V_f(a)\Big(V_g\circ f(a)\Big) = f'(a)\Big(g'\circ f(a)\Big),
\]
which is exactly what the chain rule says!  This is no coincidence. Equation
\eqref{EQVOLFORMCHAINRULE} really is the chain rule in disguise\footnote{ 
Hold the phone!  There's something strange going on.  The derivative
of a function can take negative values but by definition a volume form must take only
positive values.  So, on the surface $w'=V_{w}$ can only hold for special parameterizations
$w:\R\to\R$.  This is true, and this is all fixed in the land of differential forms
where volumes and areas are allowed to be \emph{signed}.}.  

While we're still in the mood to look at Equation \eqref{EQVOLFORMCHAINRULE}, let's
make one more observation.  We could interpret the symbol ``$V$'' as an operator
that inputs a function and outputs a new function.  Namely,
if $\vec w:\R^n\to\R^n$, then $V_{\vec w}:\R^n\to\R$ is a function associated
with $\vec w$.  Let's choose a more evocative notation for ``$V$''.  Define and alternate notation
for ``$V$'' by
\[
	V_{\vec w}(\vec x) = \frac{\d \vec w}{\d \vec x}.
\]
In this notation, Equation \eqref{EQVOLFORMCHAINRULE} says
\[
	\frac{\d \vec s}{\d \vec x} = \frac{\d\vec p}{\d \vec x}\,\frac{\d\vec q}{\d \vec p}.
\]
Now this really looks like the chain rule! At this point, using a notation that looks like
derivative notation is mostly whimsy---when $\vec w:\R^n\to\R^n$ and $n>1$,
we don't even know what the derivative of $\vec w$ should mean---but 
in the field of differential geometry, such notation can be justified.


Enough whimsy.  Let's use the chain rule for volume forms to solve a problem.
\begin{example}
	XXX Finish.  Maybe an isometric rotation and a stretching of some sort?
\end{example}

\begin{exercises}
\end{exercises}

\section{Surface Integrals}

\begin{exercises}
\end{exercises}

	\clearpage
\chapter{Vector Fields}
	A \emph{scalar field}\index{scalar field}
is a function that assigns each point in space a
scalar value.  For example, given latitude and longitude $(x,y)$,
the function that outputs the elevation of the surface of the earth
at position $(x,y)$ is a scalar field.  However,
we might ask for a function that when given $(x,y)$ returns
a vector (maybe the velocity of the wind at that point on the earth).
Functions that assign a vector to each point in space
are called \emph{vector fields}\index{vector field}.

Vector fields are key in physics because they model
vector quantities that are non-constant with respect to
space.  For example, force is a vector!  The force due to of gravity
around a point mass is always directed towards the origin---that
means, as you move in space, the direction of the force 
changes.  Suppose there is a point mass at the origin.
We might model the force\footnote{ Despite what is peddled in
Star Wars, this is a real \emph{force field}.} due to this point mass acting on a particle
at position $(x,y,z)$ with the vector field
\[
	\vec F(x,y,z) = \tfrac{1}{(x^2+y^2+z^2)^{3/2}}\mat{-x\\-y\\-z}.
\]
Indeed, the vector $\vec F(x,y,z)$ always points towards the origin,
and $\norm{\vec F(x,y,z)}=\frac{1}{x^2+y^2+z^2}$ is $1/d^2$ where $d$
is the distance from the origin.


\subsection{Notation}
Notationally, $\vec F:\R^n\to\R^m$ indicates that $\vec F$ inputs points or vectors in $\R^n$
and outputs points or vectors in $\R^m$.  Oftentimes in physics $n=m$, but doesn't need to.

When defining a vector field, sometimes it's convenient to use components
and sometimes it's easier to work with vectors.  For example, the force field
$\vec F:\R^2\to\R^3$ above was defined in terms of components as
\[
	\vec F(x,y,z) = \tfrac{1}{(x^2+y^2+z^2)^{3/2}}\mat{-x\\-y\\-z},
\]
but we could have equivalently defined it using vectors by the equation
\[
	\vec F(\vec v) = -\frac{\vec v}{\norm{\vec v}^3}.
\]


\section{Graphing Vector Fields}

Traditionally, a function $f:\R\to\R$ is graphed on a coordinate
plane by picking one axis to be the domain, one
axis to be the range, and then putting a point at each pair $(x,f(x))$.
For a function $g:\R^2\to\R$, we do something similar, except, this time,
the domain is a plane.  But, no matter, we arrange a third axis orthogonal
to a plane and graph points $(x,y,f(x,y))$ in this three-dimensional space.

With vector fields, we cannot graph in the same way.  A vector field
$\vec F:\R^2\to\R^2$ inputs vectors in a plane and outputs vectors in
a plane. To traditionally ``graph'' this function, we would need to
plot points in $\R^2\times\R^2=\R^4$.  Four-dimensional plots are hard to make!
Fortunately, the goal of graphing is rarely to make a faithful representation
of a function---it is to make a visual representation of a function
to aid our thinking.  With this perspective, representing four-dimensional
graphs won't be hard.

If we think of $\vec F:\R^2\to\R^2$ as attaching a vector to each point in
$\R^2$, we might graph $\vec F$ by drawing vectors at each point in $\R^2$.

XXX Figure

This type of plot is called a \emph{quiver plot}\index{quiver plot},
and it will be the main way we visualize vector fields.

\begin{example}
	Graph the vector field $\vec F:\R^2\to\R^2$ defined by $\vec F(x,y)=\mat{x\\y}$.

	XXX Finish
\end{example}

When making a quiver plot, there are two things to consider: (i) how
densely do you draw vectors, and (ii) how do you scale the vectors?
The answer to these two questions depends on the situation---remember,
the goal of plotting a vector field is to visualize what is happening.

Consider the vector field $\vec A:\R^2\to\R^2$ defined by $\vec A(x,y)=\mat{-y\\x}$.
If we plot vectors every $0.5$ units and plot the vectors at the same scale
as the domain, we get the following, hard to follow, figure.

XXX Figure

However, if we plot vectors every $0.5$ units but scale the vectors so they are
$1/4$ the size they'd conventionally be, we get a much more understandable figure.

XXX Figure

We might even decide that the magnitude of each vector is not so important
and totally ignore it in our plot.

XXX Figure with unit vectors


\begin{exercises}
\end{exercises}

\section{The Gradient}

We already have familiarity with one class of vector fields---gradients.
Given a differentiable function $f:\R^n\to\R$, we can always consider the
vector field $\nabla f:\R^n\to\R^n$.  And, $\nabla f$ has some nice geometric
properties.  For instance, $\nabla f$ always points in the direction
of greatest change for $f$.

XXX Figure surface with vector field at base

The connections between $f$ and $\nabla f$ run deep.  Suppose
for a moment that $f:\R^2\to\R$ is affine.  That is, $f(x,y)=\alpha x+\beta y+\gamma$. 
Now, $\nabla f(x,y)=\mat{\alpha\\\beta}$ is
constant.  If $\vec a,\vec b\in\R^2$, we can consider the change in height
of $f$ when moving from $\vec a$ to $\vec b$.  This is,
\[
	f(\vec b)-f(\vec a) = \alpha b_x+\beta b_y - \alpha a_x - \beta a_y
	=\nabla f(\vec a)\cdot (\vec b-\vec a).
\]
Of course this is no surprise---we first encountered
gradients when finding affine approximations of
functions.  Baked into the very core of the gradient is the fact that for a 
differentiable function $h:\R^n\to\R$,
\[
	h(\vec b)-h(\vec a)=\text{change in }h\text{ from }\vec a\text{ to }\vec b
	\approx \nabla h(\vec a)\cdot (\vec b-\vec a),
\]
so long as $\vec a$ and $\vec b$ are close together.  Now, let's apply
some calculus reasoning.

Let $h:\R^n\to\R$ be a differentiable function and let $\vec a,\vec b\in\R^n$
be points that \emph{aren't} close to each other.  Let $\vec a=\vec x_1,\vec x_2,\ldots,\vec
x_m=\vec b$ be a sequence of points so that $\norm{\vec x_{k+1}-\vec x_k}$ is small.
We may think of $\vec x_i$ as a sequence of tiny steps used to get from $\vec a$ to
$\vec b$.  Now, we have
\[
	h(\vec b)-h(\vec a)=\text{change in }h\text{ from }\vec a\text{ to }\vec b
	\approx 
	\sum \nabla h(\vec x_k)\cdot (\vec x_{k+1}-\vec x_k).
\]
If we define $\Delta \vec x_k=\vec x_{k+1}-\vec x_k$, we get an eerily familiar
formula:
\[
	h(\vec b)-h(\vec a)
	\approx 
	\sum \nabla h(\vec x_k)\cdot \Delta \vec x_k.
\]
On the left we have the scalar quantity $h(\vec b)-h(\vec a)$ and on the right
we have a Riemann sum approximation of the vector line integral\index{vector line integral}
of $\nabla h$ along some path starting at $\vec a$ and ending at $\vec b$.  After
taking a limit, this approximation becomes exact.

\begin{theorem}
	\label{THMFTC1}
	Let $h:\R^n\to\R$ be a differentiable function and let
	$\mathcal C\subset \R^n$ be a parameterized, bounded curve 
	which starts at $\vec a\in\R^n$ and ends at $\vec b\in\R^n$.
	Then,
	\[
		h(\vec b)-h(\vec a) = \int_{\mathcal C} \nabla h\cdot \d\vec r,
	\]
	where $\int_{\mathcal C} \nabla h\cdot \d\vec r$ is the vector line
	integral of $\nabla h$ along $\mathcal C$.
\end{theorem}

Theorem \ref{THMFTC1} should remind you of the fundamental theorem of calculus,
which states for a differentiable function $f:\R\to\R$, 
\[
	f(b)-f(a) = \int_a^b f'.
\]
In one dimension, there aren't many ways to get between two points.  In
contrast, there are infinitely many ways to get between two points
in higher dimensions.  Theorem \ref{THMFTC1} says that all these ways
are equivalent, and that $\nabla h$ takes the place of $h'$.

\subsection{Orientation}

As we delve into the realm of vector fields, we need to take note of something
new---orientation\index{orientation}.  In a scalar line integral of $f$ along the curve
$\mathcal C$, we added up the ``height'' of $f$ along the curve $\mathcal C$.
It didn't matter how we traverse $\mathcal C$, and so it
doesn't matter how we parameterize $\mathcal C$ when computing $\int_C f$.  
The area of the graph of $f$ above
$\mathcal C$ depends only on $f$ and $\mathcal C$.  However, things are different for
vector line integrals.

If $\mathcal C$ connects the points $\vec a$ and $\vec b$ and we wish to compute
the amount of work done by the force field $\vec h$ when moving from $\vec a$ to
$\vec b$ along $\mathcal C$, not all parameterizations of $\mathcal C$ will
do.  We must parameterize $\mathcal C$ such that we start at $\vec a$ and
end at $\vec b$.  Thus, the vector integral $\int_{\mathcal C} \vec h\cdot \d\vec r$
depends not only on $\vec h$ and $\mathcal C$, but also on the direction
or \emph{orientation}\index{oriented curve} of $\mathcal C$.

For curves, there are only two orientations, forwards and backwards,
and reversing the orientation of a curve in a vector line integral just multiplies
the result by $-1$.

\begin{example}
	XXX simple example with orientation flipped
\end{example}


The concept of orientation is familiar from single-variable calculus.
After all,
\[
	\int_a^b f=-\int_b^a f.
\]

From now on, if it matters, we will always specify a curve with
its orientation.  If the curve is closed, we will specify its orientation
as \emph{clockwise} or \emph{counterclockwise}.

\subsection{Conservative Vector Fields}

Theorem \ref{THMFTC1} shows that gradients and vector line integrals
interact nicely.  So nicely, in fact, that vector fields coming 
from gradients have a special name.

\begin{definition}[Conservative Vector Field \& Potential Function]
	A vector field $\vec h:\R^n\to\R^n$ is called \emph{conservative}
	if there is some function $f:\R^n\to\R$ so that $\nabla f=\vec h$.
	Such an $f$ is called a \emph{potential function for $\vec h$}.
\end{definition}

We can think of conservative vector fields\index{conservative vector field}
as those having anti-derivatives,
called \emph{potential functions}\index{potential function}\footnote{
In math, if $\nabla f=\vec h$, we say $f$ is a potential function for $\vec h$.
In physics you say that $-f$ is a potential function for $\vec h$---this aligns
more closely with a physicist's notion of \emph{potential energy}.}.  Once
we have a potential function $f$ for a vector field $\vec h$, 
Theorem \ref{THMFTC1} shows that we can compute vector line integrals
of $\vec h$ by evaluating $f$ on the endpoints of our path.  An immediate consequence
is that the integral around a closed path in a conservative vector field is
zero.

\begin{example}
	XXX Example --- work on a hill via complicated path and sub for
	and easy path.
\end{example}

Since conservative vector fields make vector line integrals so easy, it's useful
to determine whether a given vector field is conservative.  The only sure-fire
way to determine a vector field is conservative is to find a potential function.
However, there are several heuristics for showing that a vector field is not
conservative.

\bigskip

Having established that conservative vector fields are useful,
it would be nice to be able to decide if a given vector field is
conservative.  The only sure-fire way to do so is to produce a 
potential function.  However, there methods we can use to quickly
determine if a particular vector field \emph{cannot} be conservative.

The \emph{screening test}\index{screening test} 
is one such method.  Suppose that $\vec F=\mat{F_x\\F_y}$
and that $\vec F$ is conservative.  That means $\vec F=\nabla f$
for some potential function $f:\R^2\to\R$, and so
\[
	F_x = \frac{\partial f}{\partial x}\qquad\text{and}\qquad 
	F_y=\frac{\partial f}{\partial y}.
\]

If $\vec F$ is a \emph{nice} vector field (that is, it comes from a sufficiently differentiable
$f$), then
\[
	\frac{\partial F_x}{\partial y} = 
	\frac{\partial^2 f}{\partial y\partial x} = 
	\frac{\partial^2 f}{\partial x\partial y} = 
	\frac{\partial F_y}{\partial x}. 
\]

This means that if $\frac{\partial F_x}{\partial y} \neq \frac{\partial F_y}{\partial x}$,
we can conclude that $\vec F$ is \emph{not} conservative.

\begin{example}
	Is the vector field $\vec F(x,y)=\mat{x+1\\y+x^2}$ conservative?

	Define $F_x(x,y)=x+1$ and $F_y(x,y)=y+x^2$ so that $\vec F  =\mat{F_x\\F_y}$.
	Now,
	\[
		\frac{\partial F_x}{\partial y} = 0
		\neq 2x = 
		\frac{\partial F_y}{\partial x},
	\]
	and so $\vec F$ is not conservative.
\end{example}

It should be noted that the screening test can only prove vector fields
are not conservative.  It cannot be used to show that they are!

\begin{example}
	Is the vector field
	$\vec F(x,y) = \mat{\frac{-y}{x^2+y^2}\\[4pt]\frac{x}{x^2+y^2}}$ conservative?

	Defining $F_x(x,y)=\frac{-y}{x^2+y^2}$ and $F_y(x,y)=\frac{x}{x^2+y^2}$
	so that $\vec F = \mat{F_x,F_y}$, we compute
	\[
		\frac{\partial F_x}{\partial y} = \frac{y^2-x^2}{(x^2+y^2)^2}
		= 
		\frac{\partial F_y}{\partial x},
	\]
	and so the screening test cannot be used to conclude anything
	about whether or not $\vec F$ is conservative.

	
	However, letting $\mathcal C$ be the circle of radius 1 centered at the
	origin and oriented counter clockwise, we compute $\int_{\mathcal C} \vec F\cdot \d \vec r
	=2\pi$.  If $\vec F$ were a conservative vector field, the 
	vector line integral around any closed path would be $0$.  Thus, we conclude
	that $\vec F$ is not conservative.
	
\end{example}

The screening test works analogously for higher-dimensional vector fields.  One just
needs to test all combinations of mixed partials---if any combination fails, the vector
field is not conservative.

\begin{exercises}
\end{exercises}


\section{Flux and Divergence}

Suppose that the vector field $\vec F(x,y,z)=(4,0,0)$
describes the velocity of water in a river of uniform density.
You cast a $1\times 1$ square net in the river with normal
vector $(1,0,0)$ and ask the following question: how much 
water per unit time is flowing through the net?

XXX Figure

In this situation, it is easy to compute.  Let us assume
the density of the water is $1$.  In this case, since the direction
of flow is orthogonal to the net (because the normal vector is parallel
to the direction of flow), we know
\[
	\text{mass}/\text{unit time} = (\text{net area})
	(\text{water speed})=(1)(4)=4.
\]

If we angle the net slightly, so that it has a normal vector
$(1,1,0)$, computing the amount of water that goes through
the net in a time unit is still not so bad.
\[
	\text{mass}/\text{unit time} = (\text{perceived net area})
	(\text{water speed}).
\]
Here, the \emph{perceived net area} is the area that the net appears to
the water.  Since the net is angled at $45^\circ$ relative to the flow
of the water, the net \emph{appears} smaller, from the perspective of the water.

XXX Figure showing ``apparent'' height

To the water, the net appears to have width $1$ and height $\sqrt{2}/2$, giving
the net a perceived area of $\sqrt{2}/2$.  Now we can compute
\[
	\text{mass}/\text{unit time} = (\text{perceived net area})
	(\text{water speed}) =\tfrac{\sqrt 2}{2}(4)=2\sqrt{2}.
\]

Equivalently, we could have thought about water from the perspective of the net.
We know
\[
	\text{mass}/\text{unit time} = (\text{net area})
	(\text{water speed normal to net}),
\]
and computing the velocity of water normal to the net can be done with a dot product.
In particular,
\[
	\text{water speed normal to net}=\Proj_{\vec n}\vec v = \mat{2\\2\\0},
\]
where $\vec n=(1,1,0)$ is the normal vector for the net and $\vec v=(4,0,0)$
is the velocity vector of the water.  Now
\begin{align*}
	\text{mass}/\text{unit time} &= (\text{net area})
	(\text{water speed normal to net}) \\
	&= (1)\norm*{\mat{2\\2\\0}}=2\sqrt{2},
\end{align*}
which is the same quantity we computed before.

Measuring how much a vector field ``flows'' through a surface $\mathcal S$
is called the \emph{flux}\index{flux}\footnote{
The term \emph{flux} comes from \emph{fluxus},
which means ``flow'' in Latin.
} of the vector field through $\mathcal S$, and comes up in applications
ranging from fluid mechanics to electro-magnetism and relativity gravity.

\subsection{Computing Flux}

Given constant vector fields and flat surfaces, flux is not difficult to compute.
However, the vector fields we encounter in nature are rarely constant and
the surfaces are rarely flat.  But, the ideas of calculus are here
to help us!

By now, we are familiar with the idea of chopping things up into little pieces,
approximating each piece, adding them together again, and taking a limit.  Finding
the flux of a vector field through an arbitrary surface follows this pattern.

Let $\vec F:\R^3\to\R^3$ be a smooth vector field and let $\mathcal S$ be 
a surface\footnote{ An orientable surface, to be precise.}.
Let $\Delta A$ represent a tiny patch of the surface $\mathcal S$.  We can then say
\[
	\Flux_{\mathcal S} \vec F = \sum_{\Delta A} \Flux_{\Delta A} \vec F,
\]
where $\Flux_A \vec B$ means the flux of the vector field $\vec B$ through the
surface $A$.  This looks a lot like a setup for a surface integral!  Let's make
the connection exact.

Let $\vec p:\R^2\to\mathcal S$ be a parameterization of the surface $\mathcal S$,
and let $\Delta A_{(x_0,y_0)}=\vec p([x_0,x_0+\Delta x]\times [y_0,y_0+\Delta y])$
be a tiny sector of $\mathcal S$ with ``lower-left corner'' at $\vec p(x_0,y_0)$.
As long as $\Delta x$ and $\Delta y$ are small, $\Delta A_{(x_0,y_0)}$ will be
small.

Now, let's approximate.  Since $\Delta A_{(x_0,y_0)}$
is small, we could approximate $\vec F$ near $\Delta A_{(x_0,y_0)}$
by the constant vector 
field $\vec F_{(x_0,y_0)}(x,y) = \vec F\circ \vec p(x_0,y_0)$.  Further,
if $\Delta x$ and $\Delta y$ are small enough, we can approximate $\Delta A_{(x_0,y_0)}$
by a tiny parallelogram coming from the 
canonical parameterization of the tangent plane to $\mathcal S$ at $\vec p(x_0,y_0)$.

Recall, given the parameterization $\vec p$,
the canonical parameterization of the tangent plane to $\mathcal S$ at $\vec p(x_0,y_0)$
is
\[
	\vec P(t,s) = t \frac{\partial \vec p}{\partial x}(x_0,y_0) + 
	s\frac{\partial \vec p}{\partial y}(x_0,y_0)
	+\vec p(x_0,y_0).
\]
Since $\vec P$ is a canonical parameterization, if $\Delta x$ and $\Delta y$
are small, then 
\[
\Delta A_{(x_0,y_0)}=\vec p([x_0,x_0+\Delta x]\times [y_0,y_0+\Delta y])
\approx \vec P([x_0,x_0+\Delta x]\times [y_0,y_0+\Delta y]).
\]
Further, $\vec P([x_0,x_0+\Delta x]\times [y_0,y_0+\Delta y])$ is a parallelogram
with normal vector
\[
	\vec n = \frac{\partial \vec p}{\partial x}(x_0,y_0) \times
	\frac{\partial \vec p}{\partial y}(x_0,y_0)
\]
and area $\norm{\vec n}$.

We have developed several approximations.  Before we put them all together, let's
give them simple names.  Fix $x_0$ and $y_0$,
and let $\Delta A=\Delta A_{x_0,y_0}$;
let $Q(x,y) = \vec F\circ \vec p(x_0,y_0)$ be a constant
vector field approximating $\vec F$; and, let 
$\Delta R=\vec P([x_0,x_0+\Delta x]\times [y_0,y_0+\Delta y])$ be a parallelogram
approximating $\Delta A$.  We then have, so long as $\Delta x$ and $\Delta y$ are small
\begin{align*}
	\Delta A &\approx \Delta R\\
	\vec F &\approx \vec Q.
\end{align*}
Putting this all together, 
\begin{align*}
	\Flux_{\Delta A}\vec F
	\approx \Flux_{\Delta A}\vec Q 
	\approx \Flux_{\Delta R}\vec Q
\end{align*}

Now, $\tfrac{\vec n}{\norm{\vec n}}$ is a unit normal vector to $\Delta R$,
and so
\[
	\Flux_{\Delta R}\vec Q = 
	\norm{\vec n}\tfrac{\vec n}{\norm{\vec n}}\cdot \vec Q
	=\vec n\cdot \vec Q.
\]
Substituting in for $\vec n$ and $\vec Q$, we have
\begin{equation}
	\label{EQFLUXAPPROX}
	\Flux_{\Delta A}\vec F
	\approx 
	\left(\frac{\partial \vec p}{\partial x}(x_0,y_0) \times
	\frac{\partial \vec p}{\partial y}(x_0,y_0)\right)
	\cdot \Big(\vec F\circ \vec p(x_0,y_0)\Big ).
\end{equation}

We worked pretty hard to get Equation \eqref{EQFLUXAPPROX}, but
let's remind ourselves that the idea was simple.  We approximated
a changing vector field by a constant one and a curvy shape by a flat one.
The rest of the formula comes from exploiting the relationship between
the cross product and the area of a parallelogram and the computation
of the flux of a constant vector field through a flat surface.





\section{Circulation and Curl}

	\clearpage

\appendix
\renewcommand{\printchaptername}{\chapnumfont Appendix}
\chapter{Proofs}
	\label{APPENDIX-proofstyle}
	\input{chapters/appendix-proofstyle.tex}

\chapter{Creative Commons License Fulltext}
	\label{APPENDIX-license}
	\input{resources/doclicense-CC-by-sa-4.0-latex.tex}

\backmatter

%%% BIBLIOGRAPHY
%%% -------------------------------------------------------------

% \bibliographystyle{utphysics}
% \bibliography{ref}

\end{document}
