\section{Mathematical Notation}
	Mathematics is a sophisticated and precise language, and
	we best not adventure into calculus without learning some
	basic words.

	The most basic mathematical word is that of a \emph{set}\index{set}.  
	A set is an unordered collection of distinct objects.  We won't try and pin
	it down more exactly than this---our intuition about collections
	of objects will suffice\footnote{ When you pursue more rigorous math,
	you rely on definitions to get yourself out of philosophical jams.  For instance,
	with our definition of set, consider ``the set of all sets that don't
	contain themselves.''  Such a set cannot exist!
	This is called \emph{Russel's Paradox}, and shows
	that if we start talking about sets of sets, we may need more than
	intuition.}. We write a set with curly-braces $\{$ and $\}$ and
	list the objects inside.  For instance
	\[
		\Set{1,2,3}.
	\]
	This would be read aloud as ``the set containing the elements $1$, $2$, and $3$.''
	The symbol $\in$\index{$\in$} is used to specify that some object is an element of a set, and
	$\notin$ is used to specify it is not.  For example,
	\[
		3\in\Set{1,2,3}\qquad 4\notin\Set{1,2,3}.
	\]
	Sets can contain mixtures of objects, including other sets.  For example,
	\[
		\Set{1,2,a,\Set{-70,\infty}, x}
	\]
	is a perfectly valid set.

	It is tradition to use capital letters to name sets.  So we might say $A=\{6,7,12\}$
	or $X=\{7\}$.  There is, however, a special set with a special name---the
	empty set.  The \emph{empty set} is the set containing no elements
	and is written $\emptyset$ or $\Set{}$.  Note that $\Set{\emptyset}$ is \emph{not}
	the empty set.  It is the set containing the empty set!  It is also traditional
	to call elements of a set \emph{points}\index{point} regardless of whether you
	consider them ``point-like'' objects.

	\subsection{Operations on Sets}
	If the set $A$ contains all the elements that the set $B$ does, we call $A$ a \emph{superset}\index{superset}
	and $B$ a \emph{subset}\index{subset}.  We'll give this a formal
	definition.
	\begin{definition}[Subset \& Superset]
		The set $B$ is a \emph{subset} of the set $A$, written $B\subseteq A$, if for all
		$b\in B$ we also have $b\in A$.  In this case, $A$ is called a \emph{superset}
		of $B$.\footnote{
			Some mathematicians use the symbol $\subset$ instead of $\subseteq$.}
	\end{definition}

	Some simple examples are $\Set{1,2,3}\subseteq \Set{1,2,3,4}$ and $\Set{1,2,3}\subseteq\Set{1,2,3}$.
	There's something funny about that last example, though.  Those two sets are not only subsets/supersets
	of each other, they're \emph{equal}.  As surprising as it seems, we actually need to define
	what it means for two sets to be equal.
	\begin{definition}[Set Equality]
		The sets $A$ and $B$ are \emph{equal}, written $A=B$, if $A\subseteq B$ and $B\subseteq A$.
	\end{definition}
	Having a definition of equality to lean on will help us when we need to prove things about sets.

	\begin{example}
		Let $A$ be the set of numbers that can be expressed
		as $2n$ for some whole number $n$, and let $B$ be the
		set of numbers that can be expressed as $m+1$ where $m$ is
		an odd whole number.  We will show $A=B$.

		First, let us show $A\subseteq B$.  If $x\in A$ then $x=2n$
		for some whole number $n$.  Therefore $x=2n=2(n-1)+1+1=m+1$ where
		$m=2(n-1)+1$ is, by definition, an odd number.  Therefore $x\in B$.

		Now we will show $B\subseteq A$.  Let $x\in B$.  By definition,
		$x=m+1$ for some odd $m$ and so by the definition of oddness, $m=2k+1$
		for some whole number $k$.  Thus 
		\begin{align*}
			x=m+1&=(2k+1)+1=2k+2\\
			&=2(k+1)=2n,
		\end{align*} where $n=k+1$, and so $x\in A$.  Since $A\subseteq B$
		and $B\subseteq A$, by definition $A=B$.
	\end{example}
	

	\subsubsection{Set-builder Notation}
	Specifying sets by listing all their elements can be a hassle, and if there are an infinite
	number of elements, it's impossible!  Fortunately, \emph{set-builder notation}\index{set-builder notation}
	solves these problems.
	If $X$ is a set, we can define a subset 
	\[
		Y= \Set{a\in X\given\text{some rule involving }a},
	\]
	which is read ``$Y$ is the set of $a$ in $X$ \emph{such that} some rule
	involving $a$ is true.''  If $X$ is intuitive, we may omit it and
	simply write $Y=\Set{a\given\text{some rule involving }a}$\footnote{ If you want
	to get technical, to make this notation unambiguous, you define a 
	\emph{universe of discourse}.  That is, a set $\mathcal U$ containing
	every object you might want to talk about.  Then $\Set{a\given\text{some rule involving }a}$
	is short for $\Set{a\in\mathcal U\given\text{some rule involving }a}$}.  You may equivalently
	use ``$|$'' instead of ``$:$'', writing $Y=\{a\,|\,\text{some rule involving }a\}$.

	\begin{example}
		The set $\Z$ is the set of integers (positive, negative,
		and zero whole numbers).  To define $E$ as the even integers,
		we could write
		\[
			E=\Set{n\in \Z\given n=2k\text{ for some }k\in \Z}.
		\]
		To define $P$ as the set of positive integers, we could write
		\[
			P=\Set{n\in\Z\given n>0}.
		\]
	\end{example}


	There are also some common operations we can do with two sets.
	\begin{definition}[Intersections \& Unions]
		Let $A$ and $B$ be sets. Then the \emph{intersection} of $A$ and $B$, written
		$A\cap B$, is defined by
		\[
			A\cap B=\Set{x\given x\in A\text{ and }x\in B}.
		\]
		The \emph{union} of $A$ and $B$, written $A\cup B$, is defined by
		\[
			A\cup B= \Set{x\given x\in A\text{ or } x\in B}.
		\]
	\end{definition}
	For example, if $A=\Set{1,2,3}$ and $B=\Set{-1,0,1,2}$, then $A\cap B=\Set{1,2}$ and $A\cup B=
	\Set{-1,0,1,2,3}$.  Set unions and intersections are \emph{associative}, which means it doesn't
	matter how you apply parentheses to an expression involving just unions or just intersections.
	For example $(A\cup B)\cup C=A\cup(B\cup C)$, which means
	we can give an unambiguous meaning to an expression like $A\cup B\cup C$ (just put
	the parentheses wherever you like).  But watch out, $(A\cup B)\cap C$ means something
	different than $A\cup(B\cap C)$!

	\begin{definition}[Set Subtraction]
		For sets $A$ and $B$, the \emph{set-wise difference}\index{set subtraction} between $A$ and $B$,
		written $A\backslash B$, is the set
		\[
			A\backslash B = \Set{x\given x\in A\text{ and }x\notin B}.
		\]
	\end{definition}
	\begin{definition}[Cardinality]
		For a set $A$, the \emph{cardinality}\index{cardinality} of $A$,
		written $\abs{A}$ is the number of elements in $A$.  If $A$
		contains infinitely many elements, we write $\abs{A}=\infty$.
	\end{definition}

	Let's define some notation for common sets.
	\begin{align*}
		\emptyset &= \Set{}\text{, the empty set}\\
		\N &= \Set{0,1,2,3,\ldots}=\Set{\text{natural numbers}}\\
		\Z &= \Set{\ldots, -3,-2,-1,0,1,2,3,\ldots}=\Set{\text{integers}}\\
		\Q &= \Set{\text{rational numbers}}\\
		\R &= \Set{\text{real numbers}}\\
		\R^n &= \Set{\text{vectors in $n$-dimensional Euclidean space}}\\
	\end{align*}

	Besides unions, there's another way to join sets together:
	\emph{products}\index{Cartesian product}\index{set product}.
	\begin{definition}[Cartesian Product]
		Given two sets $A$ and $B$, the \emph{Cartesian product} (sometimes
		shortened to \emph{product}) of the sets $A$ and $B$ is written
		$A\times B$ and defined to be
		\[
			A\times B = \Set{(a,b)\given a\in A\text{ and }b\in B}.
		\]
	\end{definition}
	The Cartesian product of two sets is the set of all ordered pairs of elements from
	those sets.  For example, 
	\[
		\Set{1,2}\times \Set{1,2,3}=\Set{(1,1),(1,2),(1,3),(2,1),(2,2),(2,3)}.
	\]
	You can repeat this operation more than once.   $\R\times \R\times \R$
	is the set of all triples of real numbers.  Extending power notation
	notation, if you take the Cartesian product of a set
	with itself some number of times, you can represent it with
	an exponent.  Thus, $\R\times \R\times \R$ can be written as $\R^3$, which is a set we've
	seen before%
	\footnote{
		If you're scratching your head saying, ``I thought $\R^3$ was vectors in $3$-dimensional
		space.  How do we know that's the same thing as triples of real numbers?'' your mind
		is keen.  This is a theorem of linear algebra.
	}.

	\subsection{Functions}
	You're probably used to seeing functions like $f(x)=x^2$, but it's worth reviewing some of the concepts
	and terminology associated with functions.

	\begin{definition}[Function]
		A \emph{function}\index{function} with \emph{domain}\index{domain} the 
		set $A$ and \emph{co-domain}\index{co-domain} the set $B$ is an object that
		associates every point in the set $A$ with \emph{exactly one} point in the set $B$.
	\end{definition}

	If a function $f$ has domain $A$ and co-domain\footnote{ Some
	people use the word \emph{range} interchangeably with co-domain.} $B$,
	we notate this by writing $f:A\to B$.
	If we want to further specify what the function $f$ actually is, we need to
	express how $f$ associates each point in $A$ to a point in $B$.  This can be done
	with an equation.  For example, we could define the function $f:\R\to\R$ by
	\[
		f(x)=2x,
	\]
	which says that each real number gets associated to its double.  We can notate
	the same thing using a special type of arrow: ``$\mapsto$''.  Now we might write
	\[
		f:\R\to\R\text{ where } x\mapsto 2x,
	\]
	which is read ``$f$ is a function from $\R$ to $\R$ where $x\in \R$ gets mapped to $2x$.''

	
	Note that every point in the co-domain of a function doesn't need to get mapped
	to.  For example $g:\R\to\R$ given by $g(x)=x^2$ outputs only non-negative numbers,
	but it is still valid to specify $\R$ as the co-domain.  However, if we wanted
	to make a point of it, we are perfectly justified in writing $g:\R\to[0,\infty)$
	when defining $g$.

	Many common math operations give rise to functions.  For example,  
	$f(x)=\sqrt{x}$ is the familiar
	square root function.  Sometimes, when we wish to talk about a function
	for which notation already exists, we will put a ``$\ \cdot\ $'' where we would
	normally put a variable.  Thus, we might say, ``$\sqrt{\:\cdot\:}$ is the square
	root function.''\footnote{ Since $\sqrt{x}$ is ``the square root of the quantity
	$x$,'' it is technically a quantity and not a function.  This is why we write $\ \cdot\ $ instead
	of $x$ when we want to refer to the square root \emph{function}.
	}

	\begin{definition}[Range]
		The \emph{range}\index{range} of a function $f:A\to B$
		is the set of all outputs of $f$.  That is
		\[
			\Range f = \Set{y\in B\given y=f(x)\text{ for some }x\in A}.
		\]
	\end{definition}
	\begin{definition}[Image]
		Let $f:A\to B$ be a function.
		The \emph{image}\index{image} of a set $X\subseteq A$, written $f(X)$ is
		defined by
		\[
			f(X)=\Set{y\in B\given y=f(x)\text{ for some }x\in X}.
		\]
	\end{definition}

	We see that if $f:A\to B$, $\Range f = f(A)$.  In words, the range of $f$ is the image
	of its domain.  This language will become useful when we think of functions as transformations
	that move or bend space.  If $f:\R^2\to\R^2$ is a function that warps the Cartesian plane,
	then the image of $X$ under $f$ could be visualized by painting $X$ on the Cartesian plane,
	warping the whole plane, and then looking at the resulting, painted shape.  


	Closely related to images, we have the idea of \emph{restriction}\index{restriction}.
	Suppose $f:\R^2\to\R$ is defined by $f(x,y)=xy$, but we were only really interested in $f$ on the unit circle, $\mathcal C$.
	In this case, we might say $f$ attains a maximum on $\mathcal C$, or $f$ 
	\emph{restricted to} $\mathcal C$ attains a maximum, even though $f$ itself is unbounded.
	This idea comes up often enough to deserve its own notation.

	\begin{definition}[Restriction]
		If $f:A\to B$ and $X\subseteq A$, the \emph{restriction}\index{restriction}
		of $f$ to $X$ is written 
		$f\big|_X$ and is defined to be the function $g:X\to A$ where $x\mapsto f(x)$.
	\end{definition}

	The last important function-related ideas for us are function composition and inverses.
	Given two functions $f:A\to B$ and $g:B\to C$, we can \emph{compose}\index{function composition}
	$g$ and $f$ to get a new function.
	\begin{definition}[Composition]
		Given two functions $f:A\to B$ and $g:B\to C$, the \emph{composition} of
		$g$ and $f$, written $g\circ f$, is the function $h:A\to C$ where 
		$x\mapsto g(f(x))$.
	\end{definition}
	Note that the composition $g\circ f$ has the domain of $f$ and the co-domain of $g$.
	When a point is fed into $g\circ f$, it moves from $A\to B\to C$.  The composition
	$g\circ f$ only makes sense because the outputs of $f$ are allowed as inputs to $g$.
	If we wrote $f\circ g$, it wouldn't mean much, because $g$ outputs points in $C$ and $f$
	has no idea what to do with points in $C$.\footnote{
		It seems a little backward to write $f:A\to B$, $g:B\to C$ and then
		write $g\circ f$ instead of $f\circ g$.  You can thank Euler for that.
		He decided to write functions with their input on the right instead of
		the left.  If we wrote functions backwards, like $((x)f)g$ for ``$g$ of $f$ of $x$,''
		they we could just \emph{follow the arrows} and life would be simpler.
	}

	Inverses relate to composition and the \emph{identity function}\index{identity function}, the function
	that does nothing to its inputs.
	\begin{definition}[Identity Function]
		The \emph{identity function} $\Id:A\to A$ is defined by the relation
		\[
			\Id(x)=x
		\]
		for all $x\in A$.
	\end{definition}
	Notice that for every set, that set
	is the domain of an identity function.  Since the domain
	and co-domain of a function are part of its definition, we don't want to confuse them.
	After all, $f:\Set{0,1}\to \Set{0,1}$ given by $f(x)=x^2$ is a different function from $f:\R\to\R$
	given by $f(x)=x^2$.  For the special case of the identity function, we sometimes write
	the domain of the function as a subscript.  That is, 
	for $\Id:A\to A$ we'd write $\Id_A$ so it
	doesn't get confused with $\Id:B\to B$, which we'd write $\Id_B$.

	\begin{definition}[Inverse Function]
		Let $f:A\to B$ be a function.  If there exists a function $g:B\to A$ such that
		\[
			f\circ g=\Id_B\qquad\text{and}\qquad g\circ f=\Id_A,
		\]
		we say $f$ is \emph{invertible} and we call $g$ the \emph{inverse} of $f$.
		If $f$ is invertible, we notate its inverse by $f^{-1}$.
	\end{definition}

	Inverses can be tricky some times.  For example, consider $f(x)=x^2$ and $g(x)=\sqrt{x}$.
	Here $g\circ f(x)=\sqrt{x^2}=\abs{x}$ and $f\circ g(x) = \sqrt{x}^2=x$.  What's the deal?
	Well, it's all about domains.  $f:\R\to[0,\infty)$ and $g:[0,\infty)\to[0,\infty)$.
	So, the domain of $g\circ f$ is $\R$ and the domain of $f\circ g$ is $[0,\infty)$.  The
	domains are different, and indeed $f$ is not invertible.  However, $g$ \emph{is} invertible,
	and $g^{-1}=f\big|_{[0,\infty)}$.  If we only input non-negative numbers into $f$, then
	$f$ exactly undoes what $g$ did.  This subtle domain trickery can cause us a lot
	of headaches if we're not used to thinking carefully, and many of our favorite functions
	that we're used to calling ``inverse functions'' are actually only inverses when paired with
	specific domains.

\section{Proof}
	Mathematics has the highest standard of proof of any field.  In the Platonic ideal
	of mathematics, we start from some basic assumptions, called \emph{axioms}, that we have
	all agreed upon.  Then from those axioms, using the rules of logic, we deduce \emph{theorems}.
	Every single mathematical statement we make can be traced back from theorem to theorem
	and eventually to our initial axioms.

	This is contrary to other disciplines, like physics.
	In physics, based on observation, we construct
	\emph{laws}.  Laws in physics are like
	axioms in mathematics, but they have an important difference---they
	can be disproven by observation.  A mathematical axiom can never be disproven.  One can
	certainly argue that an axiom is not \emph{useful} or not \emph{interesting}, but you
	cannot say it's \emph{wrong}\footnote{
		There are multiple ways to axiomatize geometry.  In
		Euclidean geometry every pair of lines either coincides, intersects in
		exactly one place, or does not intersect.  In spherical
		geometry, every pair of lines either coincides or intersects in exactly two places.  
		Euclidean geometry is useful when your space looks flat.  Spherical
		geometry is useful when your space is the surface of a sphere (like 
		the Earth).
		Is one of these more \emph{right} than the other?  They're certainly
		contradictory.
	}.
	Of course, as human practitioners, we may misuse logic and be wrong ourselves, but that
	is no fault of the axioms.

	But now, let's deviate from philosophical perfection and visit reality.
	In reality, \emph{mathematics is a human pursuit to understand relationships
	between ideas and their consequences}.  The key there is that \emph{humans} do
	mathematics to \emph{understand} relationships.  If a theorem in math can
	ultimately be reduced to logical statements about axioms, but the argument is
	100000 steps long, it doesn't help a human understand why something is true.
	Instead, a shorter argument that skips over some steps is more useful to us.
	And, indeed, most of our mathematics to date skips over some steps\footnote{
		There are some projects to prove all of mathematics directly from
		the axioms using computer assistance.  They've made progress, but there
		are still theorems in calculus that have not been reduced to the
		axioms.  We believe that they \emph{could be} reduced to the axioms,
		but no one has taken the time to do so.}.

	We call a correct mathematical argument a \emph{proof}.  A proof starts
	from a set of assumptions, and following the rules of logic, arrives at a conclusion.
	Strictly speaking, a proof doesn't need to make sense or show motivation,
	applications, or examples.  It just has to be a sequence of correct logical steps.
	However, for us, as humans studying mathematics, we prove things for two reasons:
	to understand why things are true and to avoid making mistakes.

	Reconciling these two goals can be very hard for a novice mathematician.  If you include
	\emph{all} the steps, it won't help with understanding, but if you don't include enough
	steps, the argument may not be convincing and might contain mistakes.  
	Even professionals struggle to balance
	these competing goals, and how you balance those goals depends on your audience---if you're
	trying to convince your math professor of something your proof will need to have more
	detail than if you were trying to convince your friend (mathematicians are very skeptical!).

	Enough talk, let's go through a 2000-year-old example of a proof.
	\begin{theorem}
		There is no rational number $p/q$ such that $(p/q)^2=2$.
	\end{theorem}
	\begin{proof}
		If $p/q$ is a rational number, it can be expressed
		in lowest terms.
		Suppose $p/q$ is in lowest terms and $(p/q)^2=2$.  Then $p^2=2q^2$ and so $p^2$ is even.  Since
		$p^2$ is even, it must be that $p$ is even, and so by definition,
		$p=2m$ for some integer $m$.  Now,
		\[
			\frac{p^2}{q^2}=\frac{(2m)^2}{q^2}=\frac{4m^2}{q^2}=2,
		\]
		with the last equality following by assumption.  Multiplying both sides by $q^2$
		and dividing by $2$ we arrive at the equation
		\[
			2m^2=q^2,
		\]
		and so $q^2$ is even which means $q$ is even.  By definition, this means $q=2n$ for some integer $n$.
		But now,
		\[
			\frac{p}{q}=\frac{2m}{2n}
		\]
		is not in lowest terms!  This is a contradiction and so it cannot be that $(p/q)^2=2$.
	\end{proof}
	This is nearly identical to the argument the ancient Greeks gave.  It's elegant, beautiful,
	and convincing.  But, if we look closer, it does skip some steps.  For example, it relies on
	the fact that there is such a thing as \emph{lowest terms}.  This is something that would
	need to be proven---a priori, the conclusion of the proof could be that the assumption
	that $p/q$ could be in lowest terms is false.  
	
	You will not, overnight,
	become a master at understanding
	what steps you can leave out and what steps you must show.  However, with feedback,
	you'll get better.
	For a detailed guide on writing good proofs, please see Appendix \ref{APPENDIX-proofstyle}.

