\section{Multiple Integrals}

Fundamentally, an integral is the result of chopping a region up into tiny
pieces, adding all the pieces up again, and taking a limit as the size of
the tiny pieces goes to zero.  Thus far, the domain of this procedure---the
thing we chop into tiny pieces---has been a line or curve.  We will now consider
integrating over multi-dimensional domains.

Consider the motivating example of finding the area of a region in the plane.
Let $\mathcal R\subseteq \R^2$ be the region below the line $y=1$ and above the
curve $y=x^2$.  We wish to find the area of $\mathcal R$.

Using the usual calculus strategy, we will chop $\mathcal R$ up into little rectangles
of width $\Delta x$ and height $\Delta y$.  Then
\[
	\text{area of }\mathcal R \quad\approx \sum_{\text{tiny rectangles}}\text{area of tiny rectangle}
\]
and
\[
	\text{area of }\mathcal R =\lim_{\Delta x,\Delta y\to 0} \sum_{\text{tiny rectangles}}\text{area of tiny rectangle}.
\]

XXX Figure

Using integral notation, we would write
\[
	\text{area of }\mathcal R=\int_{\mathcal R} \d A.
\]
Here $\d A$ represents a ``tiny area,'' the subscript $\mathcal R$ represents the region of integration,
and the integral sign means we're adding things up.  In this case, we're finding area, so $\d A=1\d A$ is
exactly what we're adding up.  In other situations we'll be adding up more complicated functions.

This is all well and good, but how do we actually \emph{find} the area.  To do this, we'll
need to convert $\int_{\mathcal R}\d A$ into a more traditional-looking integral---one that we
know how to evaluate.

Let's write down our sum more carefully.  We need to sum over all tiny rectangles that fit inside
$\mathcal R$.  To do so, we can take a systematic approach: let's sum all the rectangles
in a column first and then sum up all the columns.  The lower left corner of all 
rectangles in a single column share a common $x$-coordinate.  Consider
the column with lower left corner at $(x_0,?)$.  Counting, we see there
are approximately $(1-x_0^2)/\Delta y$ such rectangles.  Further, there are approximately
$2/\Delta x$ columns.  Therefore,
\[
	\text{area of }\mathcal R\approx 
	\sum_{i=1}^{2/\Delta x}\quad\sum_{j=1}^{(1-(-1+i\Delta x)^2)/\Delta y} \Delta y\Delta x.
\]
That sum is really hard to parse, so we'll write it another way.
\[
	\text{area of }\mathcal R\approx
	\sum_{x_0=-1,-1+\Delta x,-1+2\Delta x,\ldots,1}
	\quad
	\sum_{y_0=x_0^2,x_0^2+\Delta y,x_0^2+2\Delta y,1}
	\Delta y\Delta x.
\]
This is still hard to read, but it's looking more like an integral.  The inner sum
is adding up things from $y_0=x_0^2$ to $y_0=1$ and the outer sum is adding up
things from $x_0=-1$ to $x_0=1$.  Upon taking a limit, this directly translates to an
integral, giving
\begin{equation}
	\label{EQITERATEDINT}
	\text{area of }\mathcal R\quad
	=\quad
	\int_{\mathcal R}\d A \quad =\quad \int_{x=-1}^{x=1}\int_{y=x^2}^{y=1} \d y\d x.
\end{equation}
Now, we should take a moment to make sure we understand what we've just written.  The
right side of Equation \eqref{EQITERATEDINT} is an \emph{iterated integral}\index{iterated integral}.
That is,
\[
	\int_{x=-1}^{x=1}\int_{y=x^2}^{y=1} \d y\d x
	\quad 
	=
	\quad \int_{x=-1}^{x=1}\left(\int_{y=x^2}^{y=1} \d y\right)\d x
\]
and so the integral with respect to $y$ \emph{must} be done before the integral with respect to $x$.
To be clear, $\d y$ and $\d x$ are \emph{not} being multiplied.  However, $\Delta y$ and $\Delta x$ 
\emph{were} being multiplied in our sum expression.  What happened?  The answer is some slight of
hand.  The full thought process should look like
\[
	\lim_{\Delta x,\Delta y\to 0} \sum_{x_i}\sum_{y_i} \Delta y\Delta x
	=
	\lim_{\Delta x,\Delta y\to 0} \sum_{x_i}\left(\sum_{y_i} \Delta y\right)\Delta x
	=\int_{x=-1}^{x=1}\left(\int_{y=x^2}^{y=1} \d y\right)\d x.
\]

Now we can evaluate this iterated integral to conclude
\[
	\text{area of }\mathcal R
	\quad
	=\quad
	\int_{\mathcal R}\d A \quad=\quad 
	\int_{x=-1}^{x=1}\left(\int_{y=x^2}^{y=1} \d y\right)\d x \quad=\quad
	\tfrac{4}{3}.
\]
But, there was another way we could have divided up our original sum.  We could have summed
along rows first and then summed up each row.  Working from this approach, we see
\[
	\lim_{\Delta x,\Delta y\to 0} \sum_{y_i}\sum_{x_i} \Delta x\Delta y
	=
	\lim_{\Delta x,\Delta y\to 0} \sum_{y_i}\left(\sum_{x_i} \Delta x\right)\Delta y
	=\int_{y=0}^{y=1}\left(\int_{x=-\sqrt{y}}^{x=\sqrt{y}} \d x\right)\d y.
\]
Computing this integral, we again get $4/3$, as expected.



\subsection{Integral Notation}

\section{The Volume Form}

\section{Surface Integrals}
